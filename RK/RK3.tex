\documentclass[a4paper]{article}
\usepackage[T2A]{fontenc}
\usepackage[utf8]{inputenc}	

\usepackage[english, russian]{babel}
\usepackage{misccorr}
\usepackage{amsmath}
\usepackage{tipa}
\usepackage[left=1cm,right=1cm, top=2cm,bottom=2cm]{geometry}





\begin{document}


\begin{enumerate}
\item[1.] Сформулировать определения случайной величины и функции распределения вероятностей случайной величины. Записать основные свойства функции распределения. \\
\begin{itemize}
\item Случайной величиной называют скалярную функцию $X(\omega)$, заданную на пространстве элементарных исходов, если для любого $x \in \mathbb{R}$ множество $\left\{\omega: \ X(\omega) < x\right\}$ элементарных исходов, удовлетворяющих условию $X(\omega) < x$ является событием.
\item Функцией распределения (вероятностей) случайной величины X называют функцию $F(x)$, значение которой в точке $x$ равно вероятности события $\{X < x\}$ - то есть события, состоящего из только тех элементарных исходов, для которых при $X(\omega) < x: \ F(x) = P\{X < x\}$.
\item Свойства функции распределения:
	\begin{enumerate}
	\item[1)] $o \leqslant F(x) \leqslant 1$
	\item[2)] $F(x_1) \leqslant F(x_2)$, при $x_1 < x_2$; то есть $\Phi$ - неубывающая функция
	\item[3)] $F(-\infty) = \lim\limits_{x \to -\infty} F(x) = 0$; $F(+\infty) = \lim\limits_{x \to +\infty} F(x) = 1$
	\item[4)] $P\{x_1 \leqslant X \leqslant x_2\} = F(x_2) - F(x_1)$
	\item[5)] $F(x) = F(x - 0)$, где $F(x - 0) = \lim\limits_{y \to x - 0} F(y)$; то есть $\Phi$ - непрерывная функция.
	\end{enumerate}
\end{itemize}


\item[2.] Сформулировать определения дискретной случайной величины; понятие ряда распределения. Сформулировать определение непрерывной случайной величины и функции плотности распределения вероятностей. \\
\begin{itemize}
\item Случайной величиной X называют дискретной, если множество её возможных значений конечно или счетно.
\item Рядом распределения (вероятностей) дискретной случайной величины X называют таблицу, состоящую из двух строк: в верхней строке перечислены все возможные значения случайной величины, а в нижней вероятности $p_i = P\{X = x_i\}$ того, что случайная величина принимает эти значения. 
\item Непрерывной называют случайную величину X, функцию распределения которой можно представить в виде $\displaystyle  F(x) = \int\limits_{-\infty}^{x} f(y) dy$.
\item Функцию $f(x)$ называют плотностью распределения вероятностей непрерывной случайной величины X.
\end{itemize}


\item[3.] Сформулировать определение непрерывной случайной величины. Записать основные свойства функции плотности распределения вероятностей непрерывной случайной величины.
\begin{itemize}
\item Непрерывной называют случайную величину X, функцию распределения которой можно представить в виде $\displaystyle  F(x) = \int\limits_{-\infty}^{x} f(y)dy$.
\item Свойства:
	\begin{enumerate}
	\item[1)] $\forall n \ f(n) \geqslant 0$
	\item[2)] $\displaystyle  P\{x_1 \leqslant X < x_2\} = \int\limits_{x_1}^{x_2} f(x)dx$
	\item[3)] $\displaystyle  \int\limits_{-\infty}^{+\infty} f(x)dx = 1$
	\item[4)] $P\{x \leqslant X < x + \triangle x\} \approx f(x) \triangle x$ в точках непрерывности плотности распределения
	\item[5)] $P\{X = x\} = 0$ для любого наперед заданного $x \in \mathbb{R}$
	\end{enumerate}
\end{itemize}


\item[4.] Сформулировать определения случайного вектора и его функции распределения вероятностей. Записать свойства функции распределения двумерного случайного вектора.
\begin{itemize}
\item n-мерным случайные вектором называется совокупность случайных величин $X_1 = X_1(\omega), \ldots, X_n = X_n(\omega)$, заданных на одном и том же вероятностном пространстве $(\Omega, \beta, P)$. Сами случайные величины $X_1, \ldots, X_n$ называют компонентами случайного вектора
\item Функцией распределения n-мерного случайного вектора $F(x_1, \ldots, x_n) = F_{X_1, \ldots, X_n} (x_1, \ldots, x_n)$ называют функцию, значение которой в точке $(x_1, \ldots, x_n) \in \mathbb{R}$ равно вероятности совместного осуществления событий $\{X_1 < x_1\}, \ldots, \{X_n < x_n\}$, то есть $F(x_1, \ldots, x_n) = P\{X_1 < x_1, \ldots, X_n < x_n\}$.
\item Свойства двумерной функции распределения:
	\begin{enumerate}
	\item[1)] $0 \leqslant F(x_1, x_2) \leqslant 1$
	\item[2)] $F(x_1, x_2)$ - неубывающая функция по каждому из аргументов $x_1$ и $x_2$.
	\item[3)] $F(-\infty, x_2) = F(x_1, -\infty) = 0$
	\item[4)] $F(+\infty, +\infty) = 1$ %Проверить корректность формулы
	\item[5)] $P\{a_1 \leqslant X_1 \leqslant b_1, \ a_2 \leqslant X_2 \leqslant b_2\} = F(b_1, b_2) - F(b_1, a_2) - F(a_1, b_2) + F(a_1, a_2)$
	\item[6)] $F(x_1, x_2)$ - непрерывная слева в любой точке $(x_1, x_2) \in \mathbb{R}^2$ по каждому из аргументов $x_1$, $x_2$.
	\item[7)] $F_{X_1,X_2} (x, +\infty) = F_{X_1}(x)$; $F_{X_1,X_2}(+\infty, x) = F_{X_2}(x)$.
	\end{enumerate}
\end{itemize}


\item[5.] Сформулировать определения дискретного случайного вектора, понятие таблицы распределения двумерного случайного вектора. Сформулировать определения непрерывного случайного вектора и его функции плотности распределения вероятностей. \\
\begin{itemize}
\item Двумерный случайный вектор $(X,Y)$ называют дискретным, если каждая из случайных величин X и Y является дискретной. \\
Таблицей распределения двумерного случайного вектора называют таблицу следующего вида: \\
в верхней строке перечислены всевозможные значения $y_1, \ldots, y_j, \ldots, y_n$ случайные величины Y; в левом столбце - значения $x_1, \ldots, x_i, \ldots, x_n$ случайные величины X;
на пересечении $P_x$ и $x_i$ записывается число $p_{xi} = p_{i1} + \ldots + p_{im}$; на пересечении $P_y$ и $y_j$ записывается $p_{y_i} = p_{1j} + \ldots + p_{nj}$.
\item Случайный вектор $(X_1, \ldots, X_n)$ называют непрерывным, если его совместную функцию распределения $F_{X_1, \ldots, X_n} (x_1, \ldots, x_n)$ можно представить в виде сходящегося несобственного интеграла $\displaystyle  F(x_1, \ldots, x_n) = \int\limits_{-\infty}^{x_1} \ldots \int\limits_{-\infty}^{x_n} f(y_1, \ldots, y_n) dy_1 \ldots dy_n$. \\
Функцию $f(x_1, \ldots, x_n)$ называют совместной двумерной плотностью распределения случайных величин $X_1, \ldots, X_n$, либо плотностью распределения случайного вектора $(X_1, \ldots, X_n)$; $f(x_1, \ldots, x_n) = \cfrac{\partial^n F(x_1, \ldots, x_n)}{\partial x_1 \ldots \partial x_n}$.
\end{itemize}


\item[6.] Сформулировать определения непрерывного случайного вектора и его функции плотности распределения вероятностей. Записать основные свойства функции плотности распределения двумерных случайных векторов.
\begin{itemize}
\item Случайный вектор $(X_1, \ldots, X_n)$ называют непрерывным, если его совместную функцию распределения $F_{X_1, \ldots, X_n} (x_1, \ldots, x_n)$ можно представить в виде сходящегося несобственного интеграла $\displaystyle  F(x_1, \ldots, x_n) = \int\limits_{-\infty}^{x_1} \ldots \int\limits_{-\infty}^{x_n} f(y_1, \ldots, y_n) dy_1 \ldots dy_n$. \\
Функцию $f(x_1, \ldots, x_n)$ называют совместной двумерной плотностью распределения случайных величин $X_1, \ldots, X_n$, либо плотностью распределения случайного вектора $(X_1, \ldots, X_n)$; $f(x_1, \ldots, x_n) = \cfrac{\partial^n F(x_1, \ldots, x_n)}{\partial x_1 \ldots \partial x_n}$.
\item Свойства функции плотности двумерных случайных векторов:
	\begin{enumerate}
	\item[1)] $f(x,y) \geqslant 0$
	\item[2)] $\displaystyle  P\{a_1 < X < b_1, \ a_2 < Y < b_2\} = \int\limits_{a_1}^{b_1} dx \int\limits_{a_2}^{b_2} fdy$
	\item[3)] $\displaystyle  \int\limits_{-}^{+} \int\limits_{-}^{+} f(x,y) dxdy = 1$
	\item[4)] $P\{x < X < x + \triangle x, \ y < Y < y + \triangle y\} \cong f(x,y) \triangle x \triangle y$
	\item[5)] $P\{X = x, Y = y\} = 0$
	\item[6)] $\displaystyle  P\left\{(X,Y) \in D \right\} = \iint\limits_{D} f(x,y) dxdy$
	\item[7)] $\displaystyle  f_X(x) = \int\limits_{-}^{+} f_{X,Y} (x,y) dy$
	\item[8)] $\displaystyle  f_Y(y) = \int\limits_{-}^{+} f_{X,Y} (x,y) dy$
	\end{enumerate}
\end{itemize}


\item[7.] Сформулировать определение независимых случайных величин. Сформулировать свойства независимых случайных величин. Сформулировать определение попарно независимых случайных величин, независимых в совокупности. \\
\begin{itemize}
\item Случайные величины X и Y называют независимыми, если совместная функция распределения $F_{XY}(xy)$ является произведением одномерных функций распределения: $F_{XY} (xy) = F_X(x) F_Y(y)$.
\item Случайные величины $X_1 \ldots X_n$, заданные на одном вероятностном пространстве, называются независимыми в совокупности,если $F_{X_1 \ldots X_n} (x_1 \ldots x_n) = F_{X_1}(x_1) \ldots F_{X_n}(x_n)$; независимыми попарно, если $\forall i,j = \overline{1,n}, \ i \neq j$, $X_i$ и $X_j$ независимые.
\item Свойства независимых случайный величин: 
	\begin{enumerate}
	\item[1)] Случайные величины X и Y независимы тогда и только тогда, когда $\forall x,y \in \mathbb{R}$ события $\{X \leqslant x\}, \{Y \leqslant y\}$ независимы
	\item[2)] X и Y независимы $\Leftrightarrow \forall x_1, x_2, y_1, y_2 \in \mathbb{R}$ $\{x_1 \leqslant X \leqslant x_2\}, \{y_1 \leqslant Y \leqslant y_2\}$ независимы
	\item[3)] X и Y независимы $\Leftrightarrow \forall M_1, M_2 \{x \in M_1\}, \{Y \in M_2\}$ независимы, где M - промежутки, либо объединения промежутков
	\item[4)] Если X, Y - дискретные случайные величины, то X,Y независимы $\Leftrightarrow p_{ij} \geqslant p_{x_i} \cdot p_{x_i}$; $P_{ij} = P\{X = x_i, Y = y_j\}$, $p_{x_i} = P\{X = x_i\}$, $p_{y_j} = P\{Y = y_j\}$.
	\item[5)] Если X, Y - непрерывные случайные величины, то они независимы $\Leftrightarrow f(x,y) = f_X(x) f_Y(y)$
	\end{enumerate}
\end{itemize}


\item[8.] Понятие условного распределения. Доказать формулу для вычисления условного ряда распределения одной компоненты двумерного дискретного случайного вектора при условии, что другая компонента приняла определенное значение. Записать формулу для вычисления условной плотности распределения одной компоненты двумерного непрерывного случайного вектора при условии, что другая компонента приняла определенное значение. \\
\begin{itemize}
\item Пусть дан двумерный случайный вектор $(X,Y)$ и известно, что случайная величина Y принимает значение y.
\item Пусть $(X,Y)$ - дискретный случайный вектор; $X \in \{x_1, \ldots, x_n\}, \ Y \in \{y_1, \ldots, y_n\}, \ p_{ij} = P\left\{ (X,Y) = (x_i,y_j) \right\} = P\{X = x_i, Y = y_j\}$. Пусть для некоторого $j \ Y = y_j$; $P\{X = x_i, Y = y_j\} = \cfrac{P\left\{(X,Y) = (x_i,y_j) \right\}}{P\{Y = y_j\}} = \cfrac{p_{ij}}{p_{y_j}}$. Условной вероятностью того, что случайный вектор X примет значение $x_i$ при условии что Y принимает значение $y_j$, называется число $\Pi_{ij} = \cfrac{p_{ij}}{p_{y_j}}$; набор вероятностей $\Pi_{ij}, \ \forall i,j$ называется условным распределением случайной величины X.
\item Пусть $(X,Y)$ - непрерывный случайный вектор. Условной функцией распределения случайной величины X при условии $Y = y$ называется отображение $F_X(x|Y = y) = P\{X < x|Y = y\}$. Условной плотностью распределения случайной величины X при условии $Y = y$ называется функция $f_X(x|Y = y) = \cfrac{f(x,y)}{f_Y(y)}$, где $f(x,y)$ - совместная плотность распределения случайного вектора.
\end{itemize}


\item[9.] Сформулировать определение независимых случайных величин. Сформулировать критерий независимости двух случайных величин в терминах условных распределений. \\
Пусть $(X,Y)$ - двумерный случайный вектор, Тогда: \\
\begin{enumerate}
\item[1.] Случайные величины $(X,Y)$ независимы $\Leftrightarrow \left[
\begin{array}{lll}
	F_X(x|Y = y) = F_X(x) \forall y, \text{на которых определена } F_X(x|Y = y) \\
	F_Y(y|X = x) = F_Y(y) \forall x, \text{на которых определена } F_Y(y|X = x) \\
\end{array} \right.$ 
\item[2.] Если $(X,Y)$ - непрерывный случайный вектор, то $X,Y$ независимы $\Leftrightarrow \left[
\begin{array}{lll}
	f_X(x|Y = y) = f_X(x) \\
	f_Y(y|X = x) = f_Y(y) \\
\end{array} \right.$
\item[3.] Если $(X,Y)$ - дискретный случайный вектор, то $X,Y$ независимы $\Leftrightarrow \left[
\begin{array}{lll}
	P\{X = x_i|Y = y_i\} = P\{X = X_i\} \\
	P\{Y = y_j|X = x_i\} = P\{Y = y_j\} \\
\end{array} \right.$
\end{enumerate}


\item[10.] Понятие функции случайной величины. Указать способ построения ряда распределения функции дискретной случайной величины. Сформулировать теорему о плотности распределения функции от непрерывной случайной величины. \\
\begin{itemize}
\item Случайная величина Y, которая каждому значению случайной величины X ставит в соответствие число $Y = \phi(x)$? называют скалярной функцией скалярной случайной величиной X. При этом сама Y также является случайной величиной: если X - дискретная случайная величина, то Y - также дискретная случайная величин; если X - непрерывная случайная величина, то Y может быть непрерывной случайной величиной, дискретной случайной величиной или случайной величиной смешанного типа.
\item Если X - дискретная случайная величина, то ряд распределения Y строится следующим образом - в первой строке записываются значения $y_i = \phi(x_i)$, а во вторую строку переписываются значения $p_i$, соответствовавшие $x_i$.
\item \textbf{Теорема:} если X - непрерывная случайная величина с плотностью распределения $f_X(x), \ \phi: \mathbb{R} \to \mathbb{R}$ - монотонная и непрерывно дифференцируемая скалярная функция, а $\psi$ - обратная к $\phi$, то для случайной величины $Y = \phi(x)$ функция распределения $f_Y(y) = f_X \left( \psi(y) \right) \left|\psi' (y)\right|$.
\end{itemize}


\item[11.] Понятие скалярной функции случайного векторного аргумента. Доказать формулу для нахождения значений функции распределения случайной величины Y, функциональ зависящей  от случайных величин $X_1$ и $X_2$. \\
\begin{itemize}
\item Пусть $(X_1, X_2)$ - случайный вектор, $\phi: \ \mathbb{R}^2 \to \mathbb{R}$ - скалярная функция. Случайная величина $Y = \phi(X_1, X_2)$ называют скалярной функцией случайного вектора.
\item \textbf{Теорема:} Пусть $(X_1, X_2)$ - непрерывный случайный вектор и $Y = \phi (X_1, X_2)$. Тогда $\displaystyle  F_Y(y) = \iint\limits_{D(y)} f(x_1, x_2) dx_1 dx_2$. \\
\textbf{Доказательство:} $F_Y(y) = P\{Y < y\}$. События $\{Y < y\}, \left\{(X_1, X_2) \in D(y) \right\}$ эквивалентны. Следовательно, $\displaystyle  F_Y(y) = P\left\{ (X_1,X_2) \in D(y) \right\} = \iint\limits_{D(y)} f(x_1, x_2) dx_1 dx_2$.
\end{itemize}


\item[12.] Сформулировать и доказать теорему о формуле свертки. \\
\textbf{Теорема:} Пусть (X, Y) - случайный вектор, непрерывный и независимый, а Z = X + Y. Тогда $\displaystyle  f_Z(z) = \int\limits_{-}^{+} f_X(x) f_Y(z - x) dx$. \\
\textbf{Доказательство:} $\displaystyle  F_Z(z) = P\{Z < z\} = P\{X + Y < z\} = P\{Y < z - X\} = P\left\{(X,Y) \in D \right\} = \iint\limits_{D} f(x,y) dxdy = \int\limits_{-}^{+} dx \int\limits_{-}^{z - x} f(x,y) dy$. Так как X, Y независимы, то $f(x,y) = f_X(x) f_Y(y)$, следовательно $\displaystyle  F_Z(z) = \int\limits_{-}^{+} dx \int\limits_{-}^{z - x} f_X(x) f_Y(y) dy = \int\limits_{-}^{+} f_X(x) dx \int\limits_{-}^{z - x} f(y)dy$. \\
Наконец, $\displaystyle  f_Z(z) = \cfrac{d}{dz} F_Z(z) = \cfrac{d}{dz} \left[ \int\limits_{-}^{+} f_X(x)dx \int\limits_{-}^{z - x} f_Y(y)dy \right] = \int\limits_{-}^{+} f_X(x) f_Y(z - x) dx$. \\
Выражение $\displaystyle  (f_1 \cdot f_2)(y) = \int\limits_{-}^{+} f_1(x_1) f_2(y - x) dx$ называется сверткой функций $f_1, f_2$. %проверить корректность выражения


\item[13.] Сформулировать определение математического ожидания случайной величины (дискретный и непрерывный случаи). Записать формулы для вычисления математического ожидания функции от случайной величины. Сформулировать свойства математического ожидания. Механический смысл математического ожидания. \\
\begin{itemize}
\item Дискретная случайная величина: Математическим ожиданием случайной величины X называется число $M[X] = \sum\limits_{i} p_i x_i$, где $p_i = P\{X = x_i\}$, $x_i$ пробегает множество всех значений X. \\
Непрерывная случайная величина: Математическим ожиданием случайной величины X называется число $\displaystyle M[x] = \sum\limits_{-}^{+} x f(x) dx$, где $f(x)$ - плотность распределения непрерывной случайной величины X.
\item Если X - непрерывная случайная величина, $\phi: \ \mathbb{R} \to \mathbb{R}$ - скалярная функция, то $M[\phi(X)] = \sum\limits_{i} p_i \phi(x_i)$ для дискретной случайной величины и $\displaystyle  = \int\limits_{-}^{+} \phi(x) f(x) dx$ для непрерывной случайной величины.
\item Механический смысл математического ожидания: пусть есть стержень, обладающий "вероятностной массов"\ и в $x_i$ лежит её $p_i$ часть. Тогда математическое ожидание задает $x_0$ - центр тяжести для этого стержня. В случае непрерывной случайной величины, $f(x)$ можно интерпретировать как "плотность"\ бесконечного стержня. \\
Свойства математического ожидания: \\
	\begin{enumerate}
	\item[1)] Если X принимает значение $x_0$ с вероятностью 1 (то есть не является случайной величиной), то $MX = x_0$.
	\item[2)] $M[aX + b] = aM[X] + b$
	\item[3)] $M[X + Y] = MX + MY$
	\item[4)] Если X и Y независимые, то $M[XY] = MXMY$
	\end{enumerate}
\end{itemize}


\item[14.] Сформулировать определение дисперсии случайной величины. Записать формулы для вычисления дисперсии в дискретном и непрерывном случае. Сформулировать свойства дисперсии. Механический смысл дисперсии. \\
\begin{itemize}
\item Дисперсией случайной величиной X называют математическое ожидание квадрата отклонения случайной величины X от её среднего значения: $D[X] = M[X - MX]^2$. Для дискретной случайной величины: $DX = \sum\limits_{i} (x_i - MX)^2 p_i$; для непрерывной случайной величины: $\displaystyle  DX = \sum\limits_{-}^{+} (x - MX)^2 f(x)dx$. \\
Механический смысл: Дисперсия представляет собой второй момент центрированной случайной величины X $X^o = X - MX$ 
\item Свойства дисперсии: 
	\begin{enumerate}
	\item[1)] Если случайная величина X принимает всего одно значение C с вероятностью 1, то $DC = 0$
	\item[2)] $D[aX + b] = a^2 DX$
	\item[3)] $DX = M[X^2] - (MX^2)$
	\item[4)] $D[X+Y] = DX + DY$, если X и Y - независимые случайные величины
	\end{enumerate}
\end{itemize}


\item[15.] Сформулировать определения начального и центрального моментов случайной величины. Математическое ожидание и дисперсия как моменты. Сформулировать определения кнавтили и медианы случайной величины. \\
\begin{itemize}
\item Начальным моментом K-го порядка случайной величины X называют математическое ожидание K-й степени этой случайной величины: $m_k = M\left[X^k\right] = \sum\limits_{i} x^k_i p_i$.
\item Центральным моментом K-го порядка X называют математическое ожидание K-й степени величины $X^o = X - MX: \ m^o_k = M\left[ (X - MX)^k \right] = \sum\limits_{i} (x_i - MX)^k p_i$.
\item Математическое ожидание случайной величины X - совпадает с моментом первого порядка. Дисперсия совпадает с центральным моментом 2-го порядка.
\item Квантилью случайной величины X уровня a называется число $q_a$, определяемое соотношением $P\{X < q_a\} \leqslant a, \ P\{X > q_a\} \leqslant 1 - a$. Медианой случайной величины X называется её квантиль уровня 0.5.
\end{itemize}


\item[16.] Сформулировать определение ковариации случайных величин. Записать формулы для вычисления ковариации в дискретном и непрерывном случаях. Сформулировать свойства ковариации.
\begin{itemize}
\item Ковариацией случайный величин X и Y называется число $cov(X,Y) = M \left[ (X - m_1)(Y - m_2) \right]$, где $m_1 = MX, m_2 = MY$. \\
Если X,Y - дискретные случайные величины, то ковариация $cov(X,Y) = \sum\limits_{ij} (x_i - MX)(y_j - MY) p_{ij}$: если непрерывная случайная величина - $\displaystyle  cov(X,Y) = \int\limits_{-}^{+} \int\limits_{-}^{+} (x - MX)(y - MY) f_{XY}(x,y) dxdy$.
\item Свойства ковариации 
	\begin{enumerate}
	\item[1)] $cov(X,X) = DX$
	\item[2)] $cov(X,Y) = 0$ если X,Y - независимые случайные величины
	\item[3)] Если $Y_1 = a_1 X_1 + b_1, \ Y_2 = a_2 X_2 + b_2$, то $cov(Y_1, Y_2) = a_1 a_2 cov(X_1, X_2)$
	\item[4)] $-\sqrt{DXDY} \leqslant cov(X,Y) \leqslant \sqrt{DXDY}$
	\item[5)] Равенство $\left|cov(X,Y)\right| = \sqrt{DXDY}$ верно тогда и только тогда, когда случайные величины X,Y связаны линейной зависимостью, то есть $Y = aX + b$
	\item[6)] $cov(X,Y) = M(XY) - MXMY$
	\end{enumerate}
\end{itemize}

\end{enumerate}



\end{document}



























