% 2. Теорема сложения и умножения.

\underline{Th 1} Сложения. \\
\begin{enumerate}
\item[1)] $P(A_1 + A_2) = P(A_1) + P(A_2) - P(A_1 A_2)$ \\

\item[2)] $P(A_1 + \ldots + A_n) = \sum\limits_{i = 1}^{n} P(A_i) - \sum\limits_{1 \leqslant i_1 < i_2 \leqslant n}^{n} P(A_{i_1} A_{i_2}) + \sum\limits_{1 \leqslant i_1 < i_2 \leqslant n}^{n} P(A_{i_1} A_{i_2} A_{i_3}) - \ldots + (-1)^{n - 1} P(A_1 \cdot \ldots \cdot A_n)$ \\
\end{enumerate}


\underline{Th 2} Умножения. \\
\begin{enumerate}
\item[1)] Если $P(A_1) > 0$, то $P(A_1 A_2) = P(A_) \cdot P(A_2|A_1)$ \\

\item[2)] Если $P(A_1 \cdot \ldots \cdot A_{n-1}) > 0$, то \\
$P(A_1 \cdot \ldots \cdot A_n) = P(A_1) \cdot P(A_2|A_1) \cdot P(A_3|A_1 A_2) \cdot \ldots \cdot P(A_n|A_1 \cdot \ldots \cdot A_{n-1})$ \\
\end{enumerate}


\underline{Пример}: Оператор ЗРК видит на экране 10 целей, среди которых 8 самолетов и 2 помехи (на экране они неразличимы). Оператор последовательно атакует 4 раза. \\
A = \{большая часть атакованных целей - самолеты\} \\
B = \{все помехи были атакованы\} \\
C = \{помеха была атакована но не ранее, чем 3-им выстрелом\} \\
D = \{самолеты и помехи атакованы вперемежку\} \\
\underline{Решение}: \\
\begin{enumerate}
\item[1)] Исход: $(x_1 x_2 x_3 x_4)$ (размещение без повторений из 10 по 4), где $x_i$ - номер цели, атакованный i-ым выстрелом. \\
$N = A^4_10 = \cfrac{10!}{6!}$ \\

\item[2)] $A_1$ = \{атаковано ровно 3 самолета\} \\
$A_2$ = \{атаковано ровно 4 самолета\}
$P(A) = P(A_1 + A_2) = P(A_1) + P(A_2) - \underbrace{P(A_1 A_2)}_{\text{0}} = P(A_1) + P(A_2)$ \\
%Вставить рисунок
$A_2 = \{ (\text{с, с, с, с})\}$ \\
$N_{A_2} = A^4_8 = \cfrac{8!}{4!}$ \\
$P(A_2) = \cfrac{N_{A_2}}{N} = \cfrac{\cfrac{8!}{4!}}{\cfrac{10!}{6!}}$ \\
$P(A_1) = ?$ \\
%Вставить рисунок
$8 \cdot A^3_8 = N_{A_1}$ \\
$P(A_1) = \cfrac{8 A^3_8}{A^4_10}$ \\

\item[3)] $P(B) = ?$ \\
(п, п, с, с) \\
$C^2_4$ - число
\end{enumerate}


































