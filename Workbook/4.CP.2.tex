% 2. Теорема сложения и умножения.

\underline{Th 1} Сложения. \\
\begin{enumerate}
\item[1)] $P(A_1 + A_2) = P(A_1) + P(A_2) - P(A_1 A_2)$ \\

\item[2)] $P(A_1 + \ldots + A_n) = \sum\limits_{i = 1}^{n} P(A_i) - \sum\limits_{1 \leqslant i_1 < i_2 \leqslant n}^{n} P(A_{i_1} A_{i_2}) + \sum\limits_{1 \leqslant i_1 < i_2 \leqslant n}^{n} P(A_{i_1} A_{i_2} A_{i_3}) - \ldots + (-1)^{n - 1} P(A_1 \cdot \ldots \cdot A_n)$ \\
\end{enumerate}


\underline{Th 2} Умножения. \\
\begin{enumerate}
\item[1)] Если $P(A_1) > 0$, то $P(A_1 A_2) = P(A_) \cdot P(A_2|A_1)$ \\

\item[2)] Если $P(A_1 \cdot \ldots \cdot A_{n-1}) > 0$, то \\
$P(A_1 \cdot \ldots \cdot A_n) = P(A_1) \cdot P(A_2|A_1) \cdot P(A_3|A_1 A_2) \cdot \ldots \cdot P(A_n|A_1 \cdot \ldots \cdot A_{n-1})$ \\
\end{enumerate}


\underline{Пример}: Оператор ЗРК видит на экране 10 целей, среди которых 8 самолетов и 2 помехи (на экране они неразличимы). Оператор последовательно атакует 4 раза. \\
A = \{большая часть атакованных целей - самолеты\} \\
B = \{все помехи были атакованы\} \\
C = \{помеха была атакована но не ранее, чем 3-им выстрелом\} \\
D = \{самолеты и помехи атакованы вперемежку\} \\
\underline{Решение}: \\
\begin{enumerate}
\item[1)] Исход: $(x_1 x_2 x_3 x_4)$ (размещение без повторений из 10 по 4), где $x_i$ - номер цели, атакованный i-ым выстрелом. \\
$N = A^4_10 = \cfrac{10!}{6!}$ \\

\item[2)] $A_1$ = \{атаковано ровно 3 самолета\} \\
$A_2$ = \{атаковано ровно 4 самолета\}
$P(A) = P(A_1 + A_2) = P(A_1) + P(A_2) - \underbrace{P(A_1 A_2)}_{\text{0}} = P(A_1) + P(A_2)$ \\
%Вставить рисунок
$A_2 = \{ (\text{с, с, с, с})\}$ \\
$N_{A_2} = A^4_8 = \cfrac{8!}{4!}$ \\
$P(A_2) = \cfrac{N_{A_2}}{N} = \cfrac{\cfrac{8!}{4!}}{\cfrac{10!}{6!}}$ \\
$P(A_1) = ?$ \\
%Вставить рисунок
$8 \cdot A^3_8 = N_{A_1}$ \\
$P(A_1) = \cfrac{8 A^3_8}{A^4_10}$ \\

\item[3)] $P(B) = ?$ \\
(п, п, с, с) \\
$C^2_4$ - число способов выбрать две позиции для помех.\\
$2!$ - число способов расставить 2 помехи по выбранным позициям. \\
$A^2_8$ - число способов расставить самолеты. \\
$N_B = C^2_4 \cdot 2 \cdot A^2_8$ \\
$P(B) = \cfrac{N_B}{N}$ \\

\item[4)] $C = C_1 + C_2$, где \\
$C_1$ = \{помехи впервые атакованы 3-м выстрелом\} \\
$C_2$ = \{помехи впервые атакованы 4-м выстрелом\} \\
$P(C) = P(C_1 + C_2) = P(C_1) + P(C_2) - \underbrace{P(C_1 C_2)}_{0}$ \\
$P(C_1) = ?$ \\
\underline{I способ}. \\
(с, с, п, *) \\
\underline{II способ}. \\
$Q_i$ = \{i-ым выстрелом атакован самолет\} $i = \overline{1,4}$ \\
$\overline{Q_i}$ = \{i-ым выстрелом атакована помеха\} \\
$C_1 = Q_1 \cdot Q_2 \cdot \overline{Q_3}$ \\
$P(C_1) = P(Q_1 Q_2 \overline{Q_3}) = \underbrace{P(Q_1)}_{\tfrac{8}{10}} \cdot \underbrace{P(Q_2|Q_1)}_{\tfrac{7}{9}} \cdot \underbrace{P(\overline{Q_3}|Q_1 Q_2}_{\tfrac{2}{8}} = \cfrac{2 \cdot 7}{9 \cdot 10} = \cfrac{7}{35}$ \\

$C_2 = Q_1 \cdot Q_2 \cdot Q_3 \cdot \overline{Q_4}$ \\
$P(C_2) = \underbrace{P(Q_1)}_{\tfrac{8}{10}} \cdot \underbrace{P(Q_2|Q_1)}_{\tfrac{7}{9}} \cdot \underbrace{P(Q_3|Q_1 Q_2)}_{\tfrac{6}{8}} \cdot \underbrace{P(\overline{Q_4}|Q_3 Q_2 Q_1)}_{\tfrac{2}{7}} = \cfrac{2 \cdot 6}{10 \cdot 9} = ...$ \\

\item[5)] $P(D) = ?$ \\
$D_2$ = \{(с, п, с, п)\} \\
$D_1$ = \{(п, с, п, с)\} \\
$D = D_1 + D+2$ \\
$P(D) = P(D_1 + D_2) = P(D_1) + P(D_2) - \underbrace{P(D_1 D_2)}_{0}$ \\

$D_1 = \overline{Q_1} \cdot Q_2 \cdot \overline{Q_3} \cdot Q_4$ \\
$P(D_1) = \underbrace{P(\overline{Q_1})}_{\tfrac{2}{10}} \cdot \underbrace{P(Q_2|\overline{Q_1})}_{\tfrac{8}{9}} \cdot \underbrace{P(\overline{Q_3}|\overline{Q_1} Q_2)}_{\tfrac{1}{8}} \cdot \underbrace{P(Q_4|\overline{Q_1} Q_2 \overline{Q_3})}_{\tfrac{7}{7}} = \cfrac{2}{10} \cdot \cfrac{8}{9} \cdot \cfrac{1}{8} \cdot \cfrac{7}{7} = \cfrac{2}{90} = \cfrac{1}{45}$ \\

$D_2 = Q_1 \cdot \overline{Q_2} \cdot Q_3 \cdot \overline{Q_4}$ \\
$P(D_2) = \underbrace{P(Q_1)}_{\tfrac{8}{10}} \cdot \underbrace{P(\overline{Q_2}|Q_1)}_{\tfrac{2}{9}} \cdot \underbrace{P(Q_3|Q_1 \overline{Q_2})}_{\tfrac{7}{8}} \cdot \underbrace{P(\overline{Q_4}|Q_1 \overline{Q_2} Q_3)}_{\tfrac{1}{7}} = \cfrac{8}{10} \cdot \cfrac{2}{9} \cdot \cfrac{7}{8} \cdot \cfrac{1}{7} = \cfrac{2}{90} = \cfrac{1}{45}$ \\

$P(D) = P(D_1 + D_2)$ \\
\end{enumerate}


\underline{Пример}: По каналу связи, подверженному воздействию помех, передаются кодовые последовательности из 0 и 1. При этом вероятность того, что \\
$1 \mapsto 1 \ \ p_1$ \\
$1 \mapsto 0 \ \ 1 - p_1$ \\
$0 \mapsto 0 \ \ p_2$ \\
$0 \mapsto 1 \ \ 1 - p_2$ \\
В канал подают последовательность "10"\ . Считая, что отдельные символы искажаются независимо, найти вероятности событии: \\
A = \{принята последовательность 10\} \\
B = \{приняты 2 одинаковые символы\} \\
\underline{Решение}: \\
\begin{enumerate}
\item[1)] Исход: $(x_1, x_2)$, где $x_i \in {0, 1}$ - i-ый принятый символ.

\item[2)] $A = \{(10)\}$ \\
$A = A_1 \cdot A_2$, где \\
$A_1$ = \{1-ый принятый символ 1\} \\
$A_2$ = \{2-ой принятый символ 0\} \\
$P(A)$ = (th умножения) = $\underbrace{P(A_1)}_{p_1} \cdot \underbrace{P(A_2|A_1)}_{P(A_2) = p_2 \text{т.к. отдельные символы искажаются независимо}}$ \\

\item[3)] $B = B_1 + B_2$ \\
$B_1 = \{(1,1)\}$ \\
$B_2 = \{(0,0)\}$ \\
$P(B_1) = P \left\{ \{1 \mapsto 1\} \cdot \{0 \mapsto 1\} \right\}$ = (искажения отдельных символов независимы) = $P\{1 \mapsto 1\} \cdot P\{0 \mapsto 1\} = p_1 \cdot (1 - p_2)$ \\

$P(B_2) = P \left\{ \{1 \mapsto 0\} \cdot \{0 \mapsto 0\} \right\} = P\{1 \mapsto 0\} \cdot P\{0 \mapsto 0\} = (1 - p_1) \cdot p_2$ \\

$P(B) = P(B_1 + P_2) = P(B_1) + P(B_2) - \underbrace{P(B_1 B_2)}_{0} = p_1 + p_2 - 2p_1 p_2$\\
\end{enumerate}


\underline{Пример}: На карточках написаны буквы слова "дезоксирибонуклеиновая"\ . Эти карточки тщательно перемешиваются и последовательно извлекают 5 карточек без возвращения. \\
Найти вероятность того, что в порядке появления они извлекают слово "кокос"\ . \\
\underline{Решение}: \\
A = \{карточки образуют слово "кокос"\ \} \\
Тогда \\
$A = A_1 \cdot \ldots \cdot A_5$, где \\
$A_1$ = \{на 1-ой карточке "к"\ \} \\
$A_2$ = \{на 2-ой карточке "о"\ \} \\
$A_3$ = \{на 3-ей карточке "к"\ \} \\
$A_4$ = \{на 4-ой карточке "о"\ \} \\
$A_5$ = \{на 5-ой карточке "с"\ \} \\
%Вставить рисунок

$P(A) = P(A_1 \cdot A_2 \cdot A_3 \cdot A_4 \cdot A_5) = P(A_1) \cdot P(A_2|A_1) \cdot P(A_3|A_1 A_2) \cdot P(A_4|A_1 A_2 A_3) \cdot P(A_5|A_1 A_2 A_3 A_4) = \cfrac{2}{22} \cdot \cfrac{3}{21} \cdot \cfrac{1}{20} \cdot \cfrac{2}{19} \cdot \cfrac{1}{18} = ...$ \\


\underline{Пример}: Известно, что A, B - наблюдаемые в некотором случайном эксперименте события и \\
$P(B) = 0.4$ $P(A|B) = 0.3$ $P(A|\overline{B}) = 0.2$ \\
$P(A), \ P(\overline{A} \overline{B}), \ P(\overline{A} + \overline{B}), \ P(A \vartriangle B) = ?$ \\
\underline{Решение}:
\begin{enumerate}
\item[1)] $P(AB)$ = (по th умножения) = $P(A) \cdot P(B|A)$ \\
$\Rightarrow P(AB) = P(B) \cdot P(A|B) = 0.4 \cdot 0.3 = 0.12$ \\

\item[2)] $P(A) = P(A \Omega) + P(A(B + \overline{B})) = P(AB + A\overline{B})$ = (по th сложения) = $\underbrace{P(AB)}_{0.12} + P(A \overline{B}) - \underbrace{P(ABA \overline{B})}_{0} = 0.12 + \underbrace{P(B)}_{0.6} \cdot \underbrace{P(A|\overline{B})}_{0.2} = 0.12 + 0.6 \cdot 0.2 = 0.12 + 0.12 = 0.24$ \\

\item[3)] $P(\overline{A} \overline{B}) = \underbrace{P(\overline{B})}_{0.6} \cdot \underbrace{P(\overline{A}|\overline{B})}_{= 1 - P(A|\overline{B})} = 0.6 \cdot (1 - 0.2) = 0.6 \cdot 0.8 = 0.48$ \\
%Вставить пояснение в формулу выше.

\item[4)] $P(\overline{A} + \overline{B})$ = (th сложения) = $P(\overline{A}) + P(\overline{B}) - P(\overline{A} \overline{B}) = 0.76 + 0.6 - 0.48 = 0.88$ \\

\item[5)] $P(A \vartriangle B) = P(A \backslash B) + B \backslash A) = P(A\overline{B} + B\overline{A})$ = (th сложения) = $P(A|overline{B}) + P(\overline{A}B) - \underbrace{P(A\overline{B}\overline{A}B}_{0} = 
\underbrace{P(\overline{B}}_{0.6} \cdot \underbrace{A|\overline{B}}_{0.2} + \underbrace{P(B)}_{0.4} \cdot \underbrace{P(\overline{A}|B)}_{\text{при фикс.} = 
1 - P(A|B)} = 0.12 + 0.4 \cdot 0.7 = 0.4$ \\ 
\end{enumerate}

























