% 2. Дискретные случайные величины

\underline{Определение:} Случайной величиной называют дискретной, если множество её возможных значений конечно или счетно. \\
Если случайная величина X принимает значения из конечного множества, закон её распределения можно задать с использованием таблицы, которая называется рядом распределения. \\
%Вставить таблицу
Здесь $x_i, \ i = \overline{1,n}$ - всевозможные значения случайной величины X. \\
$p_i = P\{X = x_i\}, \ i = \overline{1,n}$ \\
Очевидно, $\sum\limits_{i = 1}^{n} p_i = 1$ \\


\underline{Пример:} Бросают игральную кость. X - число выпавших очков.
\begin{enumerate}
\item[1)] Построить ряд распределения случайной величины X.
\item[2)] Найти функцию распределения случайной величины X.
\item[3)] $P\{2 \leqslant x < 5\} = ?$ 
\end{enumerate}
\underline{Решение:} \\
\begin{enumerate}
\item[1)] %Вставить таблицу

\item[2)] $F(x) = P\{X < x\}$ \\
%Вставить график
$F(x) = 
\begin{cases} 
	0, \ x \leqslant 1 \\
	\frac{1}{6}, \ 1 < x \leqslant 2 \\
	\frac{2}{6}, \ 2 < x \leqslant 3 \\
	\frac{3}{6}, \ 3 < x \leqslant 4 \\
	\frac{4}{6}, \ 4 < x \leqslant 5 \\
	\frac{5}{6}, \ 5 < x \leqslant 6 \\
	1 , 6 < x \\
\end{cases}$

\item[3)] $P \{2 \leqslant X < 5\}$ = (Свойство функции распределения) = $F(5) - F(2) = \cfrac{4}{6} - \cfrac{1}{6} = \cfrac{3}{6} = \cfrac{1}{2}$
\end{enumerate}