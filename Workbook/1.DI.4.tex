% 4. Приложения двойного интеграла.

I. Вычисление площади плоской фигуры. \\

Пусть фигура занимает область D на плоскости Oxy. \\
Тогда \\
$\displaystyle S(D) = \iint\limits_{D} dxdy$ \\
%Вставить рисунок


II. Вычисление массы пластины. \\

Пусть \\
\begin{enumerate}
\item[1)] пластина занимает область D на плоскости Oxy. \\

\item[2)] $f(x,y)$ - значение плотности (поверхностного) материалы пластины в точке $(x,y)$ \\
\end{enumerate}
Тогда масса пластины: \\
$\displaystyle  m = \iint\limits_{D} f(x,y) dxdy$ \\
%Вставить рисунок


III. Вычисление объема тела. \\
Пусть \\
\begin{enumerate}
\item[1)] Тело задано в виде \\
$T = \left\{ (x,y,z): \ (x,y) \in D_{xy}, \ z_1(x,y) \leqslant z \leqslant z_2(x,y) \right\}$ \\
%Вставить рисунок
\end{enumerate}
Тогда \\
$\displaystyle  V(T) = \iint\limits_{D_{xy}} \left[ z_2(x,y) - z_1(x,y) \right] dxdy$ \\


\underline{Пример}: Найти площадь фигуры, ограниченной линиями: \\
$x^2 + y^2 = 2x$ \\
$x^2 + y^2 = 4x$ \\
$x = y$ \\
$y = 0$ \\
\underline{Решение}:
%Вставить график
$\displaystyle  S = \iint\limits_{D} dxdy$ \\
Перейдем в полярную систему координат \\
$\displaystyle  S = 
\iint\limits_{D_{xy}} dxdy = 
\iint\limits_{D_{\rho \varphi}} \rho \ d\rho d\varphi = 
\int\limits_{0}^{\tfrac{\pi}{4}} d\varphi \int\limits_{2 \cdot \cos{\varphi}}^{4 \cdot \cos{\varphi}} \rho d\rho = 
\int\limits_{0}^{\tfrac{\pi}{4}} \left[ \cfrac{\rho^2}{2} \bigg|_{2 \cdot \cos{\varphi}}^{4 \cdot \cos{\varphi}} \right] d\varphi = 
6 \int\limits_{0}^{\tfrac{\pi}{4}} \cos^2{\varphi} d\varphi = 
6 \int\limits_{0}^{\tfrac{\pi}{4}} \cfrac{1 + \cos{2\varphi}}{2} d\varphi = 
3 \int\limits_{0}^{\tfrac{\pi}{4}} \left( 1 + \cfrac{\cos{2\varphi}}{2} \right) d2\varphi = 
3 \left[ \varphi + \cfrac{1}{2} \sin{2\varphi} \right] \bigg|_{0}^{\tfrac{\pi}{4}} = 
3 \left[\cfrac{\pi}{4} + \cfrac{1}{2} \left(\sin{\cfrac{\pi}{2}} - \sin{0}\right) \right] = 
3 \left[ \cfrac{\pi}{4} + \cfrac{1}{2} \right] = 
\cfrac{3}{4} \pi + \cfrac{3}{2}$ \\


\underline{Пример}: Вычислить объем тела, ограниченного поверхностями: \\
$z = 2x^2 + y^2 + 1$ \\
$x + y = 1$ \\
$x = 0, \ y = 0, \ z = 0$ \\
%Вставить график
%Вставить график
Метод сечений: \\
$z = x^2 + y^2$ \\
$z = a$ \\
$x^2 + y^2 = a$ \\
$a < 0 \Rightarrow \emptyset$ \\
$a = 0 \Rightarrow 0(0, 0)$ \\
$a > 0 \ \ \cfrac{x^2}{a^2} + \cfrac{y^2}{b^2} = 1$ \\ 
$\displaystyle  V = \iint\limits_{D_{xy}} \left[ z_2(x, y) - z_1(x,y) \right] dxdy = 
\iint\limits_{D_{xy}} \left[ 2x^2 + y^2 + 1 - 0 \right] dxdy = 
\iint\limits_{D_{xy}} \left[ 2x^2 + y^2 + 1 \right] dxdy = 
\int\limits_{0}^{1} dx \int\limits_{0}^{1 - x} \left[ 2x^2 + y^2 + 1 \right] dy = 
\int\limits_{0}^{1} \left[ 2x^2y + \cfrac{y^3}{3} + y \right] \bigg|_{0}^{1 - x} dx = 
\int\limits_{0}^{1} \left[ 2x^2 - 2x^3 + \cfrac{(1 - x)^3}{3} + 1 - x \right] dx = 
\cfrac{1}{3} \int\limits_{0}^{1} \left[ 6x^2 - 6x^3 + (1 - x)^3 + 3 - 3x \right] dx = 
\cfrac{1}{3} \int\limits_{0}^{1} \left[ 6x^2 - 6x^3 + 1 - 3x + 3x^2 - x^3 + 3 - 3x \right] dx = 
\cfrac{1}{3} \int\limits_{0}^{1} \left[ -7x^3 + 9x^2 - 6x + 4 \right] dx = 
\cfrac{1}{3} \left[ -\cfrac{7}{4} x^4 + \cfrac{9}{3} x^3 - \cfrac{6}{2} x^2 + 4x \right] \bigg|_{0}^{1} = 
\cfrac{1}{3} \left[ -\cfrac{7}{4} + \cfrac{9}{3} - 3 + 4 \right] = 
\cfrac{1}{3} \left[ \cfrac{- 21 + 36}{12} + 1 \right] = 
\cfrac{1}{3} \left[ \cfrac{15}{12} + 1 \right] = \cfrac{1}{3} \cdot \cfrac{27}{12} = \cfrac{3}{4}$ \\


\underline{Пример}: Найти объем тела, ограниченного поверхностями \\
$z = \cfrac{y^2}{a}$ \\
$x^2 + y^2 = r^2$ \\
$z = 0$  \\
%Вставить график
%Вставить график
$\displaystyle  V = \iint\limits_{D_{xy}} \left[ z_2(x,y) - z_1(x,y) \right] dxdy = 
\iint\limits_{D_{xy}} \cfrac{y^2}{a} dxdy$ \\
Перейдем в полярную систему координат
$\displaystyle  \cfrac{1}{a} \iint\limits_{D_{\rho \varphi}} \rho^2 \cdot \rho \cdot \sin^2{\varphi} \ d\varphi d\rho = 
\cfrac{1}{a} \int\limits_{0}^{2\pi} d\varphi \int\limits_{0}^{r} \rho^3 \cdot \sin^2{\varphi} \ d\rho = 
\cfrac{1}{a} \int\limits_{0}^{2\pi} \cfrac{\rho^4}{4} \bigg|_{0}^{r} \cdot \sin^2{\varphi} d\varphi = 
\cfrac{1}{8a} \int\limits_{0}^{2\pi} r^4 \cdot (1 - \cos{2\varphi}) d\varphi = 
\cfrac{1}{8a} r^4 \varphi \bigg|_{0}^{2\pi} = 
\cfrac{\pi r^4}{4a}$ \\


\underline{Пример}: Найти объем тела, ограниченного поверхностями: \\
$2az = x^2 + y^2$ \\
$x^2 + y^2 + z^2 = 3a^2$ \\
%Вставить график
$x^2 + y^2 = 3a^2 - z^2$ \\
$2az = 3a^2 - z^2$ \\
$z^2 + 2az - 3a^3 = 0$ \\
$\displaystyle  V = \iint\limits_{D_{xy}} \left[ z_2(x,y) - z_1(x,y) \right] dxdy$ \\
$x^2 + y^2 + z^2 = 3a^2$ \\
$z = \sqrt{3a^2 - x^2 - y^2}$ \\
$z_2(x,y) = \sqrt{3a^2 - x^2 - y^2}$ \\
$z_1(x,y) = \cfrac{x^2}{2a} + \cfrac{y^2}{2a}$ \\
%Вставить график
$\begin{cases}
	x^2 + y^2 + z^2 = 3a^2 \\
	x^2 + y^2 = 2az \\
\end{cases}
\Rightarrow z^2 = 3a^2 - 2az$ \\
$z^2 + 2az - 3a^2 = 0$ \\
$(z+2)^2 = 4a^2$ \\
$z_1  = -3a \ \ \ z_2 = a$ \\
$x^2 + y^2 + a^2 = 3a^2$ \\
$x^2 + y^2 = 2a^2$ \\
$r = a \sqrt{2}$ \\
Перейдем в полярную систему координат \\
$\displaystyle  = \iint\limits_{D_{\rho \varphi}} \left[ \sqrt{3a^2 - \rho^2} - \cfrac{\rho^2}{2a} \right] \polarJacobian = 
\int\limits_{0}^{2\pi} d\varphi \int\limits_{0}^{a\sqrt{2}} \sqrt{3a^2 - \rho^2} - \cfrac{\rho^2}{2a} \rho \ d\rho = 
2 \pi \left[ \int\limits_{0}^{a\sqrt{2}} \rho \sqrt{3a^2 - \rho^2} d\rho - \int\limits_{0}^{a\sqrt{2}} \cfrac{\rho^3}{2a} d\rho \right] = 
2\pi \left[ - \cfrac{1}{2} \int\limits_{0}^{a\sqrt{2}} \sqrt{3a^2 - \rho^2} d\left(3a^2 - \rho^2 \right) - \cfrac{\rho^4}{8a} \bigg|_{0}^{a\sqrt{2}} \right] = 
2\pi \left[ -\frac{1}{2} \cdot \cfrac{2}{3} \left(3a^2 - \rho^2\right)^{\tfrac{3}{2}} \bigg|_{0}^{a\sqrt{2}} - \cfrac{a^3}{4} \right] = 
2\pi \left[ -\cfrac{1}{3} \left( a^3 - 3\sqrt{3} a^3 \right) - \cfrac{a^3}{4} \right] = 
2\pi a^3 \left[ \sqrt{3} - \cfrac{1}{2} - \cfrac{1}{2} \right] = 
2\pi a^3 \left[\sqrt{3} - \cfrac{5}{6} \right]$ \\

































