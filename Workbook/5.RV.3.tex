% 3. Непрерывные случайные величины

\underline{Определение:} Случайная величина X называется непрерывной, если $\exists$ функция $f: \ \mathbb{R} \to \mathbb{R}$ такая, что $\forall x \in \mathbb{R}$, $\displaystyle  F(x) = \int\limits_{-\infty}^{x} f(t)dt$, \\
где 
\begin{enumerate}
\item[1)] F - функция распределения случайной величины X
\item[2)] предполагается, что несобственный интеграл в правой части сходится для всех $x \in \mathbb{R}$. \\
При этом f называют функцию плотности распределения вероятностей случайной величины X.
\end{enumerate}

\underline{Замечание:} 
\begin{enumerate}
\item[1)] %Вставить график

\item[2)] Можно показать, что во всех точках непрерывности $f(x)$ справедливо $f(x) = F'(x)$

\item[3)] Таким образом \\
$f(x) \Rightarrow F(x)$ (см. определение непрерывности случайной величины). \\
$F(x) \Rightarrow f(x)$ (см. пункт 2 замечания)
\end{enumerate}
Это означает, что функция плотности, как и функция распределения, содержит всю информацию о законе распределения случайной величины X. 


\underline{Свойства непрерывности случайной величины} \\
\begin{enumerate}
\item[$1^o$] $f(x) \geqslant 0$ 
\item[$2^o$] $\displaystyle  P\{a \leqslant X < b\} = \int\limits_{a}^{b} f(x) dx$ 
\item[$3^o$] $\displaystyle  \int\limits_{-\infty}^{+\infty} f(x) dx = 1$ (условие нормировки)
\item[$4^o$] Если X - непрерывная случайная величина, то для любого наперед заданного $x_0$ $P\{X = x_0\} = 0$
\end{enumerate}


\underline{Пример:} Функция распределения случайной величины X имеет вид $F(x) = 
\begin{cases}
	0, \ x \leqslant 2 \\
	(x - 2)^2, \ 2 < x \leqslant 3\\
	1, \ 3 < x \\
\end{cases}$\\
\begin{enumerate}
\item[1)] Найти функцию плотности и построить её график
\item[2)] $P\{X \in [1; 2,5)\}, \ P\{X \in [2,5; 3,5] \}, \ P\left\{X > \cfrac{9}{4}\right\} - ?$
\end{enumerate}
\underline{Решение:} 
\begin{enumerate}
\item[1)] %Вставить график
$f(x) = F'(x) = \left\{
\begin{array}{lll}
	0, & x \leqslant 2 \\
	2(x - 2), & 2 < x \leqslant 3 \\
	0, & 3 < x \\
\end{array} \right\} = \left\{
\begin{array}{lll}
	2(x - 2), \ 2 < x \leqslant 3 \\
	0, \ \text{иначе} \\
\end{array} \right\}$
%Вставить график

\item[2)] $P \left\{X \in [1; 2,5) \right\} = P \{ 1 \leqslant X < 2,5 \}$ = (свойство функции распределения) = $F(2,5) - F(1) = (2,5 - 2)^2 - 0 = \cfrac{1}{4}$ \\
$P\left\{ X \in [2,5; 3,5] \right\} = P\{2,5 \leqslant X \leqslant 3,5 \} = P\left\{ \underbrace{\{2,5 \leqslant X < 3,5\} + \{X = 3,5\}}_{\text{несовместны}} \right\} = P\{2,5 \leqslant X < 3,5\} + \underbrace{P\{X = 3,5\}}_{0, \text{т.к. X - непрер. сл. вел.}} = F(3,5) - F(2,5) = 1 - \cfrac{1}{4} = \cfrac{3}{4}$ \\
$P\left\{X > \cfrac{9}{4}\right\} = P\left\{ \cfrac{9}{4} < X < +\infty\right\} = \underbrace{P\left\{ X = \cfrac{9}{4} \right\}}_{= 0} + P\left\{ \cfrac{9}{4} < X < +\infty \right\}$ = (th сложения) = $P\left\{\cfrac{9}{4} \leqslant X < +\infty \right\}$ = (свойство функции распределения) = $ F(+\infty) - F\left( \cfrac{9}{4} \right) = 1 - \left( \cfrac{9}{4} - 2 \right)^2 = 1 - \cfrac{1}{16} = \cfrac{15}{16}$ \\
\end{enumerate}


\underline{Замечание:} Так как для непрерывной случайной величины $P\{X = x_0\} = 0$, то в дальнейшем для непрерывной случайной величины при вычислении вероятностей мы не будем различать события $\left\{ X \in [a;b] \right\}$, $\left\{ X \in [a,b) \right\}$, $\left\{ X \in (a;b] \right\}$, $\left\{ X \in (a,b) \right\}$. \\


\underline{Пример:} Дана функция $f(x) = ce^{-4|x-3|}, \ x \in \mathbb{R}$. 
\begin{enumerate}
\item[1)] Подобрать c = const так, чтобы f являлась функцией плотности некоторой случайной величины X.
\item[2)] Найти функцию распределения случайной величины X.
\item[3)] $P\{1 < X < 5\}$,  $P\{X \leqslant 13\}$
\end{enumerate}
\underline{Решение:}
\begin{enumerate}
\item[1)] $f(x) \geqslant 0$, если $c \geqslant 0$ \\
%Вставить график
При этом должно выполняться $\displaystyle  1 = \int\limits_{-\infty}^{+\infty} f(x) dx = 2c \int\limits_{3}^{+\infty} e^{-4|x-3|} dx = \left\{x - 3 \geqslant 0 \Rightarrow |x-3| = x - 3\right\} = 2c \int\limits_{3}^{+\infty} e^{-4(x-3)} dx = - \cfrac{2}{4} \cdot c \cdot e^{-4(x-3)} \bigg|_{3}^{+\infty} = 0 - \cfrac{2}{4} \cdot c (0 - 1) = \cfrac{1}{2} c$ \\
$1 = \cfrac{c}{2} \Rightarrow c = 2$ \\

\item[2)]  %Вставить график
$\displaystyle  F(x) = \int\limits_{-\infty}^{1} f(t) dt = 
\begin{cases}
	func 1, \ x \leqslant 3 \\
	func 2, \ x > 3 \\
\end{cases}$
function 1) $\displaystyle  = 2 \int\limits_{-\infty}^{x} e^{4(t-3)} dt = 2 \cdot \cfrac{1}{4} e^{4(t-3)} \bigg|_{-\infty}^{x} = \cfrac{1}{2} \cdot e^{4(x-3)}$ \\
function 2) $\displaystyle  = \cfrac{1}{2} \int\limits_{3}^{+\infty} e^{-4(t-3)} dt = \cfrac{1}{2} - \cfrac{1}{2} e^{-4(t-3)} \bigg|_{3}^{+\infty} = \cfrac{1}{2} - \cfrac{1}{2} \left(e^{-4(x-3)} - 1 \right) = 1 - \cfrac{1}{2} e^{-4(x-3)}$ \\

$F(x) = 
\begin{cases}
	\cfrac12 \cdot e^{4(x-3)}, \ x \leqslant 3 \\
	1 - \cfrac12 \cdot e^{-4(x-3)}, \ 3 < x \\
\end{cases}$
\end{enumerate}
































