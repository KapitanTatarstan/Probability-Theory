% 5. Условные законы распределения.


Пусть
\begin{enumerate}
	\item[1)] (X,Y) - двумерный случайный вектор.
	\item[2)] известно, что $Y = y_o$.
\end{enumerate}
Вопросы:\\
\begin{enumerate}
	\item[1)] что в этом случае можно сказать о возможных значениях случайной величины X?
	\item[2)] о распределении вероятностей между этими значениями?
\end{enumerate}


\textbf{I.} Пусть 
\begin{enumerate}
	\item[1.] (X,Y) - дискретный случайный вектор
	\item[2.] $X \in \{x_1, \ldots, x_n\}$, $P\{X = x_i\} = p_{X_i}$
	\item[3.] $Y \in \{y_1, \ldots, y_n\}$, $P\{Y = y_j\} = p_{Y_j}$
	\item[4.] $p_{ij} = P\left\{(X,Y) = (x_i, y_j)\right\}$
\end{enumerate}
На лекции было показано, что\\
$P\{X = x_i|Y = y_j\} = \cfrac{p_{ij}}{p_{X_i}}$\\


\underline{Определение:} Условным рядом/законом распределения случайной величины X при условии $Y = y_j$ называется набор вероятностей\\
$\Pi_{ij} = \cfrac{p_{ij}}{p_{Y_j}}, \ i = \overline{1,m}$\\
(для каждого $j \in \{1, \ldots, n\}$ свой условный закон распределения).\\


\underline{Замечание:} Условные законы распределения случайной величины Y при условии $X = x_i$ определяется аналогично - это наборы вероятностей\\
$\tau_{ij} = \cfrac{P_{ij}}{p_{X_i}}, \ j = \overline{1,n}$\\
(для каждого $i \in \{1, \ldots, m\}$ свой закон условного закона распределения).\\


\underline{Пример:} Двумерный дискретный случайный вектор (X,Y) имеет закон распределения, заданный таблицей\\

% Вставить таблицу

\begin{enumerate}
	\item[1.] Найти условный закон распределения X и ...
	\item[2.] Построить графики\\
	$F_X(x|Y = 1)$\\
	$F_Y(y|X = 0)$\\
\end{enumerate}
\underline{Решение:}
\begin{enumerate}
	\item[1)] Пусть $Y = 0$ (то есть $Y = y_1, \ j = 1$).\\
	Условный закон распределения случайной величины X в этом случае задается набором вероятностей\\
	$\Pi_{i1} = \cfrac{p_{i1}}{p_{Y_1}}, \ \overline{1,3}$\\
	
	\begin{tabular}{|c||c|c|}
	\hline 
	 & 0 & 1 \\ 
	\hline \hline
	-1 & $\cfrac{0.3}{04} = \cfrac{3}{4}$ & $\cfrac{0.2}{0.6} = \cfrac{2}{6}$ \\ 
	\hline 
	0 & $\cfrac{0.1}{0.4} = \cfrac{1}{4}$ & $\cfrac{0.3}{0.6} = \cfrac{3}{6}$ \\ 
	\hline 
	2 & $\cfrac{0}{0.4} = 0$ & $\cfrac{0.1}{0.6} = \cfrac{1}{6}$ \\ 
	\hline \hline
	 & 1 & 1 \\ 
	\hline 
	\end{tabular} \\
	% Вставить пояснения к таблице
	
	\begin{tabular}{|c||c|c|c|}
	\hline 
	X & -1 & 0 & 2 \\ 
	\hline 
	P & $\cfrac{1}{3}$ & $\cfrac{1}{2}$ & $\cfrac{1}{6}$ \\ 
	\hline 
	\end{tabular} \\
	%Вставить график функции
	
	\item[2)] Условные законы распределения случайной величины Y.\\

	\begin{tabular}{|c||c|c|c|}
	\hline 
	 & 0 & 1 &  \\ 
	\hline 
	\hline
	-1 & $\cfrac{0.3}{0.5} = \cfrac{3}{5}$ & $\cfrac{0.2}{0.5} = \cfrac{2}{5}$ & 1 \\ 
	\hline 
	0 & $\cfrac{1}{4}$ & $\cfrac{3}{4}$ & 1 \\ 
	\hline 
	2 & 0 & 1 & 1 \\ 
	\hline 
	\end{tabular} 

	Условный закон распределения случайной величины Y при X = -1 (то есть $X = x_i$)\\
	$\tau_{1j} = \cfrac{p_{ij}}{p_{X_i}}, \ i = \overline{1,2}$\\
	$F_Y(y|X = 0) = P\{Y < y|X = 0\}$\\
	
	\begin{tabular}{|c||c|c|}
	\hline 
	Y & 0 & 1 \\ 
	\hline 
	P & $\cfrac{1}{4}$ & $\cfrac{3}{4}$ \\ 
	\hline 
	\end{tabular}  \\
	
	% Вставить график
\end{enumerate}


\textbf{II.} Пусть
\begin{enumerate}
	\item[1)] (X,Y) - непрерывный случайный вектор.
	
	\item[2)] f(x,y) - совместная плотность распределения X и Y.
	
	\item[3)] $f_X(x), f_Y(y)$ - маргинальные плотности X и Y.
\end{enumerate}
Рассмотрим условную функцию распределения случайной величины X при условии $Y = y$.\\
$F_X(x|Y = y) = P\{X < x|Y = y\}$\\
Производная по x этой функции - условная плотность распределения случайной величины X при условии Y = y.\\
$f_X(X|Y = y)$ = (можно показать) = $\cfrac{f(x,y)}{f_Y(y)}$\\
\underline{Замечание:} Аналогично:\\
$f_Y(y|X = x) = \cfrac{f(x,y)}{f_X(x)}$ - условная плотность распределения случайной величины Y при условии X = x.\\


\underline{Пример:} Совместная плотность распределения случайной величины X и Y имеет вид:\\
$f(x,y) = 
\begin{cases}
	10y, \ (x,y) \in D\\
	0, \ \text{иначе}\\
\end{cases}$\\
% Вставить график
Найти условные плотности распределения случайных величин X и Y.\\
\underline{Решение:} 
\begin{enumerate}
	\item[1)] Найдем маргинальные плотности распределения X и Y.\\
	% Вставить график
	$\displaystyle  f_X(x) = \int\limits_{-\infty}^{+\infty} f(x,y) dy = \left\{
	\begin{array}{lll}
		0, \ \text{если} \ x \not\in (0;1)\\
		\int\limits_0^{x^2} 10y dy, \ x \in (0;1)\\
	\end{array} \right\{ = 
	\begin{cases}
		0, \ \text{если} \ x \not\in (0;1)\\
		5x^4, \ x \in (0;1)\\
	\end{cases}$\\

	$f_X(x) = 
	\begin{cases}
		5x^4, \ x \in (0;1)\\
		0, \ \text{если} \ x \not\in (0;1)\\
	\end{cases}$\\
	
	$\displaystyle  f_Y(y) = \int\limits_{-\infty}^{+\infty} f(x,y) dx = \left\{
	\begin{array}{lll}
		0, \ y \not\in (0;1)\\
		\int\limits_{\sqrt{y}}^1 10y dx, \ y \in (0;1)\\
	\end{array} \right\} = 
	\begin{cases}
		10y (1 - \sqrt{y}), \ y \in (0;1)\\
		0, \ \text{иначе}\\
	\end{cases}$\\
	
	$f_Y(y) = 
	\begin{cases}
		10y(1 - \sqrt{y}), \ y \in (0;1)\\
		0, \ \text{иначе}\\
	\end{cases}$\\
	
	\item[2)] Условные плотности\\
	% Вставить график
	$f_X(x|Y = y) = \cfrac{x,y}{f_Y(y)} = \left\{
	\begin{array}{lll}
		\text{не опр.}, \ y \not\in (0;1)\\
		0, \ y \in (0;1), \ x \not\in (\sqrt{y}; 1)\\
		\frac{10y}{10y(1 - \sqrt{y})}, \ y \in (0;1), \ x \in (\sqrt{y}; 1)\\
	\end{array} \right\} = 
	\begin{cases}
		\frac{1}{1 - \sqrt{y}}, \ y \in (0;1), \ x \in (\sqrt{y}; 1)\\
		0, \ y \in (0;1), \ x \not\in (\sqrt{y}; 1)\\
		\text{не опр.}, \ y \in (0;1)\\
	\end{cases}$\\
	
	% Вставить график
	$f_Y(y|X = x) = \cfrac{f(x,y)}{f_X(x)} = \left\{
	\begin{array}{lll}
		\text{не опр.}, \ x \not\in (0;1)\\
		0, \ x \in (0;1), \ y \not\in (0;x^2)\\
		\frac{10y}{5x^4}, \ x \in (0;1), \ y \in (0; x^2)\\
	\end{array} \right\} = 
	\begin{cases}
		\frac{2y}{x^4}, \ x \in (0;1), \ y \in (0; x^2)\\
		0, \ x \in (0;1), \ y \not\in (0; x^2)\\
		\text{не опр.}, \ x \not\in (0;1)\\
	\end{cases}$\\
\end{enumerate}


















