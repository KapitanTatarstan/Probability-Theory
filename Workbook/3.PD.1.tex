% 1. Классическое определение вероятностей.

Пусть \\
\begin{enumerate}
\item[1)] $\Omega$ - пространство элементарных исходов некоторого случайного эксперимента. \\

\item[2)] $|\Omega| = N < \infty$ \\

\item[3)] по условиям эксперимента нет оснований предпочесть тот или иной исход.
\end{enumerate}

\underline{Определение}: Вероятностью осуществления события A называется число \\
$P(A) = \cfrac{N_A}{N}$ \\


\underline{Пример}: Бросают 2 игральные кости. \\
A = \{на обеих костях выпало одинаковое число очков\}. \\
B = \{сумма выпавших очков четная\}. \\
C = \{произведение выпавших очков = 6\}. \\
P(A), P(B), P(C) - ? \\
\underline{Решение}: 
\begin{enumerate}
\item[1)] Исход: $(x_1, x_2)$, где $x_i$ - количество очков выпавшей на i-ой кости. \\
$(x_1, x_2)$ - размещение с повторениями из 6 по 2. \\
N = 36 \\

\item[2)] $A = \{ (1,1), \ldots , (6,6)\}$ \\
$N_A = 6$ \\
$P(A) = \cfrac{N_A}{N} = \cfrac{6}{36} = \cfrac{1}{6}$ \\

\item[3)] $B = \{
\begin{array}{lll} 
	(1,1) & (1,3) & (1,5) \\
	(2,2) & (2,4) & (2,6) \\
    ... \\
	(6,2) & (6,4) & (6,6) \\
\end{array} \}$
$N_B = |B| = 18$ \\
$P(B) \ cfrac{N_B}{N} = \cfrac{1}{2}$

\item[4)] $C = \{x_1x_2 = 6\} = \{ (1,6), (2,3), (3,2), (6,1)\}$ \\
$N_C = |C| = 4$ \\
$P(C) = \cfrac{N_C}{N} = \cfrac{1}{9}$ \\
\end{enumerate}


\underline{Пример}: Из колоды домино наудачу извлекают одну кость. \\
A = \{это дубль\} \\
B = \{на кости ровно одна пустышка\} \\
P(A)? P(B) = ? \\

\begin{enumerate}
\item[1)] 
%Вставить рисунок
$0 \leqslant m \leqslant n \leqslant 6$ \\
Исход: $(x_1, x_2)$, где $x_1 \leqslant x_2, \ x_i \in \{0, \ldots , 6\}$ \\
$N = 28$ \\

\item[2)] $A = \{(0,0), \ (1,1), \ \ldots, (6,6)\}$ \\
$N_A = 7$ \\
$P(A) = \cfrac{N_A}{N} = \cfrac{7}{28} = \cfrac{1}{4}$ \\ 

\item[3)] $B = \{(0,1), \ (0,2), \ldots , (0,6)\}$ \\
$N_B = |B| = 6$ \\
$P(B) = \cfrac{N_B}{N} = \cfrac{6}{28} = \cfrac{3}{14}$ \\
\end{enumerate}

























