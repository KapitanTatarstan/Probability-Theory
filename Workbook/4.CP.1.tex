% 1. Условная вероятность.

Пусть \\
\begin{enumerate}
\item[1)] $(\Omega, \beta, P)$ - вероятностное пространство \\

\item[2)] $A, B \in \beta$ \\

\item[3)] $P(B) > 0$ \\
\end{enumerate}
\underline{Определение}: Условной вероятностью осуществления события A при условии, что произошло B называется число \\
$P(A|B) = \cfrac{P(AB)}{P(B)}$ \\


\underline{Определение}: События A и B называются независимыми, если $P(AB) = P(A)P(B)$ \\
\underline{Th} \\
Пусть \\
\begin{enumerate}
\item[1)] $P(B) > 0$ \\
Тогда A, B - независимые $\Leftrightarrow P(A|B) = P(A)$ \\
\end{enumerate}
\underline{Пример}: 3 раза бросают игральную кость. \\
A = \{результаты трех бросков попарно различны\}. \\
B = \{выпало хотя бы один раз "6"\ \}. \\
$P(A), P(B), P(A|B) = ?$ \\
Указать, зависимы ли A и B. \\
\underline{Решение}: 
\begin{enumerate}
\item[1)] Исход: $(x_1, x_2, x_3)$ (размещение с повторениями из 6 по 3), где $x_i$ - количество очков, при i-ом броске. \\
$N = 6^3$ \\

\item[2)] 
$P(A) = ?$ \\
$N_A = 6 \cdot 5 \cdot 4 = A^3_6$ \\
$P(A) = \cfrac{12-}{6^3}$ \\

\item[3)] $P(B) = ?$ \\
$P(B) = 1 - P(\overline{B})$ \\
$\overline{B}$ = \{"6"\ не выпало ни разу\} \\
$N_{\overline{B}} = 5^3$ \\
$P(B) = 1 -P(\overline{B}) = 1 - \left( \cfrac{5}{6} \right)^3$ \\

\item[4)] $P(A|B) = \cfrac{P(AB)}{P(B)}$ \\
AB = \{все $x_i$ попарно различны и один из них = "6"\ \}.
%Вставить формулы
$N_{AB} = 20 \cdot 3 = 60$ \\
$P(A|B) = \cfrac{\cfrac{60}{6^3}}{1 - \left( \cfrac{5}{6} \right)^3} = \cfrac{60}{6^3 - 5^3} = \cfrac{60}{(6 - 5)(6^2 + 50 + 25)} = \cfrac{60}{91}$ \\

\item[5)] $P(A) \neq P(A|B) \Rightarrow A,B$ - зависимы.
\end{enumerate}

% Семинар № 7.

\underline{Пример}: Из полной колоды в 52 карты случайным образом извлекают 1 карту. \\
A = \{извлечен туз\} \\
B = \{извлечена карта черной масти\} \\
C = \{извлечена картинка\} \\
\begin{enumerate}
\item[1)] Установить, является ли A,B,C независимыми попарно и независимы в совокупности. \\

\item[2)] P(ABC) = ? \\
\end{enumerate}
\underline{Решение}:
\begin{enumerate}
\item[1)] $P(A) = \cfrac{4}{52} = \cfrac{1}{13}$ \\
$P(B) = \cfrac{26}{52} = \cfrac{1}{2}$ \\
$P(C) = \cfrac{16}{52} \cfrac{4}{13}$ \\

\item[2)] $P(AB)$ = (AB = \{туч черной масти\}) = $\cfrac{2}{52} = \cfrac{1}{26}$ \\
$P(AC)$ = (AC = A) = $P(A) = \cfrac{1}{13}$ \\
$P(BC)$ = (BC = \{картинка черной масти\}) = $\cfrac{8}{52} = \cfrac{2}{13}$

\item[3)] \underline{Определение}: События A и B называется независимыми, если $P(AB) = P(A)P(B)$ \\
$P(AB) = \cfrac{1}{26}$ \\
$P(A)P(B) = \cfrac{1}{13} \cdot \cfrac{1}{2} = \cfrac{1}{26}$ \\
$\Rightarrow A,B$ - независимы.

$P(AC) =\frac{1}{13}$ \\
$P(A) \cdot P(C) = \cfrac{1}{13} \cdot \cfrac{4}{13} = \cfrac{4}{169}$ \\
$\Rightarrow A,C$ - зависимы.

$P(BC) = \cfrac{2}{13}$ \\
$P(C) \cdot P(C) = \cfrac{1}{2} \cdot \cfrac{4}{13} = \cfrac{2}{13}$ \\
$\Rightarrow B,C$ - независимы.

A,B,C не является независимыми попарно (так как A и C зависимы). A, B, C не являются независимы в совокупности, так как они не являются попарно независимыми.
\end{enumerate}


\underline{Пример}: В аудитории находится 100 студентов из которых \\
английский знают: 50 человек \\
французский знают: 40 человек \\
немецкий знают: 35 человек \\

английский и французский: 20 человек \\
английский и французский: 8 человек \\
французский и немецкий: 5 человек \\

английский,французский и немецкий: 5 человек \\

Случайно выбранного студента вызывают к доске. \\
A = \{он знает английский\} \\
B = \{он знает французский\} \\
C = \{он знает немецкий\} \\

\begin{enumerate}
\item[1)] Установить, являются ли A, B, C попарно независимыми и независимыми в совокупности.

\item[2)] $P(AB|C) = ?$
\end{enumerate}
\underline{Решение}: 
\begin{enumerate}
\item[1)] $P(A) = \cfrac{50}{100} = \cfrac{1}{2}$ \\
$P(B) = \cfrac{40}{100} = \cfrac{2}{5}$ \\
$P(C) = \cfrac{35}{100} = \cfrac{7}{20}$ \\ 

\item[2)] $P(AB) = \cfrac{20}{100} = \cfrac{1}{5}$ \\
$P(AC) = \cfrac{8}{100} = \cfrac{2}{25}$ \\
$P(BC) = \cfrac{10}{100} = \cfrac{1}{10}$ \\

\item[3)] $P(AB) = \cfrac{1}{5}$ \\
$P(A) \cdot P(B) = \cfrac{1}{2} \cdot \cfrac{2}{5} = \cfrac{1}{5}$ \\
$\Rightarrow A,B$ -- независимы \\

$P(AC) = \cfrac{2}{25}$ \\
$P(A) \cdot P(C) = \cfrac{1}{2} = \cfrac{1}{2} \cdot \cfrac{7}{20} = \cfrac{7}{40}$ \\
$\Rightarrow A,C$ - зависимы \\

$P(BC) = \cfrac{1}{10}$ \\ 
$P(B) \cdot P(C) = \cfrac{2}{5} \cdot \cfrac{1}{10} = \cfrac{1}{25}$ \\
$\Rightarrow B,C$ - зависимы \\

A, B, C - не являются попарно независимы $\Rightarrow$ A, B, C не являются независимы в совокупности.

\item[4)] $P(AB|C) = \cfrac{P(ABC)}{P(C)} = \cfrac{\cfrac{5}{100}}{\cfrac{7}{20}} = \cfrac{1}{20} \cdot \cfrac{20}{7} = \cfrac{1}{7}$ \\
\end{enumerate}






















