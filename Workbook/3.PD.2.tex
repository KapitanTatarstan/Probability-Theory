% 2. Некоторые комбинаторные конфигурации.

При решении задач на классическое определение вероятности приходится подсчитывать число элементов в различных комбинаторных конфигурациях. При этом используется ряд стандартных приемов. \\

I. Сочетания без повторений. \\
Пусть \\
\begin{enumerate}
\item[1)] A - множество \\

\item[2)] $|A| = n$ \\
\end{enumerate}
Без ограничения общности можно считать, что \\
$A = \{1,2, \ldots, n\}$ \\

\underline{Определение}: Сочетанием без повторений из n по m называется любое m-элементное подмножество множества A, то есть набор \\
$\{x_1, \ldots, x_n\}$ \\


\underline{Замечание}:
\begin{enumerate}
\item[1)] в определении подразумевается, что
	\begin{enumerate}
	\item[а)] все входящие в сочетание элементы попарно различны.
	
	\item[б)] сочетание не изменится, если входящие в него элементы записать в другой последовательности например \\
	$\{1,3,10\} = \{3,10,1\}$ \\
	\end{enumerate}
\end{enumerate}


\underline{Th}: Всего $\exists \ C^m_n = \cfrac{n!}{m! (n-m)!}$  (биномиальные коэффициенты) различных сочетаний без повторений из n по m. \\


II. Размещение с повторениями. \\
$A = \{1,2, \ldots, n\}$ \\
\underline{Определение}: Размещение с повторениями называется кортеж (упорядоченный набор): \\
$(x_1, x_2, \ldots, x_m)$ \\
где $x_i \in A, \ i = \overline{1,m}$ \\


\underline{Замечание}: Размещения различаются не только составом элементов но и последовательностью, в которой они записаны. Например \\
$(1,1,3) \neq (1,3,1)$ \\


\underline{Th} Всего $\exists \stackrel{\sim}{A^m_n} = n^m$ различных размещений с повторениями из n по m. \\

%\underline{Замечание}: 
% Вставить рисунок



III. Размещение без повторений. \\
\underline{Определение}: Размещением без повторений из n по m называется кортеж \\
$(x_1, x_2, \ldots, x_m)$, \\
где $x_i \in A, \ i = \overline{1,m}, \ x_i \neq n_j$ при $i \neq j$ \\


\underline{Th} Всего $\exists \ A^m_n = \cfrac{n!}{(n-m)!}$ различных размещений без повторения из n по m.


%\underline{Замечание}
%Вставить рисунок


IV. Перестановка. \\
\underline{Определение}: Перестановка длины n называется размещение без повторений из n по n, то есть кортеж \\
$(x_1, x_2, \ldots , x_n)$, \\
где $x_i \in A, \ i = \overline{1,n}, \ x_i \neq x_j$ при $i \neq j$ \\

\underline{Th}: Число перестановок длины n равно $P_n = A^n_n = \cfrac{n!}{0!} = n!$ \\


V. Схема упорядоченных разбиений. \\
Пусть \\
\begin{enumerate}
\item[1)] имеется n попарно различных шаров
%Вставить рисунок

\item[2)] имеется m попарно различных урн. \\
%Вставить рисунок

\item[3)] За j-ой урной закреплено число $n_j \in \mathbb{N}_0$, причем $n_1 + n_2 + \ldots + n_m = n$ \\
\end{enumerate}
Вопрос: Сколькими способами можно разложить n шаром по m урнам, так, чтобы в j-ой урне лежало $n_j$ шаров? \\
\underline{Пример}:
%Вставить рисунок.

\underline{Th} Общее число способов размещений n шаров по m урнам с учетом сделанных выше ограничений составит \\
$C(n_1, \ldots, n_m) = \cfrac{n!}{n! \cdot \ldots \cdot n_m!)}$ \\

\underline{Пример}: В партии из 10 однотипных изделий 3 изделия являются бракованными. Из партии случайным образом выбираются 3 изделия. \\
A = \{в выборке ровно 1 брак\} \\
B = \{в выборке ровно 2 брака\} \\
C = \{в выборке 3 ровно 3 брака\} \\
P(A), P(B), P(C) = ? \\
\underline{Решение}:
\begin{enumerate}
\item[1)] 10 шаров
%Вставить рисунок
Исход: $\{x_1, x_2, x_3\}$, где $x_i$ - номер извлеченного шара. \\
Сочетание без повторения из 10 по 3 \\
$N = C^3_10 = \cfrac{10 \cdot 9 \cdot 8}{2 \cdot 3} = 120$ \\

\item[2)] $P(A) = ?$ \\
$\{ \underbrace{x_1}_{\text{брак}}, \underbrace{x_2, x_3}_{\text{не брак}} \}$ \\
$N_A = 3 \cdot 21 = 63$ \\
$P(A) = \cfrac{61}{120} = \cfrac{21}{40}$ \\

\item[3)] $P(B) = ?$ \\
$\{ \underbrace{x_1}_{\text{не брак}}, \underbrace{x_2, x_3}_{\text{брак}} \}$ \\
$N_B = 3 \cdot 7 = 21$ \\
$P(B) = \cfrac{21}{120} = \cfrac{7}{40}$ \\

\item[4)] $P(C) = ?$ \\
$\{ \underbrace{x_1, x_2, x_3}_{\text{брак}} \}$ \\
$N_C = 1$ \\
$P(C) = \cfrac{1}{12}$ \\
\end{enumerate}


\underline{Пример}: 10 вариантов контрольной работы написаны на 10-ти отдельных картах. Варианты раздаются 8-ми сидящим рядом студентам (по 1-ому варианту в руки). \\
A = \{варианты 1 и 2 не будут использоваться\} \\
B = \{варианты 1 и 2 достанутся сидящим рядом студентам\} \\
C = \{варианты будут распределены последовательно номера вариантов в порядке возрастания\} \\
P(A), P(B), P(C) = ? \\
\underline{Решение}: 
\begin{enumerate}
\item[1)] 
%Вставить рисунок
Исход: $(x_1, \ldots, x_8)$, \\
где $x_i$ - номер билета, который достался i-ому студенту. \\
Размещение без повторений из 10 по 8 \\
$N = \cfrac{10!}{2!}$ \\

\item[2)] $P(A) = ?$ \\
$(x_1, \ldots, x_8), \ x_i \in \{3,4, \ldots, 10\}$ \\
$N_A = A^8_8 = 8!$ \\
$P(A) = \cfrac{2 \cdot 8!}{10!} = \cfrac{1}{45}$ \\ 

\item[3)] $P(B) = ?$ \\
%Вставить формулы

$N_B = 2 \cdot A^6_8 \cdot 7 = 14 \cdot A^6_8$ \\
$P(B) = \cfrac{14 \cdot 8! \cdot 2}{2! \cdot 10!} = \cfrac{14}{9 \cdot 10} = \cfrac{7}{45}$ \\

\item[4)] $P(C) = ?$ \\
$\begin{array}{lll}
	(1,2, \ldots, 8) \\
	(2,3, \ldots, 9) \\
	(3,4, \ldots, 10) \\
\end{array}$ \\
$N_C = 3$ \\
$P(C) = \cfrac{3 \cdot 2}{10!}$ \\
\end{enumerate}


\underline{Пример}: Телефонная книга раскрывается наудачу и выбирается случайный телефонный номер. Считает, что в номере 7 цифр и все номера равно возможные. Найти вероятности следующих событий: \\
A = \{4 последние цифры одинаковы\} \\
B = \{все цифры попарно различны\} \\
C = \{1-я цифра нечетная\} \\
\underline{Решение}: 
\begin{enumerate}
\item[1)] Исход: $(x_1, x_2, \ldots, x_7)$, \\
где $x_i \in \{0, \ldots 9\}$ - i-ая цифра номера. \\
Размещение с повторениями из 10 по 7. \\
$N = \stackrel{\sim}{A^7_10} = 10^7$ \\

\item[2)] $P(A) = ?$ \\
$( \underbrace{x_1, x_2, x_3}_{10^3}, \underbrace{x_4, x_4, x_4, x_4}_{10} )$ \\
$N_A = 10^4$ \\
$P(A) = \cfrac{N_A}{N} = \cfrac{10^4}{10^7} = \cfrac{1}{10^3}$ \\

\item[3)] $P(B) = ?$ \\
$(x_1, \ldots, x_7), \ x_i \neq x_j$ при $i \neq j$ \\
$N_B = A^7_10 = \cfrac{10!}{3!}$ \\
$P(B) = \cfrac{10!}{3! \cdot 10^7} = \cfrac{10!}{3! \cdot 10^7}$ \\

\item[4)] $PC() = ?$ \\
$\stackrel{\sim}{A^6_10} = 10^6$ \\
%Вставить формулу
$x_1 \in \{ 1,3,5,6,9\}$ \\
$N_C = 5 \cdot 10^6$ \\
$P(C) = \cfrac{5 \cdot 10^6}{10^7} = \cfrac{1}{2}$ \\ 
\end{enumerate}


\underline{Пример}: На почту поступило 6 телеграмм. Их случайным образом распределяют по 4-ем каналам для обработки. \\
A = \{на 1-ом канале окажется 3 телеграммы, на 2-ом - 2 телеграммы, на 3-ем - 1 телеграмма, на 4-ом - 0 телеграмм\}. \\
P(A) = ? \\
\underline{Решение}: 
\begin{enumerate}
\item[1)] 
%Вставить рисунок
Исход: $(x_1, x_2, \ldots, x_6)$ - Размещение с повторениями, где $x_i$ - номер канала, на который попала i-ая телеграмма. \\
$\stackrel{\sim}{A^4_6} = 4^6$ \\

\item[2)]
%Вставить рисунок 
$N_A = C(3,2,1,0) = \cfrac{6!}{3! \cdot 2! \cdot 1 \cdot 1} = \cfrac{4 \cdot 5 \cdot 6}{2} = 60$ \\
$P(A) = \cfrac{60}{4^6}$ \\
\end{enumerate}

% Семинар № 8
\underline{Пример}: Партия из 50-ти изделий 4 бракованных. Из партии выбирают 10 изделий случайным образом. \\
A = \{среди выбранных изделий хотя бы одно бракованное\}. \\
P(A) = ? \\
\underline{Решение}: 
\begin{enumerate}
\item[1)]
%Вставить рисунок
Исход: $\{x_1, x_2, x_3, x_4, \ldots, x_10\}$ - сочетание без повторений из 50 по 10. \\
$x_i$ - номер извлеченного изделия. \\
$N = C^10_50 = \cfrac{50!}{10! \cdot 40!}$ \\

\item[2)] $P(A) = ?$ \\
\underline{I способ}. \\
$A = \underbrace{A_1 + A_2 + A_3 + A_4}_{\text{несовместны}}$ \\
$A_i$ - среди выбранных изделий ровно i бракованных, $i = \overline{1,4}$. \\
$P(A_i) = ?$ \\
%Вставить формулу 
$P(A) = \sum\limits_{i = 1}^{4} P(A_i) = \sum\limits_{i = 1}^{4} \left( \cfrac{C^i_4 \cdot C^{10 - i}_{46}}{C^10_50} \right)$ 
%Вставить пояснение в формулу выше.

\underline{II способ}. 
$\overline{A}$ = \{в выборке нет ни одного бракованного изделия\}. \\
$N_{\overline{A}} = C^10_46$ \\
$\{x_1, \ldots, x_10\}$ \\
$P_A(A) = 1 - P(\overline{A}) = 1 - \cfrac{C^10_46}{C^10_50}$ \\
\end{enumerate}


\underline{Пример}: В шкафу находится 10 пар ботинок (все попарно различны). Из шкафа случайным образом вынимают 4 ботика. \\
A = \{из вынутых из шкафа ботинок нельзя составить пару\}. \\
P(A) = ? \\
\underline{Решение}: \\
\begin{enumerate}
\item[1)] %Вставить рисунок
$\{x_1, x_2, x_3, x_4\}$ (сочетание без повторения из 10 по 4), где $x_i$ - номер ботинка. \\
$N = C^4_20$ \\

\item[2)] $P(A) = ?$ \\
\underline{I способ}. \\
$N_A = \cfrac{20 \cdot 18 \cdot 16 \cdot 14}{4!}$ \\

\underline{II способ}. \\
$N_A = C^4_10 \cdot 2 \cdot 2 \cdot 2 \cdot 2 = 2^4 \cdot C^4_10$ \\
$P(A) = \cfrac{N_A}{N}$ \\

\underline{III способ}. \\
$(a_1, x_2, x_3, x_4)$ \\
$N = A^4_20$ \\
$N_A = 20 \cdot 18 \cdot 16 \cdot 14$ \\
\end{enumerate}


\underline{Пример}: 6 пассажиров поднимаются в лифте 7-ми этажного дома. Считая, что движения начинается из подвала, найти вероятности событии: \\
A = \{на первых трех этажах не выйдет никто\}. \\
B = \{все выйдут на первых 6-ти этажах\}. \\
C = \{все выйдут на 1-ом этаже\}. \\
D = \{на 5-ом, 6-ом, 7-ом этажах выйдут по два человека\}. \\
\underline{Решение}: 
\begin{enumerate}
\item[1)] $(x_1, x_2, \ldots x_6)$ - размещение с повторениями из 7 по 6. \\
$x_i$ - этаж, на котором вышел i-й человек. \\
$N = 7^6$ \\

\item[2)] $P(A) = ?$ \\
$(x_1, x_2, \ldots, x_6)$ \\
$x_i \in \{4,5,6,7\}$ \\
$N_A = 4^6$ \\
$P(A) = \cfrac{N_A}{N} = \cfrac{4^6}{7^6} = \left( \cfrac{4}{7} \right)^6$ \\

\item[3)] $P(B) = ?$ \\
$(x_1, x_2, \ldots, x_6)$ \\
$x_i \in \{1,2,\ldots, 6\}$ \\
$N_B = 6^6$ \\
$P(B) = \cfrac{N_B}{N} = \left( \cfrac{6}{7} \right)^6$ \\

\item[4)] $P(C) = ?$ \\
$(x_1, x_2, \ldots, x_6), \ x_i \in {1}$ \\
$C = \{ (1,1,1,1,1,1)\}$ \\
$N_C = 1$ \\
$P(C) = \cfrac{1}{7^6}$ \\

\item[5)] $P(D) = ?$ \\
%Вставить формулу
Каждый кортеж из события D однозначно определяется номерами двух позиций в которых стоят две "5"\ и номерами в которых стоят две "6"\ . \\
%Вставить рисунок
\underline{Схема упорядоченных разбиений}. \\
%Вставить рисунок
$N_D = C_6(2,2,2) = \cfrac{6!}{2! \cdot 2! \cdot 2!}$ \\
$P(D) = \cfrac{N_D}{N}$ \\

\underline{Замечание}: $N_D$ можно подсчитать так: \\
$N_D = C^2_6 \cdot C^2_4 \cdot 1 = \cfrac{6!}{2! \cdot 4!} = \cfrac{4!}{2! \cdot 2} = C_6 (2,2,2)$ \\
%Добавить пояснения в формулу выше.
\end{enumerate}






















