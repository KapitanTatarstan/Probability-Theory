% 1. Функции распределения случайного вектора. 

Пусть $X_1, \ldots, X_n$ - случайные величины, заданные на одном вероятностном пространстве. \\

\underline{Определение:} n-мерным случайным вектором называется кортеж $(X_1, X_2, \ldots, X_n)$. \\

Закон распределения случайного вектора удобно задавать с использованием функции распределения. \\

\underline{Определение:} Функцией распределения вероятностей случайного вектора $(X_1, \ldots, X_n)$ называется отображение \\
$F: \ \mathbb{R}^n \to \mathbb{R}$ \\
определенное правилом $F(x_1, \ldots, x_n) = P\{X_1 < x_1, \ldots, X_n < x_n\}$.

\underline{Замечание:} На двумерный ($n = 2$) случай вектор $(X_1, X_2)$ можно смотреть как на случайный эксперимент, в котором на плоскость бросают точку. Значение $F(x_1^o, x_2^o)$ - функция распределения этого вектора в точке $(x_1^o, x_2^o)$ равно вероятности того, что брошенная на плоскость точка упадет левее и ниже точки $(x_1^o, x_2^o)$, так как \\
$F(x_1^o, x_2^o) = P\{X_1 < x_1^o, X_2 < x_2^o\}$ \\
% Вставить рисунок


\underline{Пример:} Закон распределения вектора $(X_1, X_2)$ задан таблицей. \\
% Вставить таблицу
Найти функцию распределения вероятностей вектора $(X_1, X_2)$. \\
\underline{Решение:} \\
% Вставить график
$F(x_1, x_2) = P\{X_1 < x_1, X_2 < x_2\} = 
\begin{cases}
	0, \ x_1 \leqslant 1 \\
	0, \ 1 < x_1 \leqslant 2, \ x_2 \leqslant -1 \\
	0.15, \ 1 < x_1 \leqslant 2, \ -1 < x_2 \leqslant 3 \\
	0.45, \ 1 < x_1 \leqslant 2, \ 3 < x_2 \leqslant 5 \\
	0.7, \ 1 < x_1 \leqslant 2, \ 5 < x_2 \\
	0, \ 2 < x_1, \ x_2 \leqslant -1 \\
	0.2, \ 2 < x_1, \ -1 < x_2 \leqslant 3 \\
	0.6, \ 2 < x_1, \ 3 < x_2 \leqslant 5 \\
	1, \ 2 < x_1, \ 5 < x_2 \\
\end{cases}$