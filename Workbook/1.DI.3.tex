% 3. Замена переменных в двойном интеграле.




Пусть 
\begin{enumerate}
\item[1)] $\displaystyle  I = \iint\limits_{D_{xy}} f(x, y) dxdy$
%Вставить график

\item[2)] $\Phi: D_{uv} \to D_{xy}$ \\
$\Phi: 
\begin{cases}
	x = x(u, v) \\
	y = y(u, v) \\
\end{cases}$ \\
\end{enumerate}
Тогда при вычислении некоторые условия \\
$\displaystyle  I = \iint\limits_{D_{uv}} f \left( x(u, v), y(u,v) \right) \left|J_\Phi (u,v) \right| cdot dudv$ \\
где $J_\Phi (u,v) = 
\begin{vmatrix}
	x'_u & x'_v \\
	y'_u & y'_v \\
\end{vmatrix}$ \\


\underline{Замечание}: Для нас основное значение будет иметь переход к полярным координатам: \\
$\begin{cases}
	x = \rho \cdot \cos{\varphi} \\
	y = \rho \cdot \sin{\varphi} \\
\end{cases}$ \\
$J_{\text{пол.}} = \rho$ \\
\fbox{$\displaystyle  \iint\limits_{D_{xy}} f(x,y) dxdy = \iint\limits_{D{\rho \varphi}} f(\rho \cdot \cos{\varphi}, \sin{\varphi}) \rho \ d\rho d\varphi$} %Проверить правильность формулы.

\underline{Пример}: В двойном интеграле \\
$\displaystyle  I = \iint\limits_{D_{xy}} f(x,y) dxdy$ \\
перейти к полярным координатам и расставить пределы по новым переменным. \\
\begin{enumerate}
\item[а)] 
%Вставить график 

$\displaystyle  I = \iint\limits_{D_{\rho \varphi}} f (\rho \cdot \cos{\varphi}, \rho \cdot \sin{\varphi}) \  \rho \ d\rho \ d\varphi = \left(
\begin{array}{lll}
	\int d\varphi \int f(\rho \cdot \cos{\varphi}, \rho \cdot \sin{\varphi}) \rho d\rho \\
	\int d\rho \int (\rho \cdot \cos{\varphi}, \rho \cdot \sin{\varphi}) \rho d\varphi \\
\end{array} \right) = 
\int\limits_{0}^{\tfrac{\pi}{4}} d\varphi \int\limits_{0}^{\tfrac{1}{\cos{\varphi}}} f (\rho \cdot \cos{\varphi}, \rho \cdot \sin{\varphi}) \rho \ d\rho$ \\
$x = 1 \Leftrightarrow \rho \cdot \cos{\varphi} = 1 \Leftrightarrow \rho \cfrac{1}{\cos{\varphi}}$ \\



\item[б)] 
%Вставить график 

$\displaystyle  I = \iint\limits_{D_{\rho \varphi}} \polarFunction $ \\
\underline{Решение}: 
$\displaystyle  I = \int\limits_{0}^{\tfrac{\pi}{4}} d\varphi \int\limits_{0}^{\tfrac{1}{\cos{\varphi}}} \polarFunction \rho \ d\rho + \int\limits_{\tfrac{\pi}{4}}^{\tfrac{\pi}{2}} d\varphi \int\limits_{0}^{\tfrac{1}{\sin{\varphi}}} \polarFunction \rho \ d\rho$ \\

\item[в)] $D_{xy}$ ограничена лемнискатой \\
$(x^2 + y^2)^2 = a^2 (x^2 - y^2)$ \\
$x = \rho \cdot \cos{\varphi}$ \\
$y = \rho \cdot \sin{\varphi}$ \\
$\left(\rho^2 \cdot \cos^2{\varphi} + \rho^2 \cdot \sin^2{\varphi} \right)^2 = a^2 \left(\rho^2 (\cos^2{\varphi} - \sin^2{\varphi}) \right)$ \\
$\rho^4 = a^2 \rho^2 (\cos^2{\varphi} - \sin^2{\varphi})$ \\
$\rho^2 = a^2 (\cos^2{\varphi} - \sin^2{\varphi})$ \\
$\rho^2 = a^2 \cdot \cos{2\varphi}$ \\
$\rho = a \cdot \sqrt{\cos{2\varphi}}$ \\
$\cos{2\varphi} \geqslant 0$ \\
$2 \varphi \in \left[ -\cfrac{\pi}{2} ; \cfrac{\pi}{2} \right] \cup \left[ \cfrac{3}{2}\pi ; \cfrac{5}{2} \pi \right] \Rightarrow 
\varphi \left[- \cfrac{\pi}{4} ; \cfrac{\pi}{4}\right] \cup \left[ \cfrac{3}{4} \pi ; \cfrac{5}{4} \pi \right]$ \\
% Вставить график
$\displaystyle  I = \iint\limits_{D_{\rho \varphi}} \polarFunction \polarJacobian = \int\limits_{-\tfrac{\pi}{4}}^{\tfrac{\pi}{4}} d\varphi \int\limits_{0}^{a \sqrt{\cos{2\varphi}}} \polarFunction \rho \ d\rho + \int\limits_{\tfrac{3\pi}{4}}^{\tfrac{5\pi}{4}} d\varphi \int\limits_{0}^{a \sqrt{\cos{2\varphi}}} \polarFunction \rho \ d\rho$ \\
\end{enumerate}

\underline{Пример}: Вычислить \\
$\displaystyle  I = \iint\limits_{D_{xy}} (x^2 + y^2) dxdy$ \\
где $D_{xy}$, ограниченна кривой \\
$x^2 + y^2 = 2ax$ \\
$x^2 - 2ax + a^2 + y^2 = 1^2$ \\
$(x-a)^2 + y^2 = a^2$ \\
%Вставить график
Перейдем в полярную систему координат: \\
$\begin{cases}
	x = \rho \cdot \cos{\varphi} \\
	y = \rho \cdot \sin{\varphi} \\
\end{cases}$  \\
$f(x,y) = x^2 + y^2$ \\
$\polarFunction = \rho^2$ \\
Уравнение границы области $D_{xy}$ \\
$x^2 + y^2 = 2ax$ \\
$\rho^2 \cos^2{\varphi} + \rho^2 \sin^2{\varphi} = 2 a \rho \cos{\varphi}$ \\
$\rho^2 = 2 a \rho \cos{\varphi}$ \\ 
$\rho = 2 a \cos{\varphi}$ \\
$\displaystyle   I = \iint\limits_{D_{\rho \varphi}} \polarFunction \polarJacobian = \int\limits_{-\tfrac{\pi}{2}}^{\tfrac{\pi}{2}} \int\limits_{0}^{2 a \cos{\varphi}} \rho^2 \cdot \rho \ d\rho = \int\limits_{-\tfrac{\pi}{2}}^{\tfrac{\pi}{2}} \cos^4{\varphi} d\varphi = 
	\left( \int\limits_{-a}^{a} f_\text{четн} (x)dx = 2\int\limits_{0}^{a} f_\text{четн} (x)dx \right) 
= 8a^4 \int\limits_{0}^{\tfrac{\pi}{2}} \cos^4{\varphi} d\varphi = 
	\left(\cos^2{\alpha} = \cfrac{1 + \cos{2\alpha}}{2} \right) 
= 8 a^4 \int\limits_{0}^{\tfrac{\pi}{2}} \left(\cfrac{1 + \cos{2\varphi}}{2}\right)^2 d\varphi = 
2 a^4 \int\limits_{0}^{\tfrac{\pi}{2}} \left( 1 + 2\cos{2\varphi} + \cos^2{2\varphi}\right) d\varphi = 
2 a^4 \int\limits_{0}^{\tfrac{\pi}{2}} \left( 1 + \cfrac{1 + \cos{4\varphi}}{2}\right) d\varphi = 
2 a^4 \int\limits_{0}^{\tfrac{\pi}{2}} \left( \cfrac{3}{2} + \cfrac{\cos{4\varphi}}{2} \right) d\varphi = 
2  a^4 \cdot \cfrac{3}{2} \cdot \cfrac{\pi}{2} = \cfrac{3}{2} \cdot a^4 \pi$ \\


\underline{Пример}: Вычислить \\
$\displaystyle  I = \iint\limits_{D_{xy}} \left( \cfrac{x^2}{a^2} + \cfrac{y^2}{b^2} \right) dxdy$, \\
где область $D_{xy}$ ограниченна эллипсом с полуосями 2a и 2b ($\parallel$-ны Ox и Oy соответственно) и центром в точке 0. \\
\underline{Решение}: \\
Перейдем в полярную систему координат. \\
$\polarFunction = \cfrac{\rho^2 \cos^2{\varphi}}{a^2} + \cfrac{\rho^2 \sin^2{\varphi}}{b^2} = \rho^2 \left( \cfrac{cos^2{\varphi}}{a^2} + \cfrac{sin^2{\varphi}}{b^2} \right)$ \\
В декартовой системе координат \\
%Вставить график
$y = \pm 2b \sqrt{1 - \cfrac{x^2}{4a^2}}$ \\
$\displaystyle  I = \int\limits_{-2a}^{2a} dx \int\limits_{-2b \sqrt{1 - \tfrac{x^2}{4a^2}}}^{2b \sqrt{1 - \tfrac{x^2}{4a^2}}} \left(\cfrac{x^2}{a^2} + \cfrac{y^2}{b^2} \right) dy = 
4 \int\limits_{0}^{2a} \left[ \cfrac{x^2}{a^2} \cdot 2b \sqrt{1 - \cfrac{x^2}{a^2}} + \cfrac{8b^3}{3b^2} \left(1 - \cfrac{x^2}{4a^2} \right) \right]^{\tfrac{3}{2}} dx$ - все сложно \\
Перейдем в обобщенную полярную систему координат \\
$\begin{cases}
	x = a \rho \cos{\varphi} \\
	y = b \rho \sin{\varphi} \\
\end{cases}$ \\

$ J_{\text{обоб. пол. с.к.}} = 
\begin{vmatrix}
	x'_\rho & x'_\varphi \\
	y'_\rho & y'_\varphi \\
\end{vmatrix} 
= ab \rho $ \\
$f \left( a \rho \cos{\varphi}, b \rho \sin{\varphi} \right) = \rho^2$ \\
Уравнение границы области $D_{xy}$:
$\cfrac{x^2}{4a^2} + \cfrac{y^2}{4b^2} = 1$ \\
Перейдем в обобщенную полярную систему координат \\
$\cfrac{\rho^2 \cos^2{\varphi}}{4} + \cfrac{\rho^2 \sin^2{\varphi}}{4} = 1$ \\
$\rho^2 = 4$ \\
$\rho = 2$ \\
$\displaystyle  I = \iint\limits_{D_\text{обю пол. с.к.}} f\left(a \cdot \rho \cdot \cos{\varphi}, b \cdot \rho \cdot \sin{\varphi} \right) a b \rho \ d\rho \ d\varphi = 
a b \int\limits_{0}^{2\pi} d\varphi \int\limits_{0}^{2} \rho \cdot \rho^2 \ d\rho = 
a b \int\limits_{0}^{2\pi} \cfrac{\rho^2}{4} \bigg|_{0}^{2} d\varphi = 
4 a b \int\limits_{0}^{2\pi} d\varphi = 8 a b \pi$\\














