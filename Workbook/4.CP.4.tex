% 4. Формула Байеса

\underline{Th} Пусть \\
\begin{enumerate}
\item[1)] $H_1, \ldots, H_n$ - полная группа событий.

\item[2)] A - событие; $P(A) > 0$ \\
\end{enumerate}
\fbox{
$P(H_i|A) = \cfrac{P(A|H_i) \cdot P(H_i)}{\underbrace{P(A|H_1) \cdot P(H_1) + \ldots + P(A|H_n) \cdot P(H_n)}_{= P(A) \text{см. формулу полной вероятности}}}$
} - формула Байеса. \\


\underline{Пример}: В ящике лежит шар неизвестного цвета - с равной вероятностью белый или черный. В ящик кладут белый шар, шары тщательно перемешивают и вынимают 1 шар. Найти вероятность того, что в урне изначально был белый шар, если известно, что извлеченный шар оказался белым. \\
\underline{Решение}: \\
\begin{enumerate}
\item[1)] Введем полную группу событий. \\
$H_1$ = \{в урне изначально был белый шар\} \\
$H_2$ = \{в урне изначально был черный шар\} \\
$P(H_1) = P(H_2) = \cfrac{1}{2}$ \\

\item[2)] A = \{извлеченный шар - белый\} \\
По формуле полной вероятности \\
$P(A) = \underbrace{P(A|H_1)}_{1} \cdot \underbrace{P(H_1)}_{\tfrac{1}{2}} + \underbrace{P(A|H_2)}_{\tfrac{1}{2}} \cdot \underbrace{P(H_2)}_{\tfrac{1}{2}} = \cfrac{1}{2} + \cfrac{1}{4} = \cfrac{3}{4}$ \\

\item[3)] $P(H_1|A)$ = (формула Байеса) = $\cfrac{P(A|H_1) \cdot P(H_1)}{P(A)} = \cfrac{cfrac{1}{2}}{cfrac{3}{4}} = \cfrac{4}{6} = \cfrac{2}{3}$ \\
\end{enumerate}
\underline{Замечание}: Вероятности $P(H_i), \ i = \overline{1,n}$ называются априорными, так как известны до проведения эксперимента. \\
Вероятности $P(H_i|A), \ i = \overline{1,n}$ называются апостериорными (известны после проведения эксперимента). \\
В рассмотреном примере \\
$P(H_1) = \cfrac{1}{2}$ \\
$P(H_i|A) = \cfrac{2}{3}$ \\
Учет дополнительной информации об исходе эксперимента привел к тому, что \\
$P(H_i|A) > P(H_i)$ \\ %Проверить корректность выражения


\underline{Пример}: (см ранее) \\
В баскетбольной команде 12 игроков. \\
4 игрока выполняют трех очковый бросок с вероятностью 0.95 \\
5 игрока выполняют трех очковый бросок с вероятностью 0.8 \\
3 игрока выполняют трех очковый бросок с вероятностью 0.5 \\
Случайно выбранный игрок выполнил трех очковый бросок. К какой части команды он вероятнее всего принадлежит? \\
\underline{Решение}: \\
\begin{enumerate}
\item[1)] Полная группа события \\
$H_1$ = \{выбран игрок из отличной группы\} \\
$H_2$ = \{выбран игрок из хорошей группы\} \\
$H_3$ = \{выбран игрок из посредственной группы\} \\

A = \{выбранный игрок выполнил 3-х очковый бросок\}. \\
$P(A) = ...$ (см выше). \\

\item[2)] Формула Байеса \\
$P(H_1|A) = \cfrac{P(A|H_1) \cdot P(H_1)}{P(A)} = \cfrac{0.95 \cdot 4}{12 \cdot P(A)} = \cfrac{3.8}{12 \cdot P(A)}$ \\
$P(H_2|A) = \cfrac{P(A|H_2) \cdot P(H_2)}{P(A)} = \cfrac{0.8 \cdot 5}{12 \cdot P(A)} = \cfrac{4}{12 \cdot P(A)}$ - MAX \\
$P(H_3|A) = \cfrac{P(A|H_3) \cdot P(H_3)}{P(A)} = \cfrac{0.5 \cdot 3}{12 \cdot P(A)} = \cfrac{1.5}{12 \cdot P(A)}$ \\
\end{enumerate}
\underline{Ответ}: Вероятнее всего этот игрок из хорошей группы.























