% 1. Функция распределения случайной величины.

\underline{Определение:} (нестрогое) \\
Пусть исход случайного эксперимента можно описать числом X. \\
Тогда X - случайная величина. \\


Пусть $(\Omega, \beta, P)$ - вероятностное пространство. \\
\underline{Определение:} Случайной величиной называется отображение $X: \Omega \to \mathbb{R}$ такое, что $\forall x \in \mathbb{R}$ множество $\{w \in \Omega: \ X(w) < x\} \in \beta$ \\ %Вставить пояснение

\underline{Замечание:} 
\begin{enumerate}
\item[1)] На случайную величину можно смотреть как на случайный эксперимент, в котором на числовую прямую бросают точку. \\
%Вставить рисунок
При этом координата $x_0$ падения точки является реализация рассматриваемой случайной величиной.

\item[2)] При многократном повторении такого эксперимента в различные области на прямой с различной частотой. \\
\underline{Определение:} Законом распределения случайной величины называют правило, которое разл. значениям (различными областям на прямой) ... вероятности, с которыми случайная величина принимает эл. значениям (попадает в эти области на прямой).

\item[3)] Универсальным способом задания закона распределения случайной величины является использование функции распределения.
\end{enumerate}


\underline{Определение:} Функцией распределения вероятностей случайной величины X называется отображение $F: \ \mathbb{R} \to \mathbb{R}$, определенное правилом $F(x) = P\{X < x\}, \ x \in \mathbb{R}$ \\

\underline{Замечание:} \\
%Вставить рисунок


\underline{Пример:} В ящике находится 5 шаров: 2 белых и 3 черных. Из ящика случайным образом вынимают 2 шара (без возвращения). X - количество белых шаров среди извлеченных. \\
Найти функцию распределения случайной величины X. \\
\underline{Решение:} \\
\begin{enumerate}
\item[1)] $X \in \{0, 1, 2\}$ \\
%Вставить таблицу 
$P\{X = 0\} = \cfrac{3}{10}$ \\
$\{X = 0\}$ = \{1-ый шар черный\}$\cdot$\{2-ой шар черный\} \\
$P\{X = 0\} = P\{A_1 \cdot A_2\}$ = (th умножения) = $\underbrace{P(A_1)}_{\frac{3}{5}} \underbrace{P(A_2|A_1)}_{\frac{2}{4}} = \cfrac{3}{10}$ \\
$P\{X = 2\} = \cfrac{1}{10}$ \\
$P\{X = 1\}$ = P\{ \{1-ый шар белый, 2-ой шар черный\} + \{1-ый шар черный, 2-ой шар белый\} \} = (th умножения) = $P(B_1) + P(B_2) = \cfrac{2}{5} \cdot \cfrac{3}{4} + \cfrac{3}{5} \cdot \cfrac{2}{4} = \cfrac{12}{20} = \cfrac{6}{10}$ \\
$P\{X = 1\} = \cfrac{6}{10}$ \\

\item[2)] $F(x) = P\{X < x\}$ \\
%Вставить график
$F(x_1) = P\{X < x_1\} = 0$ \\
$F(0) = P\{X < 0\} = 0$ \\
$F(x_2) = P\{X < x_2\} = P\{X = 0\} = \cfrac{3}{10}$ \\
$F(1) = P\{X < 1\} = \cfrac{3}{10}$ \\
$F(x_3) = P\{X < x_3\} = P \left\{ \underbrace{\{X = 0\} + \{X = 1\}}_{\text{несовместны}} \right\} = P\{X = 0\} + P\{X = 1\} = \cfrac{3}{10} + \cfrac{6}{10} = \cfrac{9}{10}$ \\
$F(2) = \cfrac{9}{10}$ \\
$F(x_4) = P\{X < x_4\} = 1$ \\
$F(x) = 
\begin{cases} 
	0, \ x \leqslant 0 \\
	\tfrac{3}{10}, \ 0 < x \leqslant 1 \\
	\tfrac{9}{10}, \ 1 < x \leqslant 2 \\
	1, \ 2 < x \\
\end{cases}$
\end{enumerate} 


\underline{Замечание:} Этот пример иллюстрирует свойства функции распределения. \\


\underline{Свойства функции распределения:} \\
\begin{enumerate}
\item[$1^o$] $0 \leqslant F(x) \leqslant 1$ 
\item[$2^o$] F является неубывающей
\item[$3^o$] $\lim\limits_{x \to -\infty} F(x) = 0$, $\lim\limits_{x \to +\infty} F(x) = 1$
\item[$4^o$] В каждой точке $x_0 \in \mathbb{R}$ непрерывна слева, то есть $\lim\limits_{x \to x_0} F(x) = F(x_0)$ 
\item[$5^o$] $P\{a \leqslant X < b\} = F(b) - F(a)$
\end{enumerate}

































