% 1. Фунции от дискретных случайных величин.

Если X - дискретная случайная величина, то Y также будет дискретной, так как функция не может принимать больше значений чем не аргумент.\\


\underline{Пример:} Закон распределения случайной величины X задан таблицей\\

\begin{tabular}{|c||c|c|c|c|c|c|}
\hline 
X & -1 & 0 & 1 & 2 & 3 & 5 \\ 
\hline 
P & 0.1 & 0.15 & 0.3 & 0.1 & 0.05 & 0.3 \\ 
\hline 
\end{tabular} 
\\
Найти закон распределения случайной величины $Y = \left| |X - 1| - 1 \right|$.\\
\underline{Решение:}\\

\begin{tabular}{|c||c|c|c|c|c|c|}
\hline 
Y & 1 & 0 & 1 & 0 & 1 & 3 \\ 
\hline 
P & 0.1 & 0.15 & 0.3 & 0.1 & 0.05 & 0.3 \\ 
\hline 
\end{tabular}  \\

\underline{Ответ:} \\

\begin{tabular}{|c||c|c|c|}
\hline 
Y & 0 & 1 & 3 \\ 
\hline 
P & 0.25 & 0.45 & 0.3 \\ 
\hline 
\end{tabular} \\

$P\{Y = 1\} = P\left\{ X \in \{-1; 1.3\} \right\}$\\
$P\{Y = 0\} = P\left\{ X \in \{0; 2\} \right\}$\\
$P\{Y = 3\} = ...$\\










































