% 3. Геометрическое определение вероятности.

Пусть 
\begin{enumerate}
\item[1)] $\Omega \subseteq \mathbb{R}^n$ \\

\item[2)] $\mu (\Omega) \subset \infty$, где $\mu$ - мера множества. 
%Вставить пояснения после формулы

\item[3)] "степень возможности"\ осуществления события $A \subseteq \Omega$ пропорциональна мере множества A и не зависит от формы события A и его расположения внутри $\Omega$. \\
\end{enumerate}


\underline{Определение}: Вероятностью осуществления события A называется число \\
$P(A) = \cfrac{\mu (A)}{\mu (\Omega}$ \\


\underline{Пример}: В отрезке [0; 1] случайным образом выбирают 2 точки. \\
A = \{произведение из координат $< \cfrac{1}{2}$\}. \\
P(A) = ? \\
\underline{Решение}:
\begin{enumerate}
\item[1)]
%Вставить рисунок
Исход: $(x_1, x_2)$б где $x_i$ - координата i-ой точки. \\
$\Omega = [0;1] x [0;1]$ \\

\item[2)] $A = \{x_1 \cdot x_2 < \cfrac{1}{2} \}$ \\
$P(A) = \cfrac{\mu (A)}{\mu (\Omega)}$ \\
$\mu (\Omega) = 1$ \\
$\displaystyle \mu (A) = \cfrac{1}{2} + \int\limits_{\tfrac{1}{2}}^{1} \cfrac{dx}{2x_1} = \cfrac{1}{2} + \cfrac{1}{2} \cdot \ln{x_1} \bigg|_{\tfrac{1}{2}}^{1} = \cfrac{1}{2} + \cfrac{1}{2} \cdot \left( \ln{1} - \ln{\cfrac{1}{2}} \right) = \cfrac{1}{2} + \cfrac{1}{2} \cdot \ln{2}$ \\
\end{enumerate}

































