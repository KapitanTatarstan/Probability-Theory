% 5. Схема Бернулли.

Рассмотрим случайный эксперимент, в ходе которого возможна реализация одного из 2-х элементарных исходов, то есть \\
$\Omega$ = \{успех, неудача\}. \\ %Вставить пояснения в данную строку

Пусть \\
P\{успех\} = $p \in (0;1)$ \\
Тогда \\
P\{неудача\} = $1 - p = q$ - обозначение. \\
\underline{Замечание}: Описанный выше случайный эксперимент будем называть испытанием. \\


\underline{Определение}: Схемой Бернулли называется серия однотипных испытаний, в которой отдельные испытания независимы в совокупности.
\underline{Замечание}: Независимость отдельных испытаний означает, что в ходе всей серии вероятность реализации успеха (и, следовательно, вероятность неудачи) неизменна. \\


\underline{Th}: \\
\begin{enumerate}
\item[1)] $P_n(k) = C^k_n p^k q^{n - k}$ - вероятность осуществления ровно k успехов в серии из n испытаний \\

\item[2)] $P_n (k_1 \leqslant k \leqslant k_2) = \sum\limits_{k = k_1}^{k_2} C^k_n p^k q^{n - k}$ - вероятность того, что в серии из n испытаний число k успехов лежит между $k_1$ и $k_2$. \\

\item[3)] $P_n (l \geqslant 1) = 1 - q^n$ - вероятность того, что в серии из n испытаний произошел хотя бы один успех. \\
\end{enumerate}


\underline{Пример}: 10 раз бросают правильную игральную кость. \\
A = \{"6"\ появится ровно 2 раза\} \\
B = \{"6"\ появится от 2 до 4 раза\} \\
C = \{"6"\ не появится ни разу\} \\
D = \{"6" появится хотя бы 1 раз\} \\
P(A) = ?, P(B) = ?, P(C) = ?, P(D) = ? \\
\underline{Решение}: \\
\begin{enumerate}
\item[1)] Используем схему Бернулли для "успех"\ - выпадение "6"\ ; "неудача"\ - выпадение "1"\ , ... , "5"\ \\
$p = \cfrac{1}{6}$ \\
$q = \cfrac{5}{5}$ \\
n = 10 \\

\item[2)] $P(A) = P^2_10 = C^2_10 p^2 q^8 = \cfrac{10!}{2 \cdot 8!} \cdot \left(\cfrac{1}{6}\right)^2 \cdot\left(\cfrac{5}{6}\right)^8 \approx 0.291$ \\

\item[3)] $P(B) = \sum\limits_{k = 2}^{4} C^k_n p^k q^{n - k} = P(A) + C^3_10\cdot \left(\cfrac{1}{6}\right)^3 \cdot \left(\cfrac{5}{6}\right)^7 + C^4_10 \cdot \left(\cfrac{1}{6}\right)^4 \cdot \left(\cfrac{5}{6}\right)^6 \approx 0.291 + 0.155 + 0.054 \approx 0.5$ \\

\item[4)] $P(C) = P_10(0) = C^0_10 \cdot p^0 \cdot q^10 = 1 \cdot 1 \cdot \left(\cfrac{5}{6}\right)^10 \approx 0.162$ \\

\item[5)] $P(D) = P_10 (l \geqslant 1) = 1 - P(C) \approx 1 - 0.162 \approx 0.838$ \\
\end{enumerate}


\underline{Пример}: ЗРК атакует самолет противника, обстреливая его зенитными ракетами. Вероятность попадания в самолет каждой ракеты равна 0.6; для поражения требуется как минимум 3 попадания. Найти вероятность поражения цели после 6 выстрелов. \\
\underline{Решение}: \\
\begin{enumerate}
\item[1)] $\Omega$ = \{"успешное попадание"\ , "неудачное попадание"\ \} \\
Серия событий - серия выстрелов. \\
Так как вероятность успеха неизменна по отдельные испытания независимы и используем схему Бернулли. \\
$p = 0.6; \ q = 0.4; \ n = 6$ \\

\item[2)] A = \{цель поражена\} \\
$P(A) = P_6 (3 \leqslant k \leqslant 6) = \sum\limits_{k = 3}^{6} C^k_6 p^k q^{n - k} = C^3_6 \cdot 0.6^3 \cdot 0.4^4 + C^4_6 \cdot 0.6^4 \cdot 0.4^2 + C^5_6\cdot  0.6^5 \cdot 0.4^1 + C^6_6 \cdot 0.6^6 \cdot 0.4^0 \approx 0.82$
\end{enumerate}


\underline{Пример}: Вероятность того, что приобретенный лотерейный билет окажется выигрышным, равна 0.01. Сколько нужно купить лотерейных билетов, чтобы вероятность выиграть хотя бы по одному из них была $\geqslant 0.95$? \\
\underline{Решение}: \\
\begin{enumerate}
\item[1)] Используем схему Бернулли. \\
"успех"\ - приобретенный билет выигрышный. \\
$p = 0.01$ \\

\item[2)] Пусть купленно n билетов. Тогда вероятность выиграть хотя бы по одному \\
$P = P_n (k \geqslant 1) = 1 - (1 - p)^n \geqslant 0.95$ \\
$1 - (1 - p^n) \geqslant 0.95$ \\
$(1 - p)^n \leqslant 0.05$ \\
$n \geqslant \log_{1 - p} 0.05 = \cfrac{\ln{0.05}}{\ln{(1 - p)}} = \cfrac{\ln{0.05}}{\ln{0.99}} \approx 298.07$ 
\end{enumerate}
\underline{Ответ}: $n \geqslant 299$. \\
\underline{Замечание}: Вообще схема Бернулли неприменима в это примере, так как отдельные испытания не являются независимыми (например, если в начале серии куплено несколько пустых билетов, то вероятность того, что очередной билет будет выигрышным, увеличивается). \\
Однако, если общий тираж велик, а количество приобретаемых билетов сравнительно невелико, то вероятность успеха изменится несущественно и схема Бернулли удовлетворяет описанному эксперименту.



























