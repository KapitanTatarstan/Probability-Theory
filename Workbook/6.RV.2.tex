% 2. Дискретные случайные векторы.

\underline{Определение:} Случайные вектор $(X_1, \ldots, X_n)$ называется дискретной, если каждая из случайных величин $X_i, \ i = \overline{1,n}$ является дискретной. \\

\underline{Пример:} Закон распределения дискретного случайного вектора задан таблицей. \\
% Вставить таблицу
Найти: \\
\begin{enumerate}
	\item[а)] $F(1.5, 0.5), F(1,2)$
	\item[б)] $P\{-2 < X \leqslant 2, \ 0 \leqslant Y \leqslant 1\}$ 
	\item[в)] Ряды распределения случайных величин X и Y
	\item[г)] $F_X, F_Y$
\end{enumerate}
\underline{Решение:} \\
\begin{enumerate}
	\item[а)] $F(1.5, 0.5) = P\{X < 1.5, Y < 0.5\} = 0.13 + 0.25 = 0.38$ \\
	$F(1,2) = P\{X < 1, Y < 2\} = 0.13 + 0.25 + 0.17 + 0.15 = 0.7$ 
	\item[б)] $P\{ -2 < X \leqslant 2, \ 0 \leqslant Y \leqslant 1\} = 0 + 0.15 + 0.09 = 0.24$ 
	\item[в)] 
	%Вставить таблицу
	%Вставить таблицу
	% Вставить исходную таблицу с P_X и P_Y
	\item[г)] 
	%Вставить график
	$F_X(x) = 
	\begin{cases}
		0, \ x < -2 \\
		0.7, \ -2 < x \leqslant 2 \\
		1, \ x < 2 \\
	\end{cases}$ \\
	% Вставить график
	$F_Y(y) = 
	\begin{cases}
		0, \ y \leqslant - 0.5 \\
		0.19, \ -0.5 < y \leqslant 0 \\
		0.44, \ 0 < y \leqslant 0.5 \\
		0.76, \ 0.5 < y \leqslant 1 \\
		1, \ y < 1 \\
	\end{cases}$ \\
\end{enumerate}






















