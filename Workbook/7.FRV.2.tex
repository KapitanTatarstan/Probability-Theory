% 2. Функции от непрерывных случайных величин.


Пусть
\begin{enumerate}
	\item[1)] X - непрерывная случайная величина.
	\item[2)] $f_X(x)$ - функция плотности случайной величины X.
	\item[3)] $\varphi: \mathbb{R} \to \mathbb{R}$ монотонная функция.
	\item[4)] $\varphi$ - непрерывно дифференцируема.
	\item[5)] $\psi = \varphi^{-1}$ - обратная к $\varphi$ функция.
	\item[6)] $Y = \varphi(X)$
\end{enumerate}
Тогда\\
$f_Y(y) = f_X \left( \psi(y) \right) \cdot |\psi'(x)|$.\\


\underline{Пример:} Плотность распределения случайной величины X имеет вид:\\
$f_X(x) = 
\begin{cases}
	\frac{1}{2} x, \ x \in (0; 2)\\
	0, \ \text{иначе}\\
\end{cases}$\\
Найти плотности распределения случайных величин\\
\begin{enumerate}
	\item[а)] $Y = X^3$
	\item[б)] $Y = X^2 + 1$
\end{enumerate}
\underline{Решение:}
\begin{enumerate}
	\item[а)] 
	 %Вставить график
	$\varphi(x) = x^3$ - монотонная на всей числовой прямой.\\
	Тогда\\
	$f_Y(y) = f_X \left(\psi(y)\right) \cdot |\psi'(y)|$\\
	$y = x^3 \Leftrightarrow x = \sqrt[3]{y}$\\
	Таким образом\\
	% Вставить график
	$f_Y(y) = f_X \left(\psi(y)\right) \cdot |\psi'(y)| = f_X(\sqrt[3]{y}) \cdot \cfrac{1}{3} \cdot \cfrac{1}{\sqrt[3]{y^2}} = \left\{
	\begin{array}{lll}
		0, \ \sqrt[3]{y} \not\in (0; 2)\\
		\cfrac{1}{2} \sqrt[3]{y} \cdot \cfrac{1}{3} \cdot \cfrac{1}{\sqrt[3]{y^2}}, \ \sqrt[3]{x} \in (0; 2)
	\end{array} \right\} =
	\begin{cases}
		0, \ y \in (0; 8)\\
		\cfrac{1}{6 \sqrt[3]{y}}, \ \in (0; 8)\\
	\end{cases}$\\
	% Вставить график
	
	\item[б)] 
	% Вставить график
	$\varphi(x) = x^2 + 1$ - не является монотонной.\\
	Однако $x \in (0; 2) \Rightarrow$ можно считать что $\varphi$ - монотонна (так как при $x \geqslant 0, \ \varphi(x)$ монотонная).\\
	$f_Y(y) = f_X \left(\psi(y)\right) \cdot |\psi(y)|$, \\
	где $\psi(y)$ - обратная к $\varphi$ на участке $x \geqslant$, то есть $\psi(y) = + \sqrt{y - 1}$\\
	$f_Y(y) = \left(x \geqslant 0 \Rightarrow \varphi(x) \geqslant 1\right) = \left\{
	\begin{array}{lll}
		0, \ y < 1\\
		f_X \left(\psi(y)\right) \cdot |\psi'(y)|, \ y > 1\\
	\end{array} \right\} = \left\{
	\begin{array}{lll}
		0, \ y < 1\\
		\cfrac{1}{2} \sqrt{y - 1} \cdot \cfrac{1}{2 \sqrt{y - 1}}, \ \sqrt{y - 1} \in (0; 2)\\
		0, \ \sqrt{y - 1} \not\in (0; 2)\\
	\end{array} \right\} =
	\begin{cases}
		\cfrac{1}{4}, \ y \in (1; 5)\\
		0, \ \text{иначе}\\
	\end{cases}$\\
\end{enumerate}


\underline{Пример:}\\
$X \sin N(m, \sigma^2)$\\
Найти плотности распределения случайных величин\\
\begin{enumerate}
	\item[а)] $Y = e^{X}$
	\item[б)] $Y = |X|$
\end{enumerate}
\underline{Решение:}
\begin{enumerate}
	\item[а)] $\varphi(x) = e^x$\\
	% Вставить график
	$y = e^x \Leftrightarrow x = \ln{y}$, то есть $\psi(y) \ln{y}$\\
	$f_X(x) = \cfrac{1}{\sqrt{2 \pi} \sigma} \cdot e^{- \cfrac{(x - m)^2)}{2 \sigma^2}}, \ x \in \mathbb{R}$\\
	$\varphi$ - монотонная $\Rightarrow$\\
	$f_Y(y) = f_X \left(\psi(y)\right) \cdot |\psi'(y)| = \left\{
	\begin{array}{lll}
		0, \ y < 0\\
		f_X(\ln{y}) \cfrac{1}{y}, \ y > 0\\
	\end{array} \right\} = 
	\begin{cases}
		0, \ y < 0\\
		\cfrac{1}{\sqrt{2 \pi} \sigma y} \cdot e^{- \cfrac{(\ln{y} - m)^2}{2 \sigma^2}}, \ y > 0\\
	\end{cases}$\\
	
	\item[б)]
	% Вставить график
	$\varphi(x) = |x|$ - не является монотонной\\
	$f_Y(y) = \sum\limits_{k = 1}^m f_X \left(\psi_k(y)\right) |\psi'_k(y)|$\\
	где $\psi_1(y), \ldots, \psi_k(y)$ - все решения уравнения $y = \varphi(x)$\\
	$f_Y(y) = 
	\begin{array}{lll}
		0, \ y < 0\\
		(*), \ y > 0\\
	\end{array}$ \fbox{=}\\
	$(*)$: при y > 0\\
	$y = |x| \Leftrightarrow x \pm y$, то есть $\psi_1(y) = -x$, $\psi_x(y) = x$, m = 2\\
	\fbox{=} $\left\{
	\begin{array}{lll}
		0, \ y < 0\\
		f_X \left(\psi_1(y)\right) \cdot \underbrace{|\psi_1'(y)|}_{= 1} + f_X \left(\psi_2(y)\right) \cdot \underbrace{|\psi_2'(y)|}_{= 1}, \ y > 0\\
	\end{array} \right\} = \left\{
	\begin{array}{lll}
		f_X(-y) + f_X(y), \ y > 0\\
		0, \ y < 0\\
	\end{array} \right\} = \left\{
	\begin{array}{lll}
		\cfrac{1}{\sqrt{2 \pi} \sigma} \cdot \left(e^{- \frac{(y - m)^2}{2 \sigma^2}} + e^{- \sqrt{(- y - m)^2}{2 \sigma^2}} \right), \ y > 0\\
		0, \ y < 0\\
	\end{array} \right\} = 
	\begin{cases}
		\cfrac{2}{\sqrt{2 \pi} \sigma} \cdot e^{- \frac{y^2 + m^2}{2 \sigma^2}} \cdot ch\left(\cfrac{my}{\sigma^2}\right), \ y > 0\\
		0, \ y < 0\\
	\end{cases}$\\
\end{enumerate}































