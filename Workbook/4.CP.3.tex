% 3. Формула полной вероятности.

Пусть $(\Omega, \beta, P)$ - вероятностное пространство некоторого случайного эксперимента. \\
\underline{Определение}: Будем говорить, что события $H_1, \ldots, H_n$ образуют полную группу, если \\
\begin{enumerate}
\item[1)] $\sum\limits_{i = 1}^{n} H_i = \Omega$ \\

\item[2)] $H_i H_j = \emptyset$, при $i \neq j$ \\

\item[3)] $P(H_i) > 0, \ i = \overline{1,n}$ \\
\end{enumerate}
%Вставить рисунок


\underline{Th} о формуле полной вероятности \\
Пусть \\
\begin{enumerate}
\item[1)] $H_1, \ldots, H_n$ - полная группа событий \\

\item[2)] A - событие
\end{enumerate}
Тогда \\
\fbox{
$P(A) = P(A|H_1) \cdot P(H_1) + P(A|H_2) \cdot P(H_2) + \ldots + P(A|H_n) \cdot P(H_n)$} \\


\underline{Пример}: В баскетбольной команде 12 игроков, из которых \\
4 выполняют трех очковый бросок с вероятностью 0.95 \\
5 выполняют трех очковый бросок с вероятностью 0.8 \\
3 выполняют трех очковый бросок с вероятностью 0.5 \\
Найти вероятности событий: \\
A = \{случайно выбранный игрок выполнит успешный трех очковый бросок\} \\
B = \{случайно выбранный игрок выполнит трех очковый бросок в серии из 3-х попыток\} \\
\underline{Решение}: \\
\begin{enumerate}
\item[1)] Рассмотрим \\
$H_1$ = \{случайно выбранный игрок из отличной группы\} \\
$H_2$ = \{случайно выбранный игрок из хорошей группы\} \\
$H_3$ = \{случайно выбранный игрок из группы новичков\} \\

Используем формулу полной вероятности: \\
$P(A) = \underbrace{P(A|H_1)}_{0.95} \cdot \underbrace{P(H_1)}_{\tfrac{4}{12}} + \underbrace{P(A|H_2)}_{0.8} \cdot \underbrace{P(H_2)}_{\tfrac{5}{12}} + \underbrace{P(A|H_3)}_{0.5} \cdot \underbrace{P(H_3)}_{\tfrac{3}{12}} = ...$ \\
%Посчитать результат в формуле выше

\item[2)] $P(B) = P(B|H_1) \cdot \underbrace{P(H_1)}_{\tfrac{4}{12}} + P(B|H_2) \cdot \underbrace{P(H_2)}_{\tfrac{5}{12}} + P(B|H_3) \cdot \underbrace{P(H_3)}_{\tfrac{3}{12}}$ \\
$P(B|H_1)$ = (схема испытаний Бернулли: успех - выполнение броска; неудача - невыполнение) = (формула для вычисления вероятности хотя бы одного успеха = $1 - q^n$ $q = 1 - 0.95 = 0.05$) = $1 - (0.05)^3$ \\
$P(B|H_2) = 1 - (0.2)^3$ \\
$P(B|H_3) = 1 - (0.5)^3)$ \\
$P(B) = \left[ 1 - (0.05)^3 \right] \cdot \cfrac{4}{12} + \left[1 - (0.2)^3 \right] \cdot \cfrac{5}{12} + \left[1 - (0.5)^3 \right] \cdot \cfrac{3}{12} = ...$
%Посчитать результат в формуле выше
\end{enumerate}


\underline{Пример}: 10 студентов пришли сдавать экзамен. Иванов и Петров знают 20 билетов из 30, Сидоров знает 15 билетов, остальные знают все билеты. \\
Вероятность сдать экзамен по известному билету составляет 0.85, по неизвестному - 0.1. \\
A = \{случайно выбранный студент группы сдал экзамен\}. \\
\underline{Решение}: \\
$H_1$ = \{выбран Иванов или Петров\}. \\
$H_2$ = \{выбран Сидоров\}. \\
$H_3$ = \{выбран студент, который знает все билеты\}. \\
$P(A) = P(A|H_1) \cdot \underbrace{P(H_1)}_{\tfrac{2}{10}} + P(A|H_2) \cdot \underbrace{P(H_2)}_{\tfrac{1}{10}} + P(A|H_3) \cdot \underbrace{P(H_3)}_{\tfrac{7}{10}}$ \\

$P(A|H_1)$ = \{рассмотрим еще одну группу событий\} \\
$B_1$ = \{получен известный билет\} \\
$B_2$ = \{получен неизвестный билет\} \\
Используя полную вероятность: \\
$P(A|H_1) = 0.85 \cdot \underbrace{\cfrac{20}{30}}_{P(B_1)} + 0.1 \cdot \underbrace{\cfrac{10}{30}}_{P(B_2)} = ...$ \\
%Вычислить результат в формуле выше
Аналогично \\
$P(A|H_2) = 0.85 \cdot \cfrac{15}{30} + 0.1 \cdot \cfrac{15}{30} = ...$ \\
%Вычислить результат в формуле выше
$P(A|H_3) = 0.85 \cdot \cfrac{30}{30} + 0.1 \cdot \cfrac{0}{30} = ...$ \\
%Вычислить результат в формуле выше


























