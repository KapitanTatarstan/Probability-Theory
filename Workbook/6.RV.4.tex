% 4. Независимые случайные величины.


Пусть (X,Y) - двумерный случайный вектор.\\
\underline{Определение:} Случайные величины X и Y называются независимыми, если\\
$F(x,y) = F_X(x) F_Y(y)$,\\
где F - совместная функция распределения X и Y\\
$F_X$ и $F_Y$ - маргинальные функции распределения X и Y.\\


\underline{Th}
\begin{enumerate}
	\item[1)] Если $(X,Y)$ - дискретный случайный вектор, то\\
	$X,Y$ - независимы $\Leftrightarrow p_{ij} \equiv p_{X_i} p_{Y_j}$,\\
	где $p_{ij} = P\left\{(X,Y) = (x_i, y_j)\right\}$\\
	$P_{X_i} = P\{X = x_i\}$\\
	$P_{Y_j} = P\{Y = y_j\}$\\
	
	\item[2)] Если (X,Y) - непрерывный случайный вектор, то\\
	X,Y - независимы $\Leftrightarrow f(x,y) = f_X(x) f_Y(y)$, где\\
	f - совместная плотность распределения X и Y\\
	$f_X, f_Y$ - их маргинальные плотности\\
\end{enumerate}


\underline{Пример:} (см. выше)\\
% Вставить таблицу
$0 \neq 0.3 \cdot 0.25$\\
то есть $P\left\{(X,Y) - (2;0)\right\} \neq P\{X = 2\} \cdot P\{Y = 0\} \Rightarrow X,Y$ - зависимы\\


\underline{Пример:} (см.выше)\\
$F(x,y) = \cfrac{1}{4 \pi^2} \left[ 4 \cdot \arctan{x} \cdot \arctan{y} + 2\pi \cdot \arctan{x} + 2\pi \cdot \arctan{y} + \pi^2 \right]$\\
Было получено:\\
$F_X(x) = \cfrac{\arctan{x}}{\pi} + \cfrac{1}{2}$\\
$F_Y(y) = \cfrac{\arctan{y}}{\pi} + \cfrac{1}{2}$\\
$F(x,y) = F_X(x) \cdot F_Y(y) \Rightarrow X,Y$ - независимы.
\underline{Другой способ:}\\
$f(x,y) = \cfrac{1}{\pi^2 (1 + x^2)(1 + y^2)}$\\
$f_X(x) = \cfrac{1}{\pi (1 + x^2)}$\\
$f_Y(y) = \cfrac{1}{\pi (1 + y^2)}$\\
Так как $f(x,y) \equiv f_X(x) \cdot f_Y(y) \Rightarrow X,Y$ - независимы.\\


\underline{Пример:} (см. выше)\\
$f(x,y) = 
\begin{cases}
	9 x^2 y^2, \ (x,y) \in k\\
	0, \ \text{иначе}\\
\end{cases}$\\
% Вставить график
Было получено:\\
$f_X(x) = 
\begin{cases}
	3x^2, \ x \in (0;1)\\
	0, \ \text{иначе}\\
\end{cases}$\\
$f_Y(y) = 
\begin{cases}
	3y^2, \ y \in (0;1)\\
	0, \ \text{иначе}\\
\end{cases}$\\
Так как $f(x,y) \equiv f_X(x) \cdot f_Y(y) \Rightarrow X,Y$ - независимы.\\

























