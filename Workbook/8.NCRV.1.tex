% 1. Математическое ожидание.


Пусть X - дискретная случайная величина, принимающая значения\\
$x_i, \ i \in I$\\
Математическое ожидание случайной величины X называется число\\
$M[X] = \sum\limits_{i \in I} p_i x_i$\\
где $p_i = P\{X = x_i\}$\\


\underline{Замечание:} если X принимает множество значений, то в определении предполагается, что соответствующий ряд сходится абсолютно.\\


\underline{Пример:} Пусть X имеет ряд распределения\\

\begin{tabular}{|c||c|c|c|c|}
\hline 
X & -1 & 0 & 1 & 4 \\ 
\hline 
P & 0.1 & 0.2 & 0.4 & 0.3 \\ 
\hline 
\end{tabular} \\

Найти MX.
\underline{Решение:}\\
по определению:\\
$MX = \sum\limits_i p_i x_i = -1 \cdot 0.1 + 0 \cdot 0.2 + 1 \cdot 0.4 + 4 \cdot 0.3 = \ldots = 1.5$\\


\underline{Определение:} Математическое ожидание непрерывной случайной величины X называется число\\
$\displaystyle  M[X] = \int\limits_{-\infty}^{+\infty} x f(x) dx$\\
где f - функция плотности распределения случайной величины X.\\


\underline{Замечание:} предполагается, что несобственный интеграл в определении сходится абсолютно. В противном случае, что $\nexists MX$\\


\underline{Пример:} Случайная величина X имеет плотность распределения\\
$f_X(x) = 
\begin{cases}
	\cfrac{1}{2\sqrt{x}}, \ x \in (0; 1)\\
	0, \ x \not\in (0; 1)\\
\end{cases}$\\
Найти MX.\\
\underline{Решение:} \\
$\displaystyle  MX = \int\limits_{-\infty}^{+\infty} x f(x) dx = (f(x) \equiv 0 \text{ вне } (0; 1)) = \int\limits_0^1 x \cfrac{1}{2 \sqrt{x}} dx = \cfrac{1}{2} \int\limits_0^1 \sqrt{x} dx = \cfrac{1}{2} \cdot \cfrac{2}{3} = \cfrac{1}{3}$\\





























