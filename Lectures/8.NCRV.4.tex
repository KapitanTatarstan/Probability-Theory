% 4. Моменты


Пусть X - случайная величина.\\
\underline{Определение:} Моментом k-ого порядка (k-м моментом k-м начальным порядком) случайной величины X называется число\\
$m_k = M\left[X^k\right], \ (k = 1, 2, \ldots)$\\
\underline{Определение:} Центральным моментом k-ого порядка случайной величины X называется число\\
$\overset{\circ}{m_k} = M\left[(x - m)^k\right]$, где $m = MX, \ (k = 1,2,\ldots)$\\


\underline{Замечание:}\\
\begin{enumerate}
	\item[1)] если X - дискретная случайная величина, то\\
	$m_k = \sum\limits_i p_i (x_i - m)^k$\\
	
	\item[2)] если X - непрерывная случайная величина, то\\
	$\displaystyle  m_k = \int\limits_{-\infty}^{+\infty} x^k f(x) dx$, \\
	$\displaystyle  \overset{\circ}{m_k} \int\limits_{-\infty}^{+\infty} (x - m)^k f(x) dx$\\
	
	\item[3)] $m_1 = MX$\\
	$\overset{\circ}{m_2} = DX$\\
	$\overset{\circ}{m_1} = 0$\\
\end{enumerate}