% Дискретные случайные величины.

\underline{Определение:} Случайной величиной X называют дискретной, если множество её возможных значений конечное или счетно.

Закон распределения дискретной случайной величины X можно задать с использованием таблицы:
%Добавить таблицу.

Здесь $p_i = P\{X = x_i\}$ \\
$\sum\limits_{i} p_i = 1$ \\

\underline{Замечание:} Эта таблица называется рядом распределения вероятностей случайной величины.

\underline{Пример:} 
\begin{enumerate}
	\item[1)]
	Пусть X - число выпадения герба при подбрасывании симметричной монеты. \\
	%Вставить таблицу
	
	\item[2)]
	Пусть X - число бросков симметричной монеты до 1-го выпадения герба (не считая тот бросок, при котором выпал герб).
\end{enumerate}
$X \in \{0, 1, 2, \ldots \}$ \\
$P\{X = 0\} = P\{\text{при 1-м броске выпал герб}\} = \cfrac{1}{2}$ \\
$P\{X = 1\} = \{\text{при первом броске - решка, при втором - гебр}\} = P\{(\text{Р,Г})\} = (\text{отдельные испытания независимы}) = P\{\text{Решка}\} P\{\text{Герб}\} = \cfrac{1}{2} \cdot \cfrac{1}{2} = \cfrac{1}{4}$ \\
$P\{X = 2\} = P\{\text{решка, решка, герб}\} = \cfrac{1}{8}$; \\
$P\{X = n\} = \cfrac{1}{2^{n+1}}, \ \ n \in \{0, 1, 2, \ldots\}$ \\
%Вставить таблицу
Проверка (условие нормировки) \\
$\sum\limits_{n = 0}^{\infty} P\{X = n\} = \cfrac{1}{2} + \cfrac{1}{4} + \cfrac{1}{8} + \ldots = \cfrac{1}{2} \left(1 + \cfrac{1}{2} + \cfrac{1}{4} + \ldots\right) = \cfrac{1}{2} + \cfrac{1}{1 - \cfrac{1}{2}} = 1$ \\
































