% 3. Определение тройного интеграла
Пусть \\
\begin{enumerate}
	\item[1)]
	$G \subseteq \mathbb{R}^3$ - тело 
	
	\item[2)]
	$f: G \rightarrow \mathbb{R}$ - функция
\end{enumerate}

Разобъем область G на части, как это было сделано в задаче о вычислении массы тела. \\

Обозначение: $R = \left\{G_1, ... , G_n\right\}$ - разбиение тела G. \\

\underline{Определение:} Диаметром разбиения R тела G - называется число $d(R) = \underset{i = \overline{1,n}}{max \ diam \ G_i}$

\underline{Определение:} Тройным интегралом функции $f(x,y,z)$  по области G называется число \\
$\iiint\limits_{G} f(x,y,z) dxdydz = 
\lim\limits_{d(R) \rightarrow 0} \sum\limits_{i = 1}^{n} f(M_i) \Delta V_i$ \\
где $M_i$, $\Delta V_i$ имеют...


\underline{Свойства тройного интеграла:} \\
Полностью аналогичны свойствам $1^o$ - $9^o$ двойного интеграла. При из записи: \\
$f(x,y) \mapsto f(x,y,z)$ \\
$\iint\limits_{D} f(x,y) dxdy \mapsto \iiint\limits_{G} f(x,y,z) dxdydz$ \\
$D \mapsto G$. \\
 
(записать самостоятельно).
