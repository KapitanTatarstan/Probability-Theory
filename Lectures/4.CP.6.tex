% Схема испытаний Бернулли.

Рассмотрим случайный эксперимент, в результате которого возможна реализация одного из 2-х этих исходов, условно независимых успехом и неудачей, то есть \\
$\Omega = \{0,1\}$, где 0 - "неудача", 1 - "успех". \\
Обозначение: P\{"успех"\} = p \\
Тогда P\{"неудача"\} = 1 - p \\

\underline{Пример}:
\begin{enumerate}
	\item[1)]
	Подбрасывают монету. \\
	"успех" - выпадение герба. \\
	"неудача" - выпадение решки. \\
	
	\item[2)]
	Бросают игральную кость. \\
	"успех" - выпадение "6". \\
	"неудача" - выпадение "1" , ... , "5". \\
	
	\item[3)]
	Наблюдают пол новорожденного. \\
	"успех" - рождение мальчика. \\
	"неудача" - рождение девочки. \\
\end{enumerate}

\underline{Пример}: Схемой испытании Бернулли будем называть серию однотипных экспериментов успешного вида, в которой вероятность реализации успеха не изменяется от эксперимента к эксперименту.

\underline{Замечание}: 
\begin{enumerate}
	\item[1)]
	Условие неизмен. вероятности успеха на протяжении всей серии означает, что отдельным исключением незав. ...
\end{enumerate}

Обозначение: $P_n(K)$ - вероятность того, что в серии из n экспериментов по схеме Бернулли произошло ровно k успехов.

\underline{Th} Пусть проводится серия из n экспериментов по схеме Бернулли с вероятностью p успеха. \\
Тогда
$P_n(k) = C^k_n p^k q^{n-k}$, где $C^k_n = \cfrac{n!}{k! (n-k)!}$ - биномиальные коэффициенты; $q = 1 - p$, $k = \overline{0,n}$.

\underline{Доказательство}.
\begin{enumerate}
	\item[1)]
	Запишем результат проведения серии из n экспериментов с использованием кортежа. \\
	$(x_1, x_2, \ldots , x_n)$ \\
	где  $x_i = 1$, если в i-ом испытании имел место успех, или 0 если имело неудача. \\
	A = \{в серии из n экспериментов произошло k успехов\} = $\left\{ (x_1, \ldots, x_n): \sum\limits_{i=1}^{n} x_i = k \right\}$
	
	\item[2)]
	$|A| = ?$ \\
	Каждый входящий в A кортеж однозначно определяется позицией, в которых стоят 1-цы. \\
	Таких наборов $\exists$ $C^k_n$ штук \\
	То есть $|A| = C^k_n$
	
	\item[3)]
	Рассмотрим исход $(x_1, \ldots , x_n) \in A$. \\
	Вероятность осуществления ...
	$P\left\{x_1, \ldots , x_n) \right\} = $...
\end{enumerate}

\underline{Следствие}:
$P_n(K_1 \leqslant k \leqslant k_2)$\footnote{вероятность того, что число успехов в серии из n экспериментов по схеме Бернулли заключенного между $k_1$ и $k_2$} = $\sum\limits_{j = k_1}^{k_2} C^j_n p^j q^{n-j}$

\underline{Доказательство}
\begin{enumerate}
	\item[1)]
	Обозначим: \\
	$A = \{k_1 \leqslant k \leqslant k_2\}$, k -число успехов. \\
	$A_j = \{k = j\}, \ \ j = \overline{k_1,k_2}$ \\
	Тогда $A = \sum\limits_{i = k_1}^{k_2} A_j$
	
	\item[2)]
	$A_i \cdot A_j$ = \{в серии произошло одновременно ровно j и L успехов\} = 
	$\begin{cases}
		A_j, \ j = L \\
		\emptyset, \ j =neq L \\
	\end{cases}$
	
	$displaystyle P(A) = P \left( \sum\limits_{j = k_1}^{k_2} A_j \right)$ = ($A_j, \ j = \overline{k_1, k_2}$ попарно) = $\displaystyle \sum\limits_{j = k_1}^{k_2} P(A_j) \stackrel{th1}{=} \sum\limits_{j = k_1}^{k_2} C^j_n p^j q ...$
\end{enumerate}

\underline{Следствие 2}:
$P_n(k \geqslant 1)$\footnote{вероятность того, что в серии из n экспериментов по схеме Бернулли произошел хотя бы 1 успех} = $1 - q^n$

\underline{Доказательство} \\
Пусть A = \{в серии ...\} \\
$P(A) = 1 - P(\overline{A} = $ ($\overline{A}$ ни одного успеха в серии) = $1 - P_n(0) = 1 - C^0_n p^0 q^n = 1 - q^n$























