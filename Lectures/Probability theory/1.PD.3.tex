% Классическое определение вероятности.

Пусть 
\begin{enumerate}
	\item[1)]
	$|\Omega| = N$
	
	\item[2)]
	по условию случайного эксперимента нет оснований предпочесть тот или иной исход остальным (в этом случае говорят, что все элементарные исходы равновероятны).
	
	\item[3)]
	$A \subseteq \Omega$, $|A| = N_A$
\end{enumerate}

\underline{Определение}: Вероятностью осуществления события A называется число 
$P(A) = \frac{N_A}{N}$

\underline{Пример}: 2 раза бросают игральную кость. \\
A = \{сумма выпавших очков $\geqslant$ 11\} \\
$P(A) = ?$ \\

\underline{Решение}: \\
Исход: $(x_1, x_2)$, где $x_i$-количество очков, выпавшем при i-ом броске. \\
$\Omega = \{1,1), (2,2), \ldots , (6,6)\}$ 
$|\Omega| = 36 = N$

\begin{enumerate}
	\item[б)]
	$A = \{(5,6), (6,5), (6,6)\}$ \\
	$n_a = |A| = 3$ \\
	
	\item[в)]
	$P(A) = \frac{N_A}{N} = \frac{3}{36} = \frac{1}{12}$
\end{enumerate}

Свойства вероятности (в соответствии с классическими определениями)
\begin{enumerate}
	\item[$1^o$]
	$P(A) \geqslant 0$
	
	\item[$2^o$]
	$P(\Omega) = 1$
	
	\item[$3^o$]
	Если $AB = \emptyset$, то $P(A+B) = P(A) + P(B)$
\end{enumerate}

\underline{Доказательство}
\begin{enumerate}
	\item[$1^o$]
	$P(A) = \frac{N_A}{N} \geqslant 0$
	
	\item[$2^o$]
	$P(\Omega) = \frac{N_{\Omega}}{N} = \frac{N}{N} = 1$
	
	\item[$3^o$]
	\begin{enumerate}
		\item[а)]
		$|A+B| = |A| + |B| - |AB|$ \\
		(Формула включений и исключений)
		По условию $|AB| = 0 \Rightarrow N_{A+B} = N_A + N_B$
		
		\item[б)]
		$P(A+B) = \frac{N_{A+B}}{N} = \frac{N_A + N_B}{N} = \frac{N_A}{N} + \frac{N_B}{N} = P(A) + P(B)$
	\end{enumerate}
\end{enumerate}

\underline{Замечание}: Недостатком классического определения вероятности.
\begin{enumerate}
	\item[1)] 
	Неприменимо в случае, когда $|\Omega| = \infty$
	
	\item[2)]
	Неприменимо, если некоторые исходы являются "более возможными", чем другие.
\end{enumerate}





































