% Формула Байеса

\underline{Th} о формуле Байеса.
Пусть
\begin{enumerate}
	\item[1)]
	Выполнено условие th о формуле полной вероятности
	
	\item[2)] 
	$P(A) > 0$
\end{enumerate}
Тогда \\
\fbox{$P(H_i|A) = \cfrac{P(A|H_i) P(H_i)}{P(A|H_1) P(H_1) + \ldots + P(A|H_n) P(H_n)}, \ \ i = \overline{1,n}$} - формула Байеса.

\underline{Доказательство}: \\
$P(H_i|A) = (P(A) > 0 \Rightarrow \text{условные вероятности определена}) = \cfrac{P(AH_i)}{P(A)} = \cfrac{P(A|H_i) P(H_i)}{P(A|H_1) P(H1) + \ldots + P(A|H_n) P(H_n)}$

\underline{Пример}: Рассмотрим пример о покупке телевизора. \\
Пусть известно, что куплен брак. Телевизор какое фирмой он вероятнее всего произведен? \\
$P(H_1|A) = \cfrac{P(A|H_1) P(H_1)}{P(A)} = \cfrac{0.07 \cdot 0.3}{P(A)} = \cfrac{0.021}{P(A)}$ \\
$P(H_2|A) = \cfrac{P(A|H_2) P(H_2)}{P(A)} = \cfrac{0.05 \cdot 0.5}{P(A)} = \cfrac{0.025}{P(A)}$ \\
$P(H_3|A) = \cfrac{P(A|H_3) P(H_3)}{P(A)} = \cfrac{0.2 \cdot 0.1}{P(A)} = \cfrac{0.02}{P(A)}$ \\
Ответ: вероятнее всего этот телевизор произведен 2-ой фирмой. \\

\underline{Замечание}:
\begin{enumerate}
	\item[1)]
	События $H_1, \ldots, H_n$ образующие полную группу, часто называют гипотезами.
	
	\item[2)]
	Вероятности $P(H_i) \ \ i = \overline{1,n}$ называются априорными, так как они известно ...
\end{enumerate}
Вероятности $P(H_i|A), \ \ i = \overline{1,n}$ которое становятся известными после ...


















