% Формула умножения вероятностей.

\underline{Th 1}: формула умножения вероятностей для 2-х событий.
Пусть
\begin{enumerate}
	\item[1)] 
	$A_1, A_2$ - события, связанные с некоторым случайным экспериментом.
	
	\item[2)]
	$P(A_1) > 0$
	Тогда
	\fbox{$P(A_1 A_2) = P(A_1) P(A_2|A_1)$} - формула умножения вероятностей для 2-х событий.
\end{enumerate}

\underline{Доказательство}.
\begin{enumerate}
	\item[1)]
	Так как $P(A_1) \neq 0$, то определена условная вероятность \\
	$P(A_2|A_1) = \frac{P(A_1 A_2)}{P(A_1)} \Rightarrow P(A_1 A_2) = P(A_1) P(A_2|A_1)$
\end{enumerate}

\underline{Th 2}: формула умножения вероятностей для n событий.
Пусть
\begin{enumerate}
	\item[1)]
	$A_1, \ldots , A_n$ - события, связанные с некоторым случайным экспериментом.
	
	\item[2)]
	$P(A_1 \cdot \ldots \cdot A_{n-1}) > 0$ \\
	Тогда \\
	\fbox{$P(A_1 \cdot A_2 \cdot \ldots \cdot A_n) = P(A_1) \cdot P(A_2|A_1) \cdot P(A_3|A_1 A_2) \cdot \ldots \cdot P(A_n|A_1 \cdot \ldots \cdot A_{n-1})$} - формула умножения вероятностей для n событий.
\end{enumerate}

\underline{Доказательство}.
\begin{enumerate}
	\item[1)]
	$A_1 \cdot \ldots \cdot A_{n-1} \subseteq A_1 \cdot \ldots \cdot A_k$, если $k \leqslant n-1$ \\
	$\Rightarrow P(A_1 \cdot \ldots \cdot A_k) \supseteq P(A_1 \cdot \ldots \cdot A_{n-1}) > 0$, $k = \overline{1, n-1}$ \\
	То есть все входящие в правую часть формулы умножения условной вероятности определены.
	
	\item[2)]
	$P(\underbrace{A_1 \cdot \ldots \cdot A_{n-1}}_{A} \cdot \underbrace{A_n}_{B}) =$ (из th умножения для 2-х событий $P(AB) = P(A) P(B|A)$) $ = P(\underbrace{A_1 \cdot \ldots}_{A}  \cdot \underbrace{A_{n-1}}_{B}) P(A_n | A_1 \cdot \ldots \cdot A_{n-1}) = P(A_1 \cdot \ldots \cdot A_{n-2}) \cdot P(A_{n-1}|A_1 \cdot \ldots \cdot A_{n-2}) \cdot P(A_n | A_1 \cdot \ldots \cdot A_n) = P(A_1) \cdot P(A_2 |A_1) \cdot P(A_3 |A_1 A_2) \cdot \ldots \cdot P(A_n |A_1 \cdot \ldots \cdot A_{n-1})$
\end{enumerate}

\underline{Пример}: На 7 карточках написаны буквы, составляющие слово "шоколад". Карточки перемешивают и случайным образом извлекают последовательно 3 карточки (без возвращения). \\
A = \{в порядке извлечения эти карточки образуют слово "код"\}.

\underline{Решение}:
\begin{enumerate}
	\item[1)] 
	Обозначение:
	$A_1$ = \{при 1-ом извлечении появилось "к"\} \\
	$A_2$ = \{при 2-ом извлечении появилось "о"\} \\
	$A_3$ = \{при 3-ем извлечении появилось "д"\} \\
	Тогда \\
	$A = A_1 A_2 A_3$
	
	\item[2)]
	$\displaystyle P(A) = P(A_1 A_2 A_3) = (\text{th умножения}) = \underbrace{P(A_1)}_{\frac{1}{7}} \cdot \underbrace{P(A_2 |A_1)}_{\frac{2}{6} = \frac{1}{3}} \cdot \underbrace{P(A_3 |A_1 A_2)}_{\frac{1}{5}} = \frac{1}{7} \cdot \frac{1}{3} \cdot \frac{1}{5} = \frac{1}{105}$
\end{enumerate}



























