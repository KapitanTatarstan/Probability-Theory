% Геометрическое определение вероятности.

Геометрическое определение вероятности является обобщением классического определения на случай бесконечное $\Omega$, когда $\Omega \subseteq \mathbb{R}^n$

Пусть
\begin{enumerate}
	\item[1)]
	$\Omega \subseteq \mathbb{R}^n$
	
	\item[2)]
	$\mu(\Omega) < \infty$ \\
	где $\mu$ - мера множества \\
	$\begin{cases}
		n = 1 - \mu - \text{длина} \\
		n = 2 - \mu - \text{площадь} \\
		... \\
	\end{cases}$
	
	\item[3)]
	Возможность принадлежности исхода эксперимента некоторому событию прямо-пропорциональна мере этого события и не зависит от его (события) формы и расположения внутри $\Omega$.
\end{enumerate}

\underline{Определение}: Вероятностью осуществления события A называется число $P\{A\} = \frac{\mu(A)}{\mu{\Omega}}$.

\underline{Пример}: "Задача о встрече". \\
Два человека договорились встретиться в условленном месте в промежутке от 12 до 13 часов. При этом если один из них пришел раньше другого, то он ждет 15 минут, после чего уходит. Какова вероятность того, что они встретятся, если появление каждого из них равновероятно в любой момент между 12 и 13 часами?

\underline{Решение}:
\begin{enumerate}
	\item[1)]
	Исход: \\
	$(x_1, x_2)$ где $x_i \in [0,1]$, $i = 1,2$ \\
	$x_i$ - время появления i-ого человека после 12. \\
	$\Omega = \left\{ (x_1, x_2): x_1 = [0;1] \right\} = [0;1]x[0;1]$
	% рисунок
	
	\item[2)] 
	A = \{эти 2 человека встретились\} \\
	$A = \left\{ (x_1, x_2): |x_1 - x_2| \leqslant \frac{1}{4} \right\}$
	
	\item[3)] 
	В соответствии с геометрическим определением
	$P(A) = \frac{\mu(A)}{\mu(\Omega)} = \frac{\mu(\Omega) - 2\mu(\Delta)}{\mu(\Omega)} = 1 - 2 \cdot \frac{1}{2} \left(\frac{3}{4}\right)^2 = 1 - \frac{9}{16} = \frac{7}{16}$
\end{enumerate}

\underline{Замечание}:
\begin{enumerate}
	\item[1)]
	Очевидно, что из геометрического определения следует те же свойства вероятности, что и из классического определения.
	
	\item[2)]
	недостатком геометрического определения является то, что некоторая область внутри $\Omega$ могут быть более предпочтительные чем другие области той же меры. Например если в разобранном примере появление каждого из этих двух человек было более вероятным в середине часа, то геометрическое определение дало бы неудовлетворительный результат.
\end{enumerate}



























