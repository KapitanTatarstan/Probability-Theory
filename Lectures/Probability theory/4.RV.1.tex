% Функция распределения случайного вектора.
Пусть 
\begin{enumerate}
\item[1)] $(\Omega, \beta, P)$ - вероятностное пространство

\item[2)] $X_1, \ldots, X_n$ - случайные величины, заданные на этом пространстве
\end{enumerate}


\underline{Определение}: Случайным вектором размерности n называется кортеж $\vec{X} =  (X_1, X_2, \ldots, X_n)$.

\underline{Замечание}:
\begin{enumerate}
\item[1)] иногда n-мерный случайные вектор называют n-мерной случайной величиной.

\item[2)] компоненты случайного вектора $(X_1, \ldots, X_n)$ называется его коорди ...
\end{enumerate}


\underline{Пример}:
\begin{enumerate}
\item[1)] Производят стрельбу по плоской мишени (пулей). \\
$(x_1, x_2)$ - координаты точки попадания пули образуют двумерный случайный вектор.
\end{enumerate}

\underline{Определение}: Функция распределения случайного вектора $(X_1, \ldots, X_n)$ называется отображением \\
$F: \mathbb{R}^n \to R$, \\
определенное следующим правилом: \\
$F(x_1, \ldots, x_n) = P\{X_1 < x_1, \ldots , X_n < x_n\}$ \\

\underline{Замечание}:
\begin{enumerate}
\item[1)] в определении под знаком вероятности написано произведение событий: \\
$\{X_1 < x_1\} \cdot \ldots \cdot \{X_n < x_n\}$

\item[2)] в случае n=2, если интерпретировать вектор $(x_1, x_2)$, как эксперимент в котором на плотность бросают точку, значение $F(x_1, x_2)$ равно вероятности того, что случайным образом брошена на плоскости точки упадет левее и ниже точки $(x_1, x_2)$
\end{enumerate}


\underline{Свойства функции распределения (n=2)}
\begin{enumerate}
\item[$1^o$] $0 \leqslant F(x_1, x_2) \leqslant 1$

\item[$2^o$] 
	\begin{enumerate}
	\item[а)] при фиксированном $x_2$ функция $F(x_1, x_2)$ как функция одной переменной $x_1$ является неубывающей.
	
	\item[б)] при фиксированном $x_1$ функция $F(x_1, x_2)$ ак функция одной переменной $x_2$ является неубывающей.
	\end{enumerate}
	
\item[$3^o$] $\lim\limits_{\begin{cases} x_1 \to -\infty \\ x_2 = const \\ \end{cases}} F(x_1, x_2) = 0$\\
$\lim\limits_{\begin{cases} x_1 \to const \\ x_2 \to -\infty \\ \end{cases}} F(x_1, x_2) = 0$ \\

\item[$4^o$] $\lim\limits_{\begin{cases} x_1 \to +\infty \\ x_2 \to -\infty \\ \end{cases}} F(x_1, x_2) = 1$ \\

\item[$5^o$] $\lim\limits_{\begin{cases} x_1 \to +\infty \\ x_2 \to const \\ \end{cases}} F(x_1, x_2) = F_{X_2}(x_2)$ \\
$\lim\limits_{\begin{cases} x_1 \to const \\ x_2 \to +\infty \\ \end{cases}} F(x_1, x_2) = F_{X_1}(x_1)$ \\
где $F_{X_1}, F_{X_2}$ - функции распределения случайных величин $X_1$ и $X_2$ соответственно. 

\underline{Замечание}: Рассмотрим случайный вектор $(X_1, X_2)$ можно временно "забыть"\ о случайной величине $X_2$ и наблюдать только за $X_1$. \\
$F_{X_1}$ - функция распределения случайной величины $X_1$.

\item[$6^o$] $P\{a_1 \leqslant X < b_1, \ a_2 \leqslant X < b_2\} = F(b_1, b_2) - F(a_1, b_2) - F(a_2, b_1) + F(a_1, a_2)$

\item[$7^o$] 
	\begin{enumerate}
	\item[а)] При фиксированном $x_2$ функция $F(x_1, x_2)$ как функция одной переменной $x_1$ является непрерывной слева в каждой точке.
	
	\item[б)] При фиксированном $x_1$ функция $F(x_1, x_2)$ как функция одной переменной $x_2$ является непрерывной слева в каждой точке.
	\end{enumerate}
\end{enumerate}


\underline{Доказательство}:
\begin{enumerate}
\item[$1^o$] $F(x_1, x_2) =\footnote{определение функции распределения} = P\{\ldots\{ \Rightarrow 0 \leqslant F(x_1, x_2) \leqslant 1$ \\

\item[$2^o$] доказывается аналогично одномерному случаю. \\

\item[$3^o$] Докажем, что $\lim\limits_{\begin{cases} x_1 \to -\infty \\ x_2 = const \\ \end{cases}} F(x_1, x_2) = 0$, \\ 
второе равенство доказывается аналогично. \\
$F(x_1, x_2) = P\left\{ \{X_1 < x_1\} \cdot \{X_2 < x_2\} \right\}$ \\
Если $x_1 \to -\infty$, то событие $\{X_1 < x_1\}$ становится невозможным; произведение невозможного события на событие $\{X_2 < x_2\}$ является невозможным $\Rightarrow \lim\limits_{\begin{cases} x_1 \to \infty \\ x_2 \to const \\ \end{cases}} F(x_1, x_2) = \left( P\{\text{невозможное}\} \cdot \{X_2 < x_2\} \right) = 0$

\item[$4^o$] $F(x_1, x_2):  P\left\{ \{X_1 < x_1\} \cdot \{X_2 < x_2\} \right\}$ \\
При $x_1 \to +\infty$ событие $\{X_1 < x_1\}$ становится достоверным; при $x_2 \to +\infty$ событие $\{X_2 < x_2\}$ также становится достоверным, а поскольку произведением достоверных событий является достоверным событием, то \\
$\lim\limits_{\begin{cases} x_1 \to +\infty \\ x_2 \to +\infty \\ \end{cases}} F(x_1, x_2) = \left( P\left\{ \{\text{достоверное}\} \cdot \{\text{достоверное}\}\right\} \right) = 1$ \\

\item[$5^o$] Докажем, что \\
$\lim\limits_{\begin{cases} x_1 \to +\infty \\ x_2 = const \\ \end{cases}} F(x_1, x_2) = F_{X_2}(x_2)$ \\
$F(x_1, x_2) = P\left\{ \{X_1 < x_1\} \cdot \{X_2 < x_2\} \right\}$ \\
При $x_1 \to +\infty$ событие $\{X_1 < x_1\}$ становится достоверным , произведение достоверного события на событие $\{X_2 < x_2\}$ равно последнему, поэтому \\
$\lim\limits_{\begin{cases} x_1 \to +\infty \\ x_2 = const \\ \end{cases}} F(x_1, x_2) = \left( P\left\{ \{\text{достоверное}\} \cdot \{X_2 < x_2\} \right\} \right) = P\{X_2 < x_2\} \ldots$ \\

\item[$6^o$] 
	\begin{enumerate}
	\item[1)] Найдем вероятность попадания случайного вектора $(X_1, X_2)$ в полуполосу: \\
	$P\{X_1 < x_1, \ a_2 \leqslant X_2 < b_2\}$ 
	%Рисунок
	
	Заметим, что \\
	$\underbrace{\{X_1 < x_1, \ X_2 < b_2\}}_{1} = \underbrace{\{X_1 < x_1, \ a_2 \leqslant X_2 < b_2\}}_{2} + \underbrace{\{X_1 < x_1, \ X_2 < a_2\}}_{3}$ \\
	% Рисунок 1]
	% Рисунок 2]
	% Рисунок 3]
	От обеих частей возьмем вероятность, и используя теорему сложения получим \\
	$P\{\underbrace{X_1 < x_1, \ X_2 < b_2}_{= F(x_1, b_2)}\} = P\{X_1 < x_1, \ a_2 \leqslant X_2 < b_2\} + P\{\underbrace{X_1 < x_1, \ X_2 < a}_{F(x_1, a_2}\}$ \\
	Таким образом \\
	$P\{X_1 < x_1, \ a_2 \leqslant X_2 < b_2\} = F(x_1, b_2) - F(x_1, a_2)$ \\
	
	\item[2)] Найдем $P\{a_1 \leqslant X_1 < b_2, \ a_2 \leqslant X_2 < b_2\}$ \\
	%Рисунок
	$\{X_1 < b_1, \ a_2 \leqslant X_2 < b_2\} = \{X_1 < a_1, \ a_2 \leqslant X_2 < b_2\} + \{a_1 \leqslant X_1 < b_2, \ a_2 \leqslant X_2 < b_2\}$ \\ %Добавить сноски под формулы
	Берем P от обеих частей и используем th сложения получаем \\
	$ \underbrace{P\{X_1 < b_1, \ a_2 \leqslant X_2 < b_2\}}_{= F(b_1, b_2) - F(b_1, a_2)} = \underbrace{P\{X_1 < a_1, \ a_2 \leqslant X_2 < b_2\}}_{= F(a_1, b_2) - F(a_1, a_2)} + P\{a_1 \leqslant X_1 < b_1, \ a_2 \leqslant X_2 < b_2\}$ \\
	$\Rightarrow P\{a_1 \leqslant X_1 < b_1, \ a_2 \leqslant X_2 < b_2\} = F(b_1, b_2) - F(b_1, a_1) - F(a_1, b_2) + F(a_1, a_2)$ \\
	\end{enumerate}
	
\item[$7^o$] Доказательство аналогично одномерному случаю.	
\end{enumerate}

\underline{Замечание}:
\begin{enumerate}
\item[1)] Рассмотрим свойство $5^o$. В нем использовались $F(x_1, x_2)$ - функции распределения вектора $(X_1, X_2)$. \\
$F_{X_1}(x_1), \ F_{X_2}(x_2)$ - функции распределения случайных величин. \\
В теории вероятностей используется следующая терминология: \\
$F(x_1, x_2)$ называется также совместной функцией распределения случайных величин $x_1$ и $x_2$, \\
$F_{X_1}(x_1), \ F_{X_2}(x_2)$ называется маргинальными (частными) функциями распределения случайных величин $x_1$ и $x_2$.

\item[2)] Если известна $F(x_1, x_2)$, то с использованием свойства $5^o$ можно найти $F_{X_1}, F_{X_2}$, то есть \\
%РИСУНОК
Вопрос: Можно ли зная $F_{X_1}$ и $F_{X_2}$, найти $F(x_1, x_2)$? \\
Вообще говоря, нет (так ка неизвестна связь между $X_1$ и $X_2$).
\end{enumerate}























