% Формула полной вероятности.

Пусть $(\Omega, \beta, P)$ - вероятностное пространство.

\underline{Определение}: Будем говорить, что события $H_1, \ldots , H_n$ образуют полную группу, если \\
\begin{enumerate}
	\item[1)]
	$H_1 + \ldots + H_n = \Omega$ \\
	
	\item[2)]
	$H_i H_j = \emptyset, \ \ i \neq j$ \\
	
	\item[3)]
	$P(H_i) > 0, \ \ i = \overline{1,n}$ \\
\end{enumerate}

% Рисунок

\underline{Th} о формуле полной вероятности.
Пусть
\begin{enumerate}
	\item[1)] 
	$H_1, \ldots, H_n$ - полная группа событий.
	
	\item[2)]
	$A \in \beta$ - некоторое событие.
\end{enumerate}
Тогда \\
\fbox{$P(A) = P(A|H_1) P(H_1) + \ldots + P(A|H_n) P(H_n)$}

\underline{Доказательство}:
%Рисунок
$P(A) = P(A \Omega) = P(A (H_1 + \ldots + H_n)) = P(AH_1 + \ldots + AH_n) = $
\begin{enumerate}
	\item[а)] 
	$H_i H_j = \emptyset$ 
	
	\item[б)]
	$(AH_i) \subseteq H_j$ \\
	$(AH_j) \subseteq H_j$ \\
	
	\item[в)]
	а) б) $\Rightarrow (AH_i)(AH_j) = \emptyset$
\end{enumerate}
 =\footnote{\text{аксиома} $3^o$ \text{вероятности}} $P(AH_1) + \ldots + P(AH_2) = (\text{th умножения}) = P(A|H_1) P(H_1) + \ldots + P(A|H_n) P(H_n)$
 
\underline{Пример}: В магазине поступили телевизоры 3-х фирм из которых: \\
30\% произвела 1-я фирма \\
50\% произвела 2-я фирма \\
20\% произвела 3-я фирма \\
Известно, что среди телевизоров \\
1-ой фирмы 7\% брака \\
2-ой фирмы 5\% брака \\
3-ей фирмы 10\% брака \\.
Найти вероятность того, что случайно выбранный телевизор окажется бракованным? \\

\underline{Решение}: \\
\begin{enumerate}
	\item[1)]
	$H_1$ = \{выбранный телевизор произведен 1-ой фирмой\} \\
	$H_2$ = \{выбранный телевизор произведен 2-ой фирмой\} \\
	$H_3$ = \{выбранный телевизор произведен 3-ей фирмой\} \\
	Обозначение: \\
	A = \{выбранный телевизор бракованный\}
	
	\item[2)] 
	Формула полной вероятности \\
	$P(A) = \underbrace{P(A|H_1)}_{0.07} \underbrace{P(H_1)}_{0.3} + \underbrace{P(A|H_2)}_{0.05} \underbrace{P(H_2)}_{0.5} + \underbrace{P(A|H_3)}_{0.1} \underbrace{P(H_3)}_{0.2} = 0.066$
\end{enumerate}
















