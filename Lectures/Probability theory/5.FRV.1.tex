% 1. Функции от одномерныз случайныз величин

Пусть 
\begin{enumerate}
\item[1)] X - случайная величина
\item[2)] $\varphi: \ \mathbb{R} \to \mathbb{R}$ 
\end{enumerate}
Рассмотрим $\varphi(x) = Y$ - тоже случайная величина.


\underline{Пример:} При изготовлении вала на токарном станке его диаметр X является случайной величиной, тогда $Y = \cfrac{\pi \cdot X^2}{4}$ - площадь поперечного сечения этого вала - тоже случайная величина.\\
В этом примере $\varphi(X) = \cfrac{\pi \cdot X^2}{4}$. \\


\underline{Основной вопрос:} Как, зная закон распределения случайной величины X и функцию $\varphi$, найти закон распределения случайной величины $Y = \varphi(X)$? \\

I. Пусть X - диаметр случайной величины, имеющее ряд распределения \\
%Вставить таблицу 
$\left( \sum\limits_{i = 1}^{n} p_i = 1 \right)$ \\
Пусть $Y = \varphi(x)$ \\
Случайная величина Y также будет дискретной, так как функция не может принимать значений больше, чем её аргументы. \\
%Вставить пояснение
Тогда случайная величина Y принимает значение из множества $\varphi(x_i), \ i = \overline{1,n}$ \\
%Вставить таблицу
Если в верхней строке некоторые значения совпадут (то есть $\varphi(x_i) = \varphi(x_i)$ при $i \neq j$), то соответствующие столбцы следует объединить, приписав общему значению суммарную вероятность. \\


\underline{Пример:} X имеет ряд распределения: \\
%Вставить таблицу
%\include{Table1.pdf} 
Найти ряд распределения случайной величины $Y = X^2 + 1$. В этом примере $\varphi(x) = x^2 + 1$ \\
%Вставить таблицы





























