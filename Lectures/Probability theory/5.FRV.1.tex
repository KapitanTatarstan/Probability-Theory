% 1. Функции от одномерныз случайныз величин

Пусть 
\begin{enumerate}
\item[1)] X - случайная величина
\item[2)] $\varphi: \ \mathbb{R} \to \mathbb{R}$ 
\end{enumerate}
Рассмотрим $\varphi(x) = Y$ - тоже случайная величина.


\underline{Пример:} При изготовлении вала на токарном станке его диаметр X является случайной величиной, тогда $Y = \cfrac{\pi \cdot X^2}{4}$ - площадь поперечного сечения этого вала - тоже случайная величина.\\
В этом примере $\varphi(X) = \cfrac{\pi \cdot X^2}{4}$. \\


\underline{Основной вопрос:} Как, зная закон распределения случайной величины X и функцию $\varphi$, найти закон распределения случайной величины $Y = \varphi(X)$? \\

I. Пусть X - диаметр случайной величины, имеющее ряд распределения \\
%Вставить таблицу 
$\left( \sum\limits_{i = 1}^{n} p_i = 1 \right)$ \\
Пусть $Y = \varphi(x)$ \\
Случайная величина Y также будет дискретной, так как функция не может принимать значений больше, чем её аргументы. \\
%Вставить пояснение
Тогда случайная величина Y принимает значение из множества $\varphi(x_i), \ i = \overline{1,n}$ \\
%Вставить таблицу
Если в верхней строке некоторые значения совпадут (то есть $\varphi(x_i) = \varphi(x_i)$ при $i \neq j$), то соответствующие столбцы следует объединить, приписав общему значению суммарную вероятность. \\


\underline{Пример:} X имеет ряд распределения: \\
%Вставить таблицу
%\include{Table1.pdf} 
Найти ряд распределения случайной величины $Y = X^2 + 1$. В этом примере $\varphi(x) = x^2 + 1$ \\
%Вставить таблицы


\underline{Th} Пусть
\begin{enumerate}
\item[1)] X - непрерывная случайная величина
\item[2)] $\varphi: \ \mathbb{R} \to \mathbb{R}$ - монотонная функция \\
($\Rightarrow \exists \varphi^{-1} = \psi$ - обратная к $\varphi$ функция) 
\item[3)] $\varphi$ - непрерывная и непрерывно дифференцируема 
\item[4)] $\varphi = \varphi(X)$
\end{enumerate}
Тогда случайная величина Y также является непрерывной, причем \\
$f_Y(y) = f_X \left(\psi(y)\right) \left|\psi'(y)\right|$, \\
где $f_X, f_Y$ - функции плотности случайных величин X и Y соответственно. \\


\underline{Доказательство} 
\begin{enumerate}
\item[1)] $F_Y(y) = P\{Y < y\} = \left(Y = \varphi(X)\right) = P\{\varphi(X) < y\}$ \\
$F_Y$ - функция распределения случайной величины Y. \\
Рассмотрим 2 случая:
	\begin{enumerate}
	\item[А.] $\varphi$ - монотонная возрастающая функция $\Rightarrow \varphi(X) < y \Leftrightarrow X < \varphi^{-1}(y) = \psi(y)$.
	\item[Б.] $\varphi$ - монотонная убывающая функция $\Rightarrow \varphi(X) < y \Leftrightarrow X > \psi(y)$.
	\end{enumerate}
Тогда
	\begin{enumerate}
	\item[А.] $F_Y(y) = P\left(X < \psi(y)\right) = F_X \left(\psi(y)\right)$.
	\item[Б.] $F_Y(y) = P\left(X > \psi(y)\right) = 1 - P\{X \leqslant \psi(y)\}$ = (X - непрерывная случайная величина) = $1 - P\{X < \psi(y)\} = 1 - F_X\left(\psi(y)\right)$.
	\end{enumerate}

\item[2)] Так как $f_Y(y) = \cfrac{d}{dy} F_Y(y)$, то 
	\begin{enumerate}
	\item[А.] $f_Y(y) = \cfrac{d}{dy} \left[F_X \left(\psi(y)\right) \right]$ = (th о произведении случайных функций) = $\underbrace{F_X'}_{f_X} \left(\psi(y)\right) \cdot \psi'(y) = f_X \left(\psi(y)\right) \cdot \varphi(y)$.
	\item[Б.] $f_Y(y) = \cfrac{d}{dy} \left[ 1 - F_X \left(\psi(y)\right) \right] = \ldots = f_X \left(\psi(y)\right) \cdot \left(- \psi'(y) \right)$
	\end{enumerate}
Оба случая описываются общей формулой
$f_Y(y) = f_X \left(\psi(y)\right) \left|\psi'(y)\right|$
\end{enumerate}


\underline{Пример:} Пусть
\begin{enumerate}
\item[1)] X - непрерывная случайная величина
\item[2)] F(x) - функция распределения случайной величины. X непрерывна.
\end{enumerate}
% График
Найдем закон распределения случайной величины $Y = F(x)$ (то есть $\varphi = F$) \\
\underline{Решение:}
Очевидно, что $Y \in [0;1]$ \\
Это означает, что
\begin{enumerate}
	\item[а)] $F_Y(y) = 0$, если $y \leqslant 0$
	\item[б)] $F_Y(y) = 1$, если $y > 0$
	\item[в)] $y \in (0;1] \Rightarrow$ (см. доказательство предыдущей th для случая монотонно возрастающей $\varphi$) $\Rightarrow F_Y(y) = \underbrace{F}_{F_X} \left(\underbrace{F^{-1}}_{\varphi^{-1}} (y) \right) = y$
\end{enumerate}
Таким образом \\
$F_Y(y) = 
\begin{cases}
	0, \ y \leqslant 0 \\
	y, 0 < y \leqslant 1 \\
	1, \ y > 1 \\
\end{cases}$ \\
% График
Тогда \\
$f_Y(y) = \cfrac{d}{dy} F_Y(y) = 
\begin{cases}
	1, \ y \in [0;1] \\
	0, \text{иначе} \\
\end{cases}$ \\
% График
Таким образом \\
$Y \sim R[0;1]$ - равномерное распределение на [0;1] случайной величины. \\


\underline{Замечание:} Из предыдущего примера следует, что если $Y \sim R[0;1]$, то случайная величина $X = F^{-1}(y)$ будет иметь функцию F своей функции распределения. \\
Этот факт широко используется при моделировании случайных величин: достаточно иметь генератор случайных чисел для $R[0;1]$, тогда для генерировании значений случайной величины X с непрерывной функцией распределения F(x) достаточно подвинуть выборку из $R[0;1)$, ... \\


\underline{Th} Случай немонотонной функции $\varphi$ \\
Пусть
\begin{enumerate}
	\item[1)] X - непрерывная случайная величина.
	\item[2)] $f_X$ - функция плотности случайной величины X.
	\item[3)] $\varphi: \ \mathbb{R} \to \mathbb{R}$ имеет n интервалов монотонности (то есть $\varphi$ является кусочно-монотонной).
	\item[4)] $\varphi$
	% Рисунок
	\item[5)] для данного $y \in \mathbb{R}$ \\
	$x_1 = \psi_1(y), \ldots, x_n = \psi_n(y)$ - все распределения уравнения $\varphi(x) = y$. \\
	При этом $\psi_1(t), \ldots, \psi_k(t)$ - функции обратные $\varphi$ на интервалах монотонности, которые ...
	\item[6)] $Y = \varphi(X)$ \\
	Тогда \\
	$f_Y(y) = \sum\limits_{j = 1}^{k} f_X \left(\psi_j(y)\right) \cdot \left|\psi'_j(y)\right|$
\end{enumerate}


\underline{Доказательство} (без доказательств).
























