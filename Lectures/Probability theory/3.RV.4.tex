% Непрерывные случчайные величины.

\underline{Определение}: Случайная величина X называется непрерывной, если $\exists$ функция $f" \mathbb{R} \to \mathbb{R}$ такая, что $\forall x \in \mathbb{R}$ $F(x) = \int\limits_{-\infty}^{x} f(t) dt$, где F - функция распределения случайной величины X. \\
При этом функция f называется функцией плотности распределения вероятностей случайной величины X.

\underline{Замечание}: 
\begin{enumerate}
	\item[1)]
	%Вставить рисунок

	\item[2)]
	Для большинства представляющих практический интерес непр. случайная величина функции плотности является кусочно-непрерывным. Это означает, что F - непрерывна. Именно по этой причине такие случайные величины называют ...
	
	\item[3)]
	Если f непрерывна в точке $x_0$ , то $f(x_0) = F'(x) \bigg|_{x = x_0}$ \\
	(мы использует th о производной интеграла с переменным верхним пределом) \\
	
	\item[4)]
	из определения непрерывной случайной величины $\Rightarrow$ \\
	$f(x) \Rightarrow F(x)$ \\
	(если известна f, то можно найти F). \\
	из замечания 3) $\Rightarrow$ \\
	$F(x) \Rightarrow f(t)$ \\
	(если известна F, то известна f) \\
	Таким образом функция f плотности, как и функция F распределения, содержит всю информацию о законе распределения случайной величины. \\
	Поэтому закон распределения непрерывной случайной величины можно задавать как с использованием функции распределения, так и с использованием функции плотности. \\
\end{enumerate}

Свойства непрерывной случайной величины (свойства функции плотности).
\begin{enumerate}
	\item[$1^o$]
	$\forall x \in \mathbb{R} \ \ f(x) \geqslant 0$ \\
	
	\item[$2^o$]
	$P\{x_1 \leqslant X , x_2\} = \int\limits_{x_1}^{x_2} f(x)dx$, где f - функция плотности случайной величины X.
	
	\item[$3^o$]
	$\displaystyle \int\limits_{-\infty}^{+\infty} f(x)dx = 1$ (условие нормировки) %Проверить написание слова "нормировки" по смыслу к этой части
	
	\item[$4^o$]
	$P\{x \leqslant X < x_0 + \Delta x\} \approx f(x_0) \Delta x$, где f - функция плотности случайной величины X; $x_0$ - точка непрерывности функции f.
	
	\item[$5^o$]
	Если X - непрерывная случайная величина, то для любого наперед заданного $x_n$ 
\end{enumerate}

\underline{Доказательство}:
\begin{enumerate}
	\item[$1^o$]
	$f(x) = F'(x)$ \\
	Так как F - неубывающая функция, то $F(x)$ ... $f(x) \geqslant 0$.
	
	\item[$2^o$]
	$P\{x_1 \leqslant X < x_2\}$ = (свойство функции распределения) = $F(x_2) - F(x_1)$ = (F - первообразная, поэтому ...) = $\displaystyle \int\limits_{-\infty}^{+\infty} f(x) dx = 
	\lim\limits_{
		\begin{cases}
			x_1 \to -\infty \\ 
			x_2 \to +\infty 
		\end{cases}} 
	\int\limits_{x_1}^{x_2} f(x)dx$ = (формула Н-Л) = 
	$\lim\limits_{
		\begin{cases}
			x_2 \to +\infty \\
			x_1 \to -\infty \\
		\end{cases}}
	\left[F(x_2) - F(x_1)\right] = 1 - 0 = 1$
	
	\item[$5^o$]
	$P\{X = x_0\} = \lim\limits_{\Delta x \to 0} P\{x_0 < X < x_0 + \Delta x\} = \lim\limits_{\Delta x \to 0} \left[F(x_0 + \Delta x) - F(x_0)\right]$ = (так как мы считает F непрерыв. (см. замечание выше), то $\lim\limits F(x_0 + \Delta x) = F(x_0)$) = $F(x_0) - F(x_0) = 0$
\end{enumerate}


\underline{Замечание}:
\begin{enumerate}
	\item[1)]
	Пусть X - непрерывная случайная величина \\
	$P\{x_1 \leqslant X \leqslant x_2\} = \left( \{x_1 \leqslant X \leqslant x_2\} = \{x_1 \leqslant X < x_2\} + \{X = x_2\} \right) = P\{\underbrace{\{x_1 \leqslant X \leqslant x_2\} + \{X = x_2\}}_{\text{несовместны}}\} = P\{x_1 \leqslant X < x_2\} + \underbrace{P\{X = x_2\}}_{0 \text{т.к. X - непрерывна}} = P\{x_1 \leqslant X < x_2\}$
	
	\item[2)]
	Аналогично можно доказать, что для непрерывной случайно величины $P\{x_1 < X < \leqslant x_2\} = P\{x_1 < X < x_2\} + P\{x_1 \leqslant X < x_2\}$ \\
	Иногда с учетом этих результатов свойства $2^o$ записаны в виде %нет символа в latex
	$ = \int\limits_{x_1}^{x_2} f(x) dx$
\end{enumerate}



















