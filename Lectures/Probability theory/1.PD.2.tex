   % Операции над событиями.
События - множества (подмножество множества $\Omega$) $\Rightarrow$ $\cup, \cap, -, \backslash, \Delta$.

\underline{Определение}: Суммой событий $A, B \subseteq \Omega$ называют событие $A+B = A \bigcup B$.

\underline{Определение}: Произведение событий $A, B \subseteq \Omega$ называют событие $AB = A \bigcap B$.

% Рисунок
% Рисунок

\underline{Определение}: $A \backslash B$ называется разностью событий A и B.

\underline{Определение}: $\overline{A}$ называется событием, противоположным A.

Свойства операции над событиями: \\
(см. теоретико-множественное тождество (основные)).

\underline{Определение}: События $A, B \in \Omega$ называется несовместными, если $AB = \emptyset$. В противном случае события A и B называются совместными.

\underline{Определение}: События $A_1, ..., A_n$ называется попарно-несовместные, если $A_i, A_j = \emptyset, \ i \neq j$ 
- несовместными в совокупности, если $A_1 \cdot \ldots \cdot A_n = \emptyset$

\underline{Замечание}: попарно-несовместные $\Rightarrow$ несовместные в совокупности.