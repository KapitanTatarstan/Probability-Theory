% Основные законы распределения случайной величины

В этом пункте мы рассмотрим некоторые примеры случайных величин, законы распределения которых часто встречаются на практике.

I. Биномиальные случайные величины.
\underline{Определение}: Говорят, что случайная величина X распределена по биномиальному закону с параметрами $n \in \mathbb{N}$ и $p \in (0,1)$, если она принимает значения $0,1, \ldots, n$ с вероятностями \\
$P\left\{ X = \cfrac{1}{k} \right\} = C^k_n \cdot p^k \cdot q^{n-k}, \ \ k \in \{0, \ldots, n\}$, где $q = 1 - p$. \\
Обозначение: $X \sim B(n,p)$ \\%Добавить текст

\underline{Замечание:} 
\begin{enumerate}
	\item[1)] 
	Очевидно, X - дискретная случайная величина %Вставить таблицу
				
	\item[2)] 
	Случайная величина $X \sim B(n,p)$ - число успехов испытании по схеме Бернулли с вероятностью успеха "p".
\end{enumerate}


II. Пуассоновская случайная величина.
\underline{Определение}: Говорят, что случайная величина X распределена по закону Пуассона с параметром $\lambda > 0$ \\
$(\lambda \in (0, +\infty))$, если она принимает значения $0, 1, 2, \ldots$ с вероятностями \\
$P\left\{X = \cfrac{1}{\lambda}\right\} = \cfrac{\lambda^k}{k!} \exp^{-\lambda}, \ \ k = 0, 1, 2, \ldots$ \\
Обозначение: $X \sim \Pi (\lambda)$

\underline{Замечание}:
\begin{enumerate}
	\item[1)]
	Проверим условие нормировки \\
	$\displaystyle \sum\limits_{k = 0}^{\infty} P\left\{X = \cfrac{1}{\lambda} \right\} \sum\limits_{k = 0}^{\infty} \cfrac{\lambda^k}{k!} \exp^{-\lambda} = \exp^{-\lambda} \sum\limits_{k = 0}^{\infty} \cfrac{\lambda^k}{k!} = 1$
	
	\item[2)]
	Распределение Пуассона называется законом редких событий, так как оно проявляется там, где происходит большое число испытаний с малой вероятностью успеха.
\end{enumerate}

III. Геометрическое распределение.
\underline{Определение}: Говорят, что случайная величина X имеет геометрическое распределение с параметрами p, если X принимает целое геометрическое значения. \\
$P\{X = k\} = pq^k, \ \ k \in \{0, 1, 2, \ldots\}$, где $p \in(0,1), \ \ q = 1 - p$ \\
\underline{Замечание}:
\begin{enumerate}
	\item[1)] 
	Проверим условие нормировки \\
	$\displaystyle \sum\limits_{k = 0}^{\infty} P\{X = k\} = \sum\limits_{i = 0}^{\infty} pq^k = p \sum\limits_{k = 0}^{\infty} pq^k = p \sum\limits_{k = 0}^{\infty} q^k = \left((q^0 - q^1 + q^2 + \ldots)) = (\text{сумма бесконечной геометрической прогрессии} = \cfrac{1}{1-q}\right) = p \cdot \cfrac{1}{1-p} = 1$
	
	\item[2)]
	С содержательной точки зрения случайная величина X, имеющая геометрическое распределение с параметром p - количество экспериментов в схеме Бернулли, которое нужно произвести \underline{\underline{до}} 1-го появления "успеха" (то есть если первый успех произошел в (n+1)-м испытании, то X = n). \\
	В самом, деле, если в серии было n+1 испытаний ...
\end{enumerate}

























