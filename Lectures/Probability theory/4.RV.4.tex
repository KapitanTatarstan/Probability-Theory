% 4. Независимые случайные величины.

%Вставить "напоминание"
Пусть 
\begin{enumerate}
\item[1)] $(X,Y)$ - двумерный дискретный случайный вектор

\item[2)] $X \in \{x_1, \ldots, x_n\}$ \\
$Y \in \{y_1, \ldots, y_n\}$ \\
\end{enumerate}
Для такого вектора определение независимости случайных величин можно дать следующим образом: \\
случайные величины X и Y называется независимыми, если \\
$\forall i \in \{1, \ldots, n\}$ \\
$\forall j \in \{1, \ldots, n\}$ \\
$P\left\{ (X,Y = (x_i, y_j) \right\} = P\{X = x_i\} P\{Y = y_j\}$ \\
Посмотрим, что означает выполнение этого "промежуточного"\ определения для функции распределения вектора $(X,Y)$ \\
%Вставить график
$F(x,y) = P\{X < x, Y < y\} = 
\left\{ \begin{array}{lll}
	x_i = \max\limits_{i = \overline{1,n}} \{x_i: \ x_i < x\} \\
	y_j = \max\limits_{j = \overline{1,n}} \{y_j: \ y_j < y\} \\
\end{array} \right\} = 
P\left\{ \left\{ X \in \{x_1, \ldots, x_k\} \right\} \cdot \left\{ Y \in \{y_1, \ldots, y_l\} \right\} \right\} = 
P\left\{ (X,Y) = \{x_i, y_j\}, \ i = \overline{1,k}, \ j = \overline{1,l} \right\} = 
\sum\limits_{i = 1}^{k} \sum\limits_{i = 1}^{l} p\left\{ (X,Y) = (x_i, y_j) \right\}$ = 
(... "предварит."\ определение независимости случайных величин) = 
$\sum\limits_{i = 1}^{k} \sum\limits_{j = 1}^{l} \underbrace{P\{X = x_i\}}_{\text{независим. от } j} P\{Y = y_j\} = 
\sum\limits_{i = 1}^{k} P\{X = x_i\} \sum\limits_{j = 1}^{l} P\{Y = y_j\} = P\{X < x\} P\{Y = y\} = F_X (x) F_Y (x)$ \\
Таким образом для \underline{произвольного} случайного вектора $(X,Y)$ ... сформулировать (полноценное) определение: \\
\underline{Определение}: случайные величины X и Y называются независимы, если \\
$F(x,y) = F_X (x) F_Y (y)$, где F - функция распределения вектора (X,Y), $F_X, F_Y$ - частные функции распределения случайных величин X и Y. \\


\underline{Свойства независимых случайных величин} 
\begin{enumerate}
\item[$1^o$] Случайные величины X и Y независимы $\Leftrightarrow \forall x \ \forall y$ события \{X < x\} и \{Y < y\} независимы.

\item[$2^o$] Случайные величины X и Y независимы $\Leftrightarrow \forall \forall x_1, x_2, y_1, y_2$ события $\{x_1 \leqslant X < x_2\}$ и $\{y_1 \leqslant Y < y_2\}$ независимы.

\item[$3^o$] случайные величины X и Y независимы $\Leftrightarrow$ события $\{X \in M_1\}$ и $\{Y \in M_2\}$ независимы для любых $M_1$ и $M_2$ - промежутков или объединений промежутков в $\mathbb{R}$.

\item[$4^o$] Пусть
	\begin{enumerate}
	\item[1)] (X,Y) - дискретный случайный вектор
	\item[2)] $p_{ij} = P\left\{ (X,Y) = (x_i, y_j) \right\}, \ i = \overline{1,m}, \ j = \overline{1,n}$
	\item[3)] $p_{X_i} = P\{X = x_i\}, \ i = \overline{1,m}$ \\
	$p_{Y_j} = P\{Y = y_j\}, \ j = \overline{1,n}$ \\
	\end{enumerate}
Тогда X и Y -независимы $\Leftrightarrow p_{ij} \equiv p_{x_i} o_{y_j}, \ i = \overline{1,m}, \ j = \overline{1,n}$

\item[$5^o$] Пусть
	\begin{enumerate}
	\item[1)] (X,Y) - непрерывный случайный вектор
	\item[2)] f(x,y) - совместная плотность распределения X и Y
	\item[3)] $f_X, f_Y$ - маргинальная плотности
	\end{enumerate}
Тогда X, Y - независимые $\Leftrightarrow f(x,y) \equiv f_X (x) f_Y (y)$
\end{enumerate}


\underline{Доказательство}
\begin{enumerate}
\item[$1^o$] непосредственно следует из определения независимой случайной величины и определения функции распределения.

\item[$2^o$ а)] \fbox{$\Rightarrow$} необходимость \\
Пусть $F(x,y) = F_X (x) F_Y (y)$ \\
Тогда \\
$P\{x_1 \leqslant X < x_2, \ y_1 \leqslant Y < y_2\}$ = (по свойству двум. функции распределения) = $F(x_2, y_2) - F(x_1, y_2) - F(x_2, y_1) + F(x_1, y_1)$ = (X,Y - независимы) = $F_X(x_2) F_Y(y_2) - F_X(x_1) F_Y(y_2) - F_X(x_2) F_Y(y_1) + F_X(x_1) F_Y(y_1) = F_X(x_2) \left[ F_Y(y_2) - F_Y(y_1) \right] - F_X(x_1) \left[F_Y(y_2) - F_Y(y_1)\right] = \underbrace{\left[ F_X(x_2) - F_X(x_1)\right]}_{P\{x_1 \leqslant X < x_2\}} \underbrace{\left[ F_Y(y_2) - F_Y(y_1) \right]}_{P\{y_1 \leqslant Y < y_2\}} = P\{x_1 \leqslant X < x_2, \ y_1 \leqslant Y < y_2\} \Rightarrow$ события $\{x_1 \leqslant X < x_2\}$ и $\{y_1 \leqslant Y < y_2\}$ независимы.

\item[$2^o$ б)] \fbox{$\Leftarrow$} достаточность \\
Пусть $\underbrace{P\{x_1 \leqslant X < x_2, \ y_1 \leqslant Y < y_2\}}_{(*)} = P\{x_1 \leqslant X < x_2\} P\{y_1 \leqslant Y < y_2\}$ \\
Тогда \\
$F(x,y) = P\{X < x, Y < y\} = P\{\underbrace{-\infty}_{"x_1"\ } < X < \underbrace{x}_{"x_2"\ }, \underbrace{-\infty}_{"y_1"\ } < Y < \underbrace{y}_{"y_2"\ } \} = (*) = P\{X < x\} P\{Y < y\} = F_X(x) F_Y(y)$

\item[$3^o$] является обобщением свойств $1^o$ и $2^o$ и доказывается аналогично.

\item[$4^o$] а) \fbox{$\Rightarrow$} достаточность была доказана выше при рассуждениях между  "предвар."\ и "полноценным"\ определениям. \\
б) \fbox{$\Leftarrow$} необходимость доказать самостоятельно

\item[$5^o$ а)] \fbox{$\Rightarrow$} (необходимость) \\
Пусть X,Y - независимы, то есть \\
$F(x,y) = F_X(x) F_Y(y)$ \\
Тогда \\
$f(x,y) = \cfrac{\partial^2 F(x,y)}{\partial x \partial y} = \cfrac{\partial^2}{\partial x \partial y} \left[ F_X(x) F_Y(y)\right] = \left[\cfrac{\partial}{\partial x} F_X(x) \right] \cdot \left[ \cfrac{\partial}{\partial y} F_Y(y) \right] = f_X(x) f_Y(y)$

\item[$5^o$ б)] \fbox{$\Leftarrow$} (достаточность) \\
Пусть $f(x,y) = f_X(x) f_Y(y)$ \\
Тогда \\
$\displaystyle  F(x,y) = \int\limits_{-\infty}^{x} dt_1 \int\limits_{-\infty}^{y} f(t_1, t_2) dt_2 = \left( f(t_1, t_2) = f_X(t_1) f_Y(t_2) \right) = \left( \int\limits_{-\infty}^{x} f_X(t_1) dt_1 \right) \cdot \left( \int\limits_{-\infty}^{y} f_Y(t_2)dt_2 \right) = F_X(x) F_Y(y) \Rightarrow X,Y$ - независимы. \\
\underline{Пример:} Рассмотрим двумерный случайный вектор из примера выше \\
%Вставить таблицу
\end{enumerate}


\underline{Пример:} (см выше) \\
Функция плотности непрерывности случайного вектора $(X,Y)$ имеет вид $f(x,y) = 
\begin{cases}
	4xy, \ (x,y) \in K \\
	0, \ \text{иначе} \\
\end{cases}$ \\
%Вставить график
Исследовать независимость случайных величин X и Y.\\
\underline{Решение:} \\
Ранее были найдены маргинальная плотность \\
$f_X(x) = 
\begin{cases}
	2x, \ x \in [0,1] \\
	0, \ \text{иначе} \\
\end{cases}$ \\
$f_Y(y) = 
\begin{cases} 
	2x, \ y \in [0;1] \\
	0, \ \text{иначе} \\
\end{cases}$ \\
Так как $f(x,y) \equiv f_X(x) f_Y(y)$ то случайные величины X и Y - независимы (свойство $5^o$).


\underline{Определение:} Случайные величины $X_1, \ldots, X_n$ называются попарно независимыми, если \\
$\forall \forall i,j \in \{1, \ldots, n\}, \ i \neq j \Rightarrow$ $X_i$ и $Y_j$ - независимы. \\
-независимы в совокупности, если $F(x_1, \ldots, x_n) \equiv F_{X_1}(x_1) \cdot \ldots \cdot F_{X_n}(x_n)$, где F - функция распределения случайного вектора $(X_1, \ldots, X_n)$, \\
а $F_{X_i}, \ i = \overline{1,n}$ - маргинальная функция распределения его компонента. \\


\underline{Замечание:} Можно доказать \\
\begin{enumerate}
\item[1)] если $X_1, \ldots, X_n$ независимы в совокупности $\Rightarrow X_1, \ldots, X_n$ - попарно независимы.

\item[2)] для случайных величин $X_1, \ldots, X_n$ будут справедливы обобщения свойств $4^o$ и $5^o$ на случай независимых в совокупности случайных величин.
\end{enumerate}


















