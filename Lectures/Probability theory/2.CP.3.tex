% Независимые события.

Пусть A,B - события, связанные со случайным экспериментом.

\underline{Определение}: События A и B называются независимыми, если: \\
$P(AB) = P(A) P(B)$ \\

\underline{Th} 
\begin{enumerate}
	\item[1)]
	Пусть $P(B) > 0$ \\
	Тогда A,B - независимы $\Leftrightarrow P(A|B) = P(A)$
	
	\item[2)]
	Пусть $P(A) > 0$ \\
	Тогда A,B - независимы $\Leftrightarrow P(B|A) = P(B)$
\end{enumerate}

\underline{Доказательство}
Докажем 1).
\begin{enumerate}
	\item[а)]
	$\Rightarrow$ необходимость \\
	$P(B) > 0 \Rightarrow P(A|B) = \frac{P(BA)}{P(B)} = (\text{A,B - независимы}) = \frac{P(A) P(B)}{P(B)} = P(A)$
	
	\item[б)] 
	$\Leftarrow$ достаточность \\
	$P(AB) = (P(B) > 0 \Rightarrow \text{используем th умножения вероятностей}) = P(B) P(A|B) = (\text{по условию P(A|B) = P(A)}) = P(A) P(B) \Rightarrow$ A и B  независимы.
\end{enumerate}

\underline{Замечание}: В качестве определения независимых событий кажется более логичным выбрать условие $P(A|B) = P(A)$ или $P(B|A) = P(B)$, а не условие $P(AB) = P(A) P(B)$. Однако последнее условие работает всегда, в то время как первые два условия работают лишь при $P(B) > 0 (P(A) > 0)$.

\underline{Пример}: из колоды в 36 карт случайным образом извлекают одну карту. \\
A = \{извлечен туз\} \\
B = \{извлечена карта красной масти\}. \\
Выяснить,являются ли события A и B  независимыми. \\
Решение? \\
\begin{enumerate}
	\item[1)] 
	$P(A) = \frac{4}{36} = \frac{1}{9}$ \\
	
	\item[2)]
	$P(B) = \frac{18}{36} = \frac{1}{2}$ \\
	P(AB) = (AB = \{извлечен туз красной масти\} = $\frac{2}{36} = \frac{1}{18}$
	
	\item[3)]
	$P(AB) \stackrel{?}{=} P(A) P(B)$ \\
	$\frac{1}{18} = \frac{1}{9} \cdot \frac{1}{2}$ верно $\Rightarrow$ A,B - независимы.
\end{enumerate}

\underline{Th} Пусть A, B - независимы. Тогда независимыми являются события: \\
\begin{enumerate}
	\item[1)] 
	$\overline{A} \text{и} B$ \\
	
	\item[2)]
	$A \text{и} \overline{B}$
	
	\item[3)]
	$\overline{A} \text{и} \overline{B}$
\end{enumerate}

\underline{Доказательство}:
\begin{enumerate}
	\item[1)]
	Проверим равенство $P(\overline{A}B) = P(\overline{A}) P(B)$
	\begin{enumerate}
		\item[а)]
		если $\underbrace{P(B)}_{\overline{A}B \subseteq B \Rightarrow P(\overline{A}B) \leqslant P(B) = 0} = 0 \Rightarrow$ правая часть = 0 
		
		\item[б)]
		если $P(B) > 0$, то $P(\overline{A} B) = (\text{th умножения}) = P(B) \underbrace{P(\overline{A} | B}_{1 - P(A|B)} = (\text{условная вероятность обладает всеми свойствами вероятности}) = P(B) (1 - P(A|B) = (\text{A,B - независимы} \Rightarrow P(A|B) = P(A)) = P(B) \underbrace{1 - P(A)}_{P(\overline{A})} = P(\overline{A}) P(B)$. \\
		Левая часть = правая часть $\Rightarrow$ верно.
	\end{enumerate}
\end{enumerate}

2), 3) доказываются аналогично (доказать самостоятельно)

\underline{Пример}: События $A_1, \ldots , A_n$ называются независимыми попарно, если \\
$\forall \forall i,j \in \{1, \ldots , n\} \ \ i \neq j$ \\
$A_i$ и $A_j$ - независимы.

\underline{Определение}: События $A_1, \ldots , A_n$ называются независимыми в совокупности, если \\ 
$\forall K \in \{2, \ldots , n \}$ $\forall \forall i_1, \ldots , i_k: \ \ 1 \leqslant i_1 < i_2 < \ldots < i_k \leqslant n$ \\
$P(A_1 \cdot \ldots \cdot A_n) = P(A_{i_1}) \cdot \ldots \cdot P(A_{i_k})$.

\underline{Замечание}: Это определение означает, что $A_1, \ldots , A_n$ - независимые в совокупности, если:
\begin{enumerate}
	\item[1)] 
	$P(A_{i_1}, A_{i_2}) = P(A_{i_1}) P(A_{i_2}), \ \ i_1 < i_2. (k = 2)$ \\
	
	\item[2)]
	$P(A_{i_1} \cdot A_{i_2} \cdot A_{i_3}) = P(A_{i_1}) \cdot P(A_{i_2}) \cdot P(A_{i_3}). (k = 3)$
	
	\item[...]
	......
	
	\item[k-1)]
	$P(A_i \cdot \ldots \cdot A_n) = P(A_i) \cdot \ldots \cdot P(A_n). (k = n)$
\end{enumerate}

\underline{Замечание}:
\begin{enumerate}
	\item[1)]
	Очевидно, что \\
	$A_1, \ldots , A_n$ независимы в совокупности $\Rightarrow$ $A_1, \ldots , A_n$ независимы попарно. \\
	Обратное неверно.

\end{enumerate}

\underline{Пример}: Бернштейна.
Рассмотрим правильный тетраэдр на сторонах которого написаны цифры 1, 2, 3. \\
%Рисунок
Тетраэдр подбрасывают. \\
$A_1$ = \{на нижней грани есть 1\}
$A_2$ = \{на нижней грани есть 2\}
$A_3$ = \{на нижней грани есть 3\}
	
\begin{enumerate}
	\item[1)]
	$P(A_1) = \cfrac{2}{4} = \cfrac{1}{2}$\\
	$P(A_2) = \cfrac{1}{2}$ \\
	$P(A_3) = \cfrac{1}{2}$ \\

	\item[2)]
	$P(A_1 A_2) = (\text{на нижней грани есть 1 и 2}) = \cfrac{1}{4} = P(A_2 A_3) = P(A_3 A_1)$
	Таким образом \\
	$P(A_1 A_2) = P(A_1) P(A_2)$ \\
	$P(A_1 A_3) = P(A_1) P(A_3)$ \\
	$P(A_2 A_3) = P(A_2) P(A_3)$ \\
	$\Rightarrow A_1 A_2 A_3$ - независимы попарно.
	
	\item[3)]
	$P(A_1 A_2 A_3) = (\text{на нижней грани присутствуют 1, 2, 3}) = \cfrac{1}{4} \neq \cfrac{1}{8} = P(A_1) P(A_2) P(A_3)$ \\
	Таким образом $A_1 , A_2 , A_3$ не являются независимыми в совокупности.
\end{enumerate}	













