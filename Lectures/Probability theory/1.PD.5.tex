% Статистическое определение вероятности.

Пусть 
\begin{enumerate}
	\item[1)]
	$\Omega$ - пространство элементарных исходов некоторого случайного эксперимента.
	
	\item[2)]
	$A \subseteq \Omega$ - событие, связанные с этим экспериментом.
	
	\item[3)]
	Этот случайным эксперимент произведен n раз, при этом событие A произошло $n_A$ раз.
\end{enumerate}

\underline{Определение}: Вероятностью события A называется эмпирический (то есть из эксперимента) предел \\
$P(A) = \lim\limits_{n \rightarrow \infty} \frac{n_A}{n}$

\underline{Замечание}:
\begin{enumerate}
	\item[1)] 
	из статистического определения ... те же свойства вероятности, что и из двух предыдущих определений.
	
	\item[2)]
	Недостатки статистического определения:
	\begin{enumerate}
		\item[-]
		никакой эксперимент невозможно осуществить бесконечное число раз.
		
		\item[-]
		с точки зрения современной математики эти определения являются архаизмом, так как не дают достаточной базы для развития теории.
	\end{enumerate}
\end{enumerate}









































