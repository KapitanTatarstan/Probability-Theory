% Дискретные случайные векторы.

\underline{Определение}: Случайный вектор $\vec{X} = (X_1, \ldots , X_n)$ называется дискретным, если каждая из случайных величин $X_i, \ i = \overline{1,n}$ является дискретной. \\

Рассмотрим случай n = 2- случайный вектор $(x,y)$. Для упрощения рассуждений будем считать, что случайная величина X и Y принимают значения из конечных множеств. \\
$X \in \{x_1, \ldots, x_n\}$, $Y = \{y_1, \ldots , y_n\}$ \\
Закон распределения такого случайного вектора удобно задавать с использованием таблицы \\
%Вставить таблицу

Здесь \\
$p_{ij} = P\{(X,Y) = (x_i, y_i)\} = P\{ \{X = x_i\} \cdot \{Y = y_j\} \}$ \\ 
Эту таблицу дополняют строкой и столбцом. В j-й клетке дополнительной строки записывают величину $P_{Y_j} = \sum\limits_{i=1}^{m} p_{ij}$ \\
В i-ой клетке дополнительной строки записывают $P_{X_i} = \sum\limits_{j=1}^{n} p_{ij}$ \\
Покажем, что \\
$P_{X_i} = P\{X = x_i\}, \ p_{Y_j} = P\{Y = y_j\}$ \\
$P\{X = x_i\} = P\{(X,y) \in \{(x_i, y_1), \ldots, (x_i, y_n\} \} = P\{ \{(X, Y) = (x_i, y_1)\} + \ldots + \{(X,Y) = (x_i, y_n)\} \}$ = (th сложения) = $\sum\limits_{j = 1}^{n} \underbrace{P\{(X, Y) = (x_i, y_j)\}}_{= P_{ij}} = P_{X_i}$ \\
Второе равенство доказывается аналогично. \\
При этом, очевидно, должно выполняться условие нормировки: \\
$\sum\limits_{i = 1}^{m} P_{X_i} = \sum\limits_{j = 1}^{n} P_{Y_j} = \sum\limits_{i = 1}^{m} \sum\limits_{j = 1}^{n} p_{ij} = 1$ \\

\underline{Пример}: Симметричную монету подбрасывают 2 раза. \\
X - количество выпадений герба. \\
Y = - номер броска, при котором герб выпал впервые (будем считать, что Y = 3, если герб ни разу не выпал). \\

%Вставить таблицу 
$P\{(X, Y) = (0, 1)\}$ = P\{\{герб не выпал ни разу\} $\cdot$ \{герб выпал при 3-м броске\} \\
$P\{(X, Y) = (0, 3)\}$ = P\{\{при первом броске была решка\} $\cdot$ \{при втором - решка\}\} = $\cfrac{1}{4}$ \\
$P\{(X, Y) = (1, 1)\}$ = P\{\{герб\} $\cdot$ \{решка\}\} = $\cfrac{1}{4}$ \\
$P\{(X, Y) = (1, 2)\}$ = P\{(Р, Г)\} = $\cfrac{1}{4}$ \\
$P\{(X, Y) = (1, 3)\}$ = 0 \\
$P\{(X, Y) = (2, 1)\}$ = P\{(Г, Г)\} = $\cfrac{1}{4}$ \\
$P\{(X, Y) = (2, 2)\} = 0$ \\
$P\{(X, Y) = (2, 3)\} = 0$ \\



























