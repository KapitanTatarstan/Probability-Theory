% 5. Условные распределения.

Пусть 
\begin{enumerate}
\item[1)] $(X,Y)$ - двумерный случайный вектор.
\item[2)] известно, что случайная величина Y приняла значение $Y = y_0$
\end{enumerate}
Вопрос: \\
\begin{enumerate}
\item[1)] что в этом случае можно сказать о значениях случайной величины X?
\item[2)] что можно сказать о распределении вероятностей между этими возможными значениями случайной величины X?
\end{enumerate}


I. Случай дискретного случайного вектора. \\
Пусть: \\
\begin{enumerate}
\item[1)] $(X,Y)$ - дискретный случайный вектор. 
\item[2)] $X \in \{x_1, \ldots, x_m\}$ \\
$Y = \{y_1, \ldots, y_n\}$ \\
\end{enumerate}
$p_{ij} = P \left\{ (X,Y) = (x_i,y_j) \right\}, \ i = \overline{1,m}, \ j = \overline{1,n}$ \\
$p_{x_i} = P\{X = x_i\}, \ i = \overline{1,m}$ \\
$p_{y_j} = P\{Y = y_j\}, \ j = \overline{1,n}$ \\
%Вставить напоминание
$P\{ \underbrace{X = x_i}_{A} | \underbrace{Y = y_j}_{B} \}$ = (определение условной вероятности) = $ \cfrac{P \left\{ (X,Y) = (x_i,y_j) \right\}}{P\{Y = y_j\}} = \cfrac{p_{ij}}{p_{y_j}}$


\underline{Определение:} Условным распределением компоненты X двумерного дискретного случайного вектора при условии, что случайная величина $Y = y_j$ называется набор чисел \\
$\Pi_{ij} = \cfrac{p_{ij}}{p_{y_j}}, \ i = \overline{1,n}$ \\
(Для каждого значения j будет "свое"\ условие распределения случайной величины X, так как для каждого j имеет место "свое"\ условие $Y = y_j$). \\


\underline{Замечание:} Аналогичным образом определяется условное распределение случайной величины Y при условии $X = x_i$: \\
$\tau_{ij} = \cfrac{p_{ij}}{p_{x_i}}, \ i = \overline{1,n}$ \\
(Для каждого $i = \{1, \ldots, m\}$ "свое"\ условие $\{X = x_i\}$ и "свое условное распределение"\ распределение).


\underline{Пример:} Рассмотрим двумерный случайный вектор из задачи о подбрасывании монеты. \\
%Вставить таблицу
\begin{enumerate}
\item[а)] Найдем условное распределение случайной величины X. \\
$\Pi_{ij} = \cfrac{p_{ij}}{p_{y_j}}, \ j = \overline{1,2}$ \\
%Вставить таблицу
$\Pi_{i1} = \cfrac{p_{ij}}{p_{y_j}}$ \\ %Вставить пояяснения
$\Pi_{i2} = \cfrac{p_{ij}}{p_{y_j}}$ \\

\item[б)] Найдем условное распределение случайной величины X. \\
$\tau_{ij} = \cfrac{p_{ij}}{p_{x_i}}, \ j = \overline{1,3}$ \\
%Вставить таблицу
\end{enumerate}


Пусть $(X,Y)$ - \underline{\underline{произвольный}} (не обязательно дискретный или непрерывный) случайный вектор. \\
\underline{Определение:} Условной функцией распределения случайной величины X при условии, что $Y = y$ называется отображение \\
$F_X(x|Y = y) = P\{X < x|Y = y\}$ \\

\underline{Замечание:} 
\begin{enumerate}
\item[1)] Условная функция распределения компоненты Y определяется аналогично: \\
$F_Y(y|X = x) = P\{Y < y|X = x\}$

\item[2)] Если $(X,Y)$ - дискретный случайный вектор из пункта 1), то \\
$F_X(x|Y = y_j) = \cfrac{P\{X < x|Y = y_j\}}{P\{Y = y_j\}} = \cfrac{\sum p_{ij}}{p_{y_j}}$ $(*)$
\end{enumerate}


II. Случай непрерывной случайного вектора. \\
Пусть
\begin{enumerate}
\item[1)] $(X,Y)$ - непрерывный случайный вектор
\item[2)] $f(x,y)$ - совместная плотность распределения случайной величины X и Y.
\end{enumerate}
В случае непрерывного случайного вектора использовать формулу $(*)$ "в лоб"\ не получится, так как для любого наперед заданного y $P\{Y = y\} = 0$ \\


Рассуждение, аналогичные ... приведенными в пункте \fbox{I.} приводят к следующему определению. \\
\underline{Определение:} Условной плотностью распределения случайной величины X при условии $Y = y$ называется функция \\
$f_X(x|Y = y) = \cfrac{f(x,y)}{f_Y(y)}$, где \\
$f_Y(y)$ - маргинальная функция плотности распределения случайной величины Y; \\
$f_Y(y) \neq 0$


\underline{Замечание:} Условной плотностью случайной величины Y при условии $X = x$ определяется аналогично \\
$f_Y(y|X = x) \ cfrac{f(x,y)}{f_X(x)}$ \\
где $f_X(x) \neq 0$ - маргинальная функция плотности распределения случайной величины X. \\


\underline{Th} критерии независимости случайной величины в терминах условных распределений.
\begin{enumerate}
\item[1)] Пусть $(X,Y)$ - двумерный случайный вектор. \\
Тогда 3 следующих утверждения эквивалентны: \\
	\begin{enumerate}
	\item[а)] X и Н независимы
	\item[б)] $F_X(x) \equiv F_X(x|Y = y)$ для всех y, при которых определена функция $F_X(x|Y = y)$
	\item[в)] $F_Y(y) \equiv F_Y(y|X = x)$ для всех x, для которых определена функция $F_Y(y|X = x)$
	\end{enumerate}
	
\item[2)] Пусть $(X,Y)$ - дискретный случайный вектор. \\
Тогда следующие 3 утверждения эквивалентны: \\
	\begin{enumerate}
	\item[а)] $X,Y$ - независимы 
	\item[б)] $P\{X = x_i\} \equiv P\{X = x_i|Y = y_j\}$ для всех $y_j$
	\item[в)] $P\{Y = y_j\} \equiv P\{Y = y_j|X = x_i\}$ для всех $x_i$
	\end{enumerate}
	
\item[3)] Пусть $(X,Y)$ - непрерывные случайные величины. \\
Тогда следующие 3 утверждения эквивалентны \\
	\begin{enumerate}
	\item[а)] $(X,Y)$ - независимы
	\item[б)] $f_X(x) \equiv f_X(x|Y = y)$ для всех y, для которых определена $f_X(x|Y = y)$
	\item[в)] $f_Y(y) \equiv f_Y(y|X = x)$ для всех x, для которых определена $f_Y(y|X = x)$
	\end{enumerate}
\end{enumerate}


\underline{Пример:} Рассмотрим задачу о подбрасывании монеты (см. выше). \\
$ \left. 
\begin{array}{lll}
	P\{X = 0\} = \cfrac{1}{4} \\
	P\{X = 0|Y = 3\} = 1 \\
\end{array} \right\} \Rightarrow \left(1 \neq \cfrac{1}{4} \right) \Rightarrow$ X и Y - зависимы.



























