% 6. Ковариация.


До сих пор мы изучали численные характеристики одномерных случайных величин. Ковариация является характеристикой случайного вектора.\\
Пусть $(X,Y)$ - двумерный случайный вектор.\\


\underline{Определение:} Ковариацией случайной величины X и Y называется число\\
$cov(X, Y) = M\left[(X - m_1)(Y - m_2)\right]$\\
где $m_1 = MX$, $m_2 = MY$.\\


\underline{Замечание:}\\
\begin{enumerate}
	\item[1)] если $(X, Y)$ -  дискретный случайный вектор, то\\
	$cov(X, Y) = \sum\limits_i \sum\limits_j p_{ij} (x_i - m_1)(y_j - m_2)$,\\
	где $p_{ij} = P\{(X, Y) = (x_i, y_j)\}$\\
	
	\item[2)] если $(X, Y)$ - непрерывный случайный вектор, то\\
	$\displaystyle  cov(X, Y) = \iint\limits_{\mathbb{R^2}} (x - m_1)(y - m_2) f(x,y) dx dy$,\\
	где $f(x,y)$ - совместная плотность распределения X и Y.\\
\end{enumerate}


\underline{Свойства ковариации}\\
\begin{enumerate}
	\item[$1^o$] $D(X + Y) = DX + DY + 2cov(X, Y)$
	\item[$2^o$] $cov(X, X) = DX$
	\item[$3^o$] Если X, Y - независимы, то $cov(X, Y) = 0$
	\item[$4^o$] $cov(a_1 X + a_2, b_1 Y + b_2) = a_1 b_1 \cdot cov(X, Y)$
	\item[$5^o$] $|cov(X, Y)| \leqslant \sqrt{DX \cdot DY}$\\
	 причем\\
	 $|cov(X, Y)| = \sqrt{DX \cdot DY} \Leftrightarrow$ X и Y связаны мн. зависимостью, то есть  $Y = aX + b$, где $a, b = const$ 
	 \item[$6^o$] $cov(X, Y) = M[XY] - (MX)(MY)$
\end{enumerate}


\underline{Доказательство} 
\begin{enumerate}
	\item[$1^o$] $D(X + Y) = M\left[ \left\{ (X + Y) - (M(X + Y))\right\}^2 \right] = \left\{
	\begin{matrix}
		m_1 = MX \\
		m_2 = MY \\
	\end{matrix} \right\} = 
	M\left[ \left\{ (X - m_1) + (Y - m_2) \right\}^2 \right] = 
	M \left[ (X - m_1)^2 + (Y - m_2)^2 + 2M\left[ (X - m_1)(Y - m_2)\right] \right] = 
	\underbrace{M\left[(X - m_1)^2\right]}_{DX} + \underbrace{M\left[(Y - m_2)^2\right]}_{DY} + 2 \underbrace{M\left[(X - m_1)(Y - m_2)\right]}_{cov(X, Y)} = DX + DY + 2 cov(X, Y)$\\
	
	\item[$2^o$] $cov(X, X) = M\left[(X - m)(X - m)\right] = M\left[(X - m)^2\right] = DX$\\
	
	\item[$3^o$] $cov(X, Y) - M\left[(X - m_1)(Y - m_2)\right] = \underbrace{M\left[(X - m\right]}_{m = 0} \cdot M[Y - m_2] = 0$\\
	%  Вставить пояснения в формулу
	
	\item[$4^o$] $cov(a_1 X + a_2, \ b_1 Y + b_2) = \left\{
	\begin{array}{lll}
		\text{Обозначения:}\\
		\mu_1 = M[a_1 X + a_2] \\
		\mu_2 = M[b_1 Y + b_2] \\
	\end{array} \right\} = 
	M\left[(a_1 X + a_2 - \mu_1)(b_1 Y + b_2 - \mu_2)\right] = \left\{
	\begin{array}{lll}
		\mu_1 = a_1 m_1 + a_2 \\
		\mu_2 = b_1 m_2 + b_2 \\
	\end{array} \right\} = 
	M\left[(a_1 X - a_2 m_1)(b_1 Y - b_1 m_2)\right] = 
	\left\{\mu - \text{ линейно}\right\} = a_1 b_1 M\left[(X - m_1)(Y - m_2)\right] = a_1 b_1 \cdot cov(X, Y)$.\\
	
	\item[$6^o$] $cov(X,Y) = M\left[(X - m_1)(Y - m_2)\right] = M\left[XY - m_2 X - m_1 Y + m_1 m_2\right] = M[XY] - \underbrace{m_2 MX}_{m_1 m_2} - \underbrace{m_1 MY}_{m_1 m_2} + m_1 m_2 = M[XY] - m_1 m_2$\\
	
	\item[$5^o$] 
	\begin{enumerate}
		\item[а)] Покажем, что $|cov(X, Y)| \leqslant \sqrt{DX \cdot DY}$\\
		Рассмотрим случайную величину $Z(t) = tX - Y, \ t \in \mathbb{R}$\\
		Тогда\\
		$D[Z(t)] = D[tX - Y] = \left\{\text{свойство } 1^o \right\} = D(tX) - D(-Y) + 2 cov(tX, -Y) = \left\{D(aX) = a^2 DX - \text{свойство } 4^o cov \right\} = t^2 DX + DY - 2t cov(X, Y) = (DX)t^2 - 2(cov(X, Y))t + DY \geqslant 0$\\
		% Вставить пояснения в формулу
		% Вставить график
		Дисперсия = $\left\{ \left(\cfrac{t^2}{2}\right) - ac\right\} = |cvo(X, Y)|^2 - DX \cdot DY \leqslant 0 \Rightarrow |cov(X, Y) \leqslant \sqrt{DX \cdot DY}$\\
		
		\item[б)] Покажем, что\\
		$|cov(X, Y) = \sqrt{DX \cdot DY} \Leftrightarrow Y = aX + b$\\
		\fbox{$\Rightarrow$} Необходимость\\
		Если $|cov(X, Y)| = \sqrt{DX \cdot DY} \Rightarrow$ дисперсия = 0 $\Rightarrow$ уравнение $D[Z(t)] = 0$ имеет единственный корень.\\
		Обозначим его $t = a$\\
		Тогда случайная величина $Z(a) = aX - Y$  имеет $D[Z(a)] = 0 \Rightarrow$ Она принимает одно единственное значение (обозначим его - b)\\
		с вероятностью 1, то есть $Z(a) = \underbrace{aX - Y = -b}_{Y = aX + b}$\\
		
		\fbox{$\Leftarrow$} Достаточность\\
		Если $Y = aX + b$, то случайная величина $Z(a) = -b \Rightarrow D(Z(a)) = 0 \Rightarrow$ Дисперсия = 0 $\Rightarrow |cov(X, Y)| = \sqrt{DX \cdot DY}$\\
	\end{enumerate}
\end{enumerate}


\underline{Замечание:}\\
\begin{enumerate}
	\item[1)] Свойства $1^o$ с учетом $4^o$ допускает обобщение\\
	$D[aX + bY + c] = a^2 DX + b^2 DY + 2ab cov(X, Y)$\\
	
	\item[2)] Рассмотрим свойство $5^o$.\\
	Пусть $Y = aX + b \Rightarrow$ $\left\{\text{в соответствии с } 4^o\right\}$ $\Rightarrow cov(X, Y) = cov(X, aX + b) = a \cdot cov(X, X) = a DX$\\
	Если $DX \geqslant 0$, то знак $(cov(X, Y))$ =  значу $(a)$.\\
	Таким образом свойство $5^o$ можно уточнить:\\
	Если $Y = aX + b$, то\\
	$cov(X, Y) = 
	\begin{cases}
		\sqrt{DX \cdot DY}, \ \text{если } a > 0\\
		-\sqrt{DX \cdot DY}, \ \text{если } a < 0\\
	\end{cases}$
\end{enumerate}


\underline{Определение:} Случайная величина X и Y называется некоррелированными, если\\
$cov(X, Y) = 0$\\


\underline{Замечание:} из свойства $3^o$ $\Rightarrow$\\
X, Y - независимы $\Rightarrow$ X, Y - некоррелированные.\\
Обратное неверно.\\

ДЗ: привести пример случайного вектора (X, Y), для которого X, Y - зависимы, но некоррелируемы.\\


\underline{Замечание:} недостатком ковариации является то, что она имеет размерность произведения случайных величин X и Y. В то же самое время удобно иметь некоторую безразмерную характеристику.\\


\underline{Определение:} Коэффициентом корреляции случайных величин X и Y называют число\\
$\rho(X, Y) = \cfrac{cov(X, Y)}{\sqrt{DX \cdot DY}}$\\
(здесь предполагается, что $DX \cdot DY > 0$\\


\underline{Свойства коэффициента корреляции}\\
\begin{enumerate}
	\item[$1^o$] $\rho(X, X) = 1$
	\item[$2^o$] Если X, Y - независимы, то $\rho(X, Y) = 0$
	\item[$3^o$] $\rho(a_1 X + a_2, \ b_1 Y b_2) = \pm \rho(X, Y)$\\
	где "+"\ , если $a_1 a_2 > 0$\\
	" - "\  , если $a_1 a_2 < 0$
	
	\item[$4^o$] $|\rho(X, Y)| \leqslant 1$, причем\\
	$|\rho(X, Y) = 1| \Leftrightarrow$ X и Y связаны линейной зависимостью $Y = aX + b$.\\
	При этом \\
	$\rho(X, Y) = 
	\begin{cases}
		1, \ \text{если } a > 0 \\
		-1, \ \text{если } a < 0 \\
	\end{cases}$\\
\end{enumerate}


\underline{Доказательство}\\
Эти свойства являются прямым следствием свойств ковариации.\\
(Доказать самостоятельно).\\


\underline{Замечание:} Коэффициент корреляции показывает "степень"\ линейной зависимости случайный величин X и Y.\\
Пусть проведено достаточно много экспериментов, и все реализации вектора (X, Y) изображена на плоскости.\\
% Вставить график
Чем ближе эти реализации группируются около некоторой прямой, тем ближе значение $(\rho)$ к 1.\\
Если все эти значения лежат на одной прямой, то $|\rho| = 1$.\\
При этом если соответствующая прямая имеет положительный коэффициент, то $\rho > 0$ и $\rho \leqslant 1$. \\
Если оно...
%  Добавить концовку


Пусть $\vec{X} = (X_1, \ldots, X_n)$ - n-мерный случайный вектор.
\\
\underline{Определение:} Ковариационной матрицей вектора $\vec{X}$ называется матрица \\
$sum = (\sigma_{ij})$ \ $i, j = \overline{1,n}$,\\
где $\sigma_{ij} = cov(X_i, X_j)$
\\
\underline{Замечание:}
\begin{enumerate}
	\item[1)] $\sigma_{ij} = D[X_i]$, $i = \overline{1,n}$
	
	\item[2)] $\sum^T = \sum$, так как \\
	$cov(X_i, X_j) = cov(X_i, X_j)$
	
	\item[3)] Если  $\vec{Y} = \vec{X}B + \vec{C}$\\
	где $B \in M_{m,m} (\mathbb{R})$\\
	$\vec{X} = (X_1, \ldots, X_m)$\\
	$\vec{Y} = (Y_1, \ldots, Y_m)$\\
	$\vec{C} = (C_1, \ldots, C_m)$\\
	(B, $\vec{C}$ - числовые матрицы)\\
	то $\sum \vec{Y} = B^T \sum_{\vec{X}} B$\\
	
	\item[4)] Матрица $\sum$ является неотрицательно определенной (то есть соответствующая квадратичная форма является неотрицательно определенной) то есть\\
	$\forall \vec{x} = (x_1, \ldots, x_n) \in \mathbb{R}$\\
	$f(\vec{x}) = \vec{x} \sum \vec{x}^T \geqslant 0$\\
	
	\item[5)] Если все компоненты случайного вектора $\vec{X}$ попарно независимы, то его ковариационная матрица является диагональной, так как\\
	$cov(X_i, X_j)$ = (если $i \neq j$, то $X_i, X_j$ - независимы) = 0.
\end{enumerate}


\underline{Определение:} Ковариационной матрицей вектора $\vec{X}$ называется матрица\\
$P = (\rho_{ij})$, $i, j = \overline{1,n}$\\
где $\rho_{ij} = \rho(X_i, X_j)$\\





























