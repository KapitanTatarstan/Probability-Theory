%Определение условной вероятности.

Пусть
\begin{enumerate}
	\item[1)]
	A, B - случайные события связанные с некоторым экспериментом.
	
	\item[2)]
	Известно, что в результате эксперимента произошло событие B.
\end{enumerate}

Как эта информация повлияет на вероятность что, в результате этого эксперимента произошло событие A?

\underline{Пример}: Из колоды в 36 карт случайным образом извлекают одну карту. \\
A = \{извлекают туз\} \\
B = \{извлечена картинка\} \\
Тогда \\
$P(A) = \frac{4}{36} = \frac{1}{9}$ \\
$P_B(A) =$ (наступило B $\Rightarrow$ извлечена карта: Валет, Дама, Король, Туз, то есть извлечена одна из 16 карт) $ = \frac{4}{16} = \frac{1}{4}$

\underline{Замечание}: Рассмотрим классическую схему для определения вероятности. Имеется N равно возможных исходов. \\
$|A| = N_A, \ |B| = N_B$
%Рисунок

Так как известно, что в результате случайного эксперимента наступило B, то все исходы, не попавшие B, можно не рассматривать. \\
В этом случае событие A может наступить лишь при реализации одного из исходов, входящих в AB. \\
Поэтому \\
$P_B(A) = \frac{N_{AB}}{N_B} = \frac{N_{AB} / N}{N_B / N} = \frac{P(AB)}{P(B)}$

Пусть $(\Omega, \beta, P)$ - вероятностное пространство (не обязательно оно реализует классическую схему).

\underline{Определение}: Условной вероятностью осуществления события A при условии, что произошло событие B, называется число \\
$P(A|B) = \frac{P(AB)}{P(B)}$, $P(B) \neq 0$

\underline{Замечание}:
\begin{enumerate}
	\item[1)]
	Для того, чтобы подчеркнуть разницу, "об..." вероятностью иногда будем называть безусловной.
	
	\item[2)]
	Зафиксируем некоторое событие B и будем рассматривать $P(A|B)$ как функцию события $A \in \beta$
\end{enumerate}

\underline{Th}: Условная вероятность $P(A|B)$ удовлетворяет трем аксиомам безусловной вероятности.


\underline{Доказательство}
\begin{enumerate}
	\item[1)]
	$P(A|B) = \frac{P(AB)}{P(B)} \geqslant 0$
	
	\item[2)]
	$P(\Omega|B) = \frac{P(\Omega B)}{P(B)} = \frac{P(B)}{P(B)} = 1$
	
	\item[3)]
	Пусть $A_1, \ldots , A_n, \ldots$ - набор попарно непересекающихся событий \\
	$P(A_! + A_2 + \ldots |B) \frac{P(A_1 + A_2 + \ldots |B)}{P(B)} = (\text{свойство ...}) = \frac{P(A_1 B + A_2 B + \ldots)}{P(B)} = $
	\begin{enumerate}
		\item[а)]
		$A_i \cdot A_j = \emptyset$ при $i \neq j$
		
		\item[б)]
		$A_i B \subseteq A_i$ \\
		$A_j B \subseteq A_j$ \\
		
		\item[в)]
		а)б) $\Rightarrow (A_i B)(A_j B) = \emptyset \Rightarrow$ рассматрив. аксиому сложения для набора $A_1 B, A_2 b, \ldots$
	\end{enumerate}
	$= \frac{P(A_1 B) + P(A_2 B) + \ldots}{P(B)} = (\text{... свойство сходимости рядов}) = \frac{P(A_1 B)}{P(B)} + \frac{P(A_2 B)}{P(B)} + \ldots = P(A_1|B) + P(A_2|B) + \ldots$
\end{enumerate}

\underline{Следствие}. Условная вероятность $P(A|B)$  при фиксированном событии B обладает всеми свойствами безусловной вероятности:
\begin{enumerate}
	\item[$1^o$]
	$P(\overline{A}|B) = 1 - P(A|B)$
	
	\item[$2^o$]
	$P(\emptyset|B) = 0$
	
	\item[$3^o$]
	Если $A_1 \subseteq A_2$, то $P(A_1|B) \subseteq P(A_2|B)$
	
	\item[$4^o$]
	$0 \leqslant P(A|B) \leqslant 1$
	
	\item[$5^o$]
	$P(A_1 + A_2 |B) = P(A_1|B) + P(A_2|B) - P(A_1 A_2|B)$
	
	\item[$6^o$]
	Для любого конечного набора событий $A_1, \ldots, A_n$ \\
	$P(A_1 + \ldots + A_n|B) = 
	\sum\limits_{1 \leqslant i_1 \leqslant n} P(A_{i_1}|B) - 
	\sum\limits_{1 \leqslant i_1 < i_2 \leqslant n} P(A_{i_1} A_{i_2}|B) + 
	\sum\limits_{1 \leqslant i_1 < i_2 < i_3 \leqslant n} P(A_{i_1} A_{i_2} A_{i_3}|B) + \ldots + 
	(-1)^{n-1} P(A_1 \cdot A_2 \cdot \ldots \cdot A_n|B)$
\end{enumerate}

\underline{Доказательство}
Свойства $1^o - 6^o$ безусловной вероятности является следствием аксиом $1^o - 3^o$ вероятности. Так как, условная вероятность удовлетворяет этим аксиомам, то для нас выполняются и аналогии свойств $1^o - 6^o$.

\underline{Пример}: Среди 15 лотерейных билетов 5 выигрышных. Сначала 1-ый игрок тянет 1 билет затем 2-ой игрок тянет 1 билет. \\
$A_1$ = \{1-ый игрок достал выигрышный билет\} \\
$A_2$ = \{2-ой игро достал выигрышный билет\} \\
$P(A_2|A_1) = ?$ \\

\underline{Решение}: 
\underline{I способ}: по определению условной вероятности \\
$P(A_2|A_1) = \frac{P(A_1 A_2)}{P(A_1)}$ \\
\begin{enumerate}
	\item[1)] 
	Исход: $(x_1, x_2)$, где $x_I$ - номер билета, извлеченный i-ым игроком, $x_i \in \{1, \ldots , 15\}$
	%Рисунок
	$(x_1, x_2)$ - размещение без повторений из 15 по 2. \\
	$N = 15 \cdot 14$ \\
	
	\item[2)]
	$N_{A_1} = 5 \cdot 14$ \\
	$P(A_1) = \frac{5 \cdot 14}{15 \cdot 14} = \frac{1}{3}$
	
	\item[3)]
	$N_{A_1 A_2} = 5 \cdot 4 = 20$ \\
	$P(A_1 A_2) = \frac{20}{15 \cdot 14} = \frac{2}{11}$ \\
	
	\item[4)]
	$P(A_2|A_1) = \frac{P(A_1 A_2)}{P(A_1)} = \frac{2/21}{1/3} = \frac{1}{7}$ \\
\end{enumerate}

\underline{II способ}: подсчитает $P(A_2|A_1)$, перестроив пространство $\Omega$. \\
$P(A_2|A_1)$ = (известно, что наступило $A_1 \Rightarrow$ осталось 14 билетов из которых 4 выигрышных) = $\frac{4}{14} = \frac{2}{7}$



























