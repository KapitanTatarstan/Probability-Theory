% 5. Замена переменных в тройном интеграле.

\underline{Th:} Пусть 
\begin{enumerate}
	\item[1)]
	$G_{xyz} = \Phi (G_{uvw}).$
	
	\item[2)]
	$\Phi: G_{uvw} \to G_{xyz}$ \\
	$\Phi: 
	\begin{cases}
		x = x(u,v,w) \\
		y = y(u,v,w) \\
		z = z(u,v,w) \\
	\end{cases}$
	
	\item[3)] 
	Отображение $\Phi$ биективно
	
	\item[4)]
	$\Phi$ непрерывна и непрерывно-дифференцируемы в $G_{uvw}$
	
	\item[5)]
	$J_\Phi (u,v,w) = 
	\begin{vmatrix}
		x'_u & x'_v & x'_w \\
		y'_u & y'_v & y'_w \\
		z'_u & z'_v & z'_w \\
	\end{vmatrix} \neq 0$
	
	\item[6)] 
	$f(x,y,z)$ интегрируема в $G_{xyz}$ 
\end{enumerate}

Тогда \\
$\iiint\limits_{G} f(x,y,z) dxdydz = 
\iiint\limits_{G_{uvw}} f\left( x\left(u,v,w\right), y\left(u,v,w\right), z\left(u,v,w\right) \right)  \left| J_\Phi (u,v,w) \right| du dv dw$

\underline{Пример:} Цилиндрическая система координат: \\

Декартовая система координат:

Цилиндрическая система координат:

Связь цилиндрической и декартовой системой координат: \\

$\begin{cases}
	x = \rho \cdot \cos{\varphi} \\
	y = \rho \cdot \sin{\varphi} \\
	z = z \\
\end{cases}$\\

$J_\text{цил.}  = 
\begin{vmatrix}
	\cos{\varphi} & -\rho \cdot \sin{\varphi} & 0 \\
	\sin{\varphi} & \rho \cdot \cos{\varphi} & 0 \\
	0 & 0 & 1 \\
\end{vmatrix} = \rho$ \\


Сферическая система координат:

Связь декартовой и полярной системы координат:
$\begin{cases}
	x = r \cdot \cos{\Theta} \cdot \cos{\varphi} \\
	y = r \cdot \cos{\Theta} \cdot \sin{\varphi} \\
	z = r \cdot \sin{\Theta} \\
\end{cases} $ \\

$\left| J_\text{сф.}  \right| = 
\begin{vmatrix}

\end{vmatrix} = r^2 \cdot \cos{\Theta}$





























