% Аксиоматическое определение вероятности.

Пусть
\begin{enumerate}
	\item[1)]
	$\Omega$ - пространство элементарных исходов некоторого случайного эксперимента.
	
	\item[2)]
	$\beta$ - $\sigma$-алгебра на $\Omega$
\end{enumerate}

\underline{Определение}: Вероятностью (вероятностной мерой) называется отображение \\
$P : \beta \rightarrow \mathbb{R}$, \\
обладающее свойствами: 
\begin{enumerate}
	\item[$1^o$]
	$\forall A \in \beta$ $P(A) \geqslant 0$ (аксиома неотрицательности)
	
	\item[$2^o$]
	$P(\Omega) = 1$ (аксиома нормированности)
	
	\item[$3^o$]
	если $A_1, \ldots , A_n, \ldots \in \beta$ попарно-несовместные события, то $P(A_1 + \ldots + A_n + \ldots) = P(A_1) + \ldots + P(A_n) + \ldots$ (расширенная аксиома сложения).
\end{enumerate}

\underline{Определение}: Тройка $(\Omega, \beta, P)$ называется вероятностным пространством.

\underline{Свойства вероятности} (следствия из аксиоматического определения.
\begin{enumerate}
	\item[$1^o$]
	$P(\overline{A}) = 1 - P(A)$
	
	\item[$2^o$]
	$P(\emptyset) = 0$
	
	\item[$3^o$]
	Если $A \subseteq B$, то $P(A) \subseteq P(B)$
	
	\item[$4^o$]
	$\forall A \in \beta$ $0 \leqslant P(A) \leqslant 1$
	
	\item[$5^o$]
	$P(A+B) = P(A) + P(B) - P(AB)$
	
	\item[$6^o$]
	Для любого \underline{конечного} набора событий $A_1, \ldots , A_n \in \beta$ справедливо $P(A_1 + \ldots + A_n) = \sum\limits_{1 \leqslant i \leqslant n} P(A_i) - 
	\sum\limits_{i \leqslant i_1 < i_2 \leqslant n} P(A_{i_1} A_{i_2}) + 
	\sum\limits_{i \leqslant i_1 < i_2 < i_3 \leqslant n} P(A_1 A_2 A_3) - 
	\ldots (-1)^{n+1} P(A_1 \cdot \ldots \cdot A_n)$
\end{enumerate}

\underline{Доказательство}
\begin{enumerate}
	\item[$1^o$]
	\begin{enumerate}
		\item[а)]
		$A + \overline{A} = \Omega$
		
		\item[б)]
		$A \overline{A} = \emptyset \Rightarrow (\text{акс } 3^o) \Rightarrow P(A + \overline{A}) = P(A) + P(\overline{A})$
		
		\item[в)]
		б) $\Rightarrow P(\overline{A}) = 1 - P(A)$
	\end{enumerate}
	
	\item[$2^o$]
	$P(\emptyset) = P(\overline{\Omega}) = (\text{свойство} 1^o) = 1 - \underbrace{P(\Omega)}_{1} = 0$
	
	\item[$3^o$]
	$A \subseteq B$ \\
	$B = A + (B \backslash A) \Rightarrow P(B) = P(A) + \underbrace{P(B \backslash A)}_{\geqslant 0} \Rightarrow P(B) \geqslant P(A)$
	
	\item[$4^o$]
	\begin{enumerate}
		\item[а)]
		$P(A) \geqslant 0$ вытекает из аксиомы $1^o$
		
		\item[б)] 
		Покажем, что $P(A) \leqslant 1$ $A \subseteq \Omega \Rightarrow (\text{по свойству} 3^o) \Rightarrow P(A) \leqslant \underbrace{P(\Omega}_{1}$
	\end{enumerate}
	
	\item[$5^o$]
	\begin{enumerate}
		\item[a)]
		%РИСУНОК
		$A + B = A + (B \backslash A) \Rightarrow (\text{аксиома} 3^o) \Rightarrow P(A + B) = P(A) + P(B \backslash A)$
		
		\item[б)]
		%РИСУНОК
		$B = (B \backslash A) + (AB)$ \\
		(аксиома $3^o$) $\Rightarrow P(B) = P(B \backslash A) + P(AB) \Rightarrow$ \fbox{$P(B \backslash A) P(B) - P(AB)$} 
		
		\item[в)] 
		Подставляем выражение для $P(B \backslash A)$ из пункта б) в пункт а): $P(A+B) = P(A) + P(B) - P(AB)$
	\end{enumerate}
	
	\item[$6^o$]
	Является обобщение свойства $5^o$ на случай n событий.
\end{enumerate}

Пусть $\Omega$ - пространство элементарных исходов исходов $\beta$ - $\sigma$-алгебра на $\Omega$.

\underline{Определение}: Вероятностью называется отображение $P: \beta \rightarrow \mathbb{R}$, обладающее следующими свойствами:
\begin{enumerate}
	\item[$1^o$]
	$P(A) \geqslant 0$ (аксиома неотрицательности)
	
	\item[$2^o$]
	$P(\Omega) = 1$ (аксиома нормированности)
	
	\item[$3^o$]
	любых попарно-несовместных событий $A_1, A_2, \ldots$ \\
	$P(A_1 + A_2 + \ldots) = P(A_1) + P(A_2) + \ldots$ (расширенная аксиома сложения)
\end{enumerate}

\underline{Замечание}: Иногда вместо расширенной аксиомы сложения $3^o$ рассматривают аксиому 
\begin{enumerate}
	\item[$3')$] 
	для любых попарно несовместных событий $A_1, \ldots , A_n$ \\
	$P(A_1 + \ldots + A_n) = P(A_1) + \ldots + P(A_n)$
	
	\item[$3)$]
	для любой неубывающей последовательности событий $A_1 \subseteq A_2 \subseteq \ldots \subseteq A_n \subseteq \ldots$ справедливо $\lim\limits_{n \rightarrow \infty} P(A_n) = P(A)$, где $A = A_1 + \ldots + A_n + \ldots$ (аксиома непрерывности).
\end{enumerate}

Можно показать, что \\
$3^o \Leftrightarrow
\begin{cases}
	3' \\
	3'' \\
\end{cases}$






































