Пусть D - фигура на плоскости. \\

Как ввести понятие площадт фигуры D? \\

Если D является треугольником (или прямоугольником), то понятие площади очевидно. \\

Если D является многоугольником, то её можно разбить на треугольники, а площадь области D определить как сумму составляющих её треугольников. \\

Что делать, если D - произвольная фигура?

\begin{enumerate}
\item[а)] 
Рассмотрим множество многоугольников m, каждое из которых целиком содержатся в D.\\
Обозначение: $S_* = \sup\limits_{m} S(m)$ \\
m - многоугольник
S(m) - площадь многоугольника m 

\item[б)] 
Рассмотрим множество многоугольников M, каждый из которых целиком содержат в себе D. \\
Обозначение: $S^* = \inf\limits_{M} S(M)$
\end{enumerate}

\underline{Определение:} Область D на плоскости называется квадрируемой, если $\exists$ конечные значения $S_*, S^*$, причем $S_* = S^*$. При этом число $S = S_* = S^*$ называется площадью области D.

\underline{Определение:} Говорят, что множество D точек плоскости имеет площадь нуль, если D можно целиком заключить в многоугольник сколь угодно малой площади, т.е. $\forall \varepsilon > 0$ $\exists$ многоугольник M площади $\varepsilon$ такой, что $D \subseteq M$.


\underline{Пример:}
\begin{enumerate}
\item[1)]
$D = \{A\}, A$ - точка.

\item[2)]
$D = [AB]$ - отрезок.

\item[3)] Спрямляемая (т.е. имебщая конечную длину) кривая.
\end{enumerate}


\underline{Th.}
Пусть D - замкнутая плоская область. \\
Тогда D - квадрируемая $\Leftrightarrow$ граница D имеет площадь D. \\

\underline{Th.}
Пусть L - плоская спрямляемая кривая. \\
Тогда L - имеет площадь нуль. \\


\underline{Следствие:}
Пусть \\
\begin{enumerate}
\item[1)] 
D - область на плоскости

\item[2)]
D - ограничена конечным числом спрямляемых кривых.
\end{enumerate}

Тогда D квадрируема.


\underline{Замечание:} в дальнейшем мы будем рассматривать только квадрируемые области.

















