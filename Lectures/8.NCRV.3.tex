% 3. Математические ожидания и дисперсии некоторых случайных величин.


\textbf{I.} Биномиальная случайная величина $X \sim B(n,p)$ \\
$P\{X = k\} = C^k_n p^k q^{n-k}, \ k = \overline{0,n}$ \\
Найти MX и DX.\\
\begin{enumerate}
	\item[а)] X - число успехов в серии из n испытаний по схеме Бернулли.
	
	\item[б)] Введем случайную величину\\
	$X_i = 
	\begin{cases}
		1, \text{если в i-ом испытании произошел успех} \\
		0, \text{иначе}\\
	\end{cases}$\\
	$i = \overline{1,n}$\\
	Тогда\\
	% Вставить таблицу
	$q = 1 - p$\\
	Ранее $MX_i = p$, $DX_i = pq$, $i = \overline{1,n}$
	
	\item[в)] $X = \sum\limits_{i = 1}^n X_i$\\
	$M[X] = M\left[\sum\limits_{i = 1}^n X_i\right] = \sum\limits_{i = 1}^n MX_i = pn$.\\
	$D[X] = D\left[\sum\limits_{i = 1}^n X_i\right]$ = (в схеме испытании Бернулли отдельные испытания независимы $\Rightarrow$ все $X_i$ независимы) = $\sum\limits_{i = 1}^n DX_i = npq$.\\
\end{enumerate}


\textbf{II.} Пуассоновская случайная величина $X \sim \Pi(\lambda)$.\\
$P\{X = k\} = \cfrac{\lambda^k}{k!} e^{-\lambda}, \ k = 0, 1, 2, \ldots$\\
\begin{enumerate}
	\item[а)] 
	\begin{math} 
	MX = \left(\sum p_i x_i\right) = 
	\sum\limits_{k = 0}^\infty \cfrac{\lambda^k}{k!} e^{-\lambda} \cdot k = 
	e^{-\lambda} \sum\limits_{k = 1} \cfrac{\lambda^k}{k!} \cdot k = 
	e^{-\lambda} \sum\limits_{k = 1}^\infty \cfrac{\lambda^k}{(k - 1)!} = 
	e^{-\lambda} \cdot \lambda \cdot \sum\limits_{k = 1}^\infty \cfrac{\lambda^{k - 1}}{(k - 1)!} = 
	\left[j = k - 1; \ k = 1 \Rightarrow j = 0\right]= 
	e^{-\lambda} \cdot \lambda \cdot \underbrace{\sum\limits_{j = 0}^\infty \cfrac{\lambda^j}{j!}}_{= e^\lambda} = e^{-\lambda} \cdot e^\lambda \cdot \lambda = \lambda
	\end{math}
	Аналогично можно показать, что $DX = \lambda$.
\end{enumerate}


\textbf{III.} Геометрическое распределение.\\
Пусть X имеет геометрическое распределение с параметрами $p \in (0;1)$.\\
то есть $P\{X = k\} = pq^k, \ k = 0, 1, 2, \ldots$\\
Тогда можно показать, что\\
$MX = \cfrac{q}{p}$\\
$DX = \cfrac{q}{p^2}$\\


\textbf{IV.} Равномерное распределение случайной величины.\\
$X \sim P[a,b]$\\
$f(x) = 
\begin{cases}
	\frac{1}{b - a}, \ x \in [a,b]\\
	0, \ \text{иначе}\\
\end{cases}$\\
% Вставить рисунок
$\displaystyle  M[X] = \int\limits_{-\infty}^{+\infty} x f(x) dx = \left(f(x) \equiv 0 \text{ вне } [a,b]\right) = \int\limits_{a}^{b} x \cfrac{1}{b - a} dx = \cfrac{b^2 - a^2}{a(b - a)} = \cfrac{a + b}{2}$\\
$\displaystyle  DX = M\left[\left(X - MX\right)^2\right] = \int\limits_{-\infty}^{+\infty} \left(x - \cfrac{a + b}{2}\right)^2 f(x)dx = \cfrac{1}{b-a} \int\limits_a^b \left(x - \cfrac{a + b}{2}\right)^2 dx = \ldots = \cfrac{(b - a)^2}{12}$\\


\textbf{V.} Экспоненциальное распределение.\\
$\lambda = Exp(\lambda)$\\
$f(x) = 
\begin{cases}
	\lambda e^{-\lambda x}, \ x > 0\\
	0, \ x < 0\\
\end{cases}$\\
% Вставить график

$\displaystyle  MX = \int\limits_{-\infty}^{+\infty} xf(x)dx = \left(f(x) \equiv 0 \text{ при } x < 0\right) = \lambda \int\limits_0^{+\infty} x e^{-\lambda x} dx = - \int\limits_0^{+\infty} x e^{-\lambda x} = -x e^{\lambda x} \bigg|_0^{+\infty} + \int\limits_0^{+\infty} e^{-\lambda x} dx = -\cfrac1\lambda e^{-\lambda x} \bigg|_0^{+\infty} = \cfrac1\lambda$\\
$DX = M[X^2] - (MX)^2$\\
$\displaystyle  M[X^2] = \int\limits_{-\infty}^{+\infty} x^2 f(x) dx = \lambda \int\limits_0^{+\infty} x^2 e^{-\lambda x} dx = \ldots = \cfrac{2}{\lambda^2}$\\
Тогда $DX = \cfrac{2}{\lambda^2} - \left(\cfrac{1}{\lambda}\right)^2 = \cfrac{1}{\lambda^2}$\\


\textbf{VI.} Нормальное распределение: $X \sim N(m, \sigma^2)$\\
$f(x) = \cfrac{1}{\sqrt{2\pi} \sigma} \cdot e^{-\cfrac{(x-m)^2}{2 \sigma^2}}, \ x \in \mathbb{R}$\\
%  Вставить рисунок

$displaystyle  MX \hm=
\int\limits_{-\infty}^{+\infty} xf(x)dx \hm=
\cfrac{1}{\sqrt{2\pi} \sigma} \int\limits_{-\infty}^{+\infty} x \cdot e^{-\cfrac{(x-m)^2}{2 \sigma^2}} \ dx \hm= 
\left\{ \begin{array}{lll}
	\frac{x - m}{\sigma} = t\\
	dx = \sigma dt\\
	x = \sigma t + m\\
\end{array} \right\} \hm= 
\cfrac{1}{\sqrt{2 \pi} \sigma} \cdot \sigma \int\limits_{-\infty}^{+\infty} (\sigma t + n) e^{-\frac{t^2}{2}} \ dt \hm= 
\cfrac{1}{\sqrt{2\pi}} \left[ \sigma \int\limits_{-\infty}^{+\infty} t e^{-\frac{t^2}{2}} dt + m \int\limits_{-\infty}^{+\infty} e^{-\frac{t^2}{2}} dt \right] \hm=
m \cdot \cfrac{1}{\sqrt{2\pi}} \int\limits_{-\infty}^{+\infty} e^{- \frac{t^2}{2}} dt = m$\\
% Вставить пояснения в формулу
$\cfrac{1}{\sqrt{2\pi}} \cdot e^{- \frac{t^2}{2}} = f_{0,1} (t)$ - функция плотности распределения $N(0;1)$\\
$\displaystyle  DX = M\left[(X - MX)^2\right] = \int\limits_{-\infty}^{+\infty} (x - m)^2 f(x) dx = \cfrac{1}{\sqrt{2\pi} \sigma} \int\limits_{-\infty}^{+\infty} (x - m)^2 \cdot e^{- \frac{(x - m)^2}{2 \sigma^2}} \ dx = \left\{ \cfrac{(x - m)}{\sigma} = t \right\} = \cfrac{1}{\sqrt{2\pi} \sigma} \cdot \sigma \cdot \sigma^2 \cdot \int\limits_{-\infty}^{+\infty} t^2 \cdot e^{- \frac{t^2}{2}} dt = 
\cfrac{\sigma^2}{\sqrt{2\pi}} \int\limits_{-\infty}^{+\infty} t \cdot e^{-\frac{t^2}{2}} d\left(\cfrac{t^2}{2}\right) = 
\cfrac{- \sigma^2}{\sqrt{2\pi}} \int\limits_{-\infty}^{+\infty} t de^{- \frac{t^2}{2}} = \left\{\text{по частям}\right\} = 
-\cfrac{\sigma^2}{\sqrt{2\pi}} \left[ t \cdot e^{-\frac{t^2}{2}} bigg|_{-\infty}^{+\infty} - \int\limits_{-\infty}^{+\infty} e^{- \frac{t^2}{2}} dt\right] = 
\sigma^2 \cdot \cfrac{1}{\sqrt{2\pi}} \int\limits_{-\infty}^{+\infty} e^{-\frac{t^2}{2}} dt = \sigma^2$\\
% Вставить пояснения в формулу
\fbox{
Таким образом \newline
если $X \sim N(m, \sigma^2)$ \\
то  $m = MX$ \ $\sigma^2 = DX$ \\
} \\
% Исправить текст в рамке



















