%Определения и свойства двойного интеграла

Пусть D - квадрируемая замкнутая плоская область. \\

\underline{Определение:} Разбиением области D называется множество \\
$R = \left\{D_1, ... , D_n\right\}$, \\
где \\
\begin{enumerate}
\item[1)] 
$D = \bigcup\limits_{i = 1}^{\infty} D_i$

\item[2)]
$int D_i \cap int D_j = \emptyset$ при $i \neq j$ 

\item[3)] 
$D_i$ - квадрируемая, $i = \overline{1,n}$
\end{enumerate}

\underline{Определение:} Диаметром разбиения $R = \left\{D_1, ..., D_n\right\}$ называется число $d(R) = \underset{i = \overline{1,n}}{max diam (D_i)}$. 

Пусть D - квадрируемая замкнутая область на плоскости Oxy \\
$f: D \rightarrow \mathbb{R}$ \\
(f является функцией двух переменных, т.к. D - область на плоскости). \\

\underline{Определение:} Двойным интегралом функции f по области D называется число \\
$\iint\limits_{D} f(x,y)dxdy = \lim\limits_{d(R) \rightarrow 0} \sum f(M_i)\Delta S_i$, где \\
$R = \left\{D_1, ... , D_n\right\}$ - разбиение области D, \\
$M_i \in D_i$, $i = \overline{1,n}$ - ...

\underline{Замечание:} В определении подразумевается, что указанный предел $\exists$, конечен и не зависит от разбиения R области D и ...


\underline{Свойства двойного интеграла} \\
\begin{enumerate}
\item[$1^o$]
$\iint\limits_{D} 1 dxdy = S(D)$

\item[$2^o$] 
Линейность \\
Если $f,g$ - интегрируемы в D ..., то
	\begin{enumerate}
	\item[а)] 
	$f \pm g$ также интегрирума в D, причем \\
	$\iint\limits_{D} (f \pm g) dxdy = \iint\limits_{D} fdxdy \pm \iint\limits_{D} gdxdy$
	
	\item[б)]
	$c \cdot f$, $c = const$, также интегрируема в D, причем \\
	$\iint\limits_{D} c \cdot f dxdy = c \iint\limits_{D} f dxdy$
	\end{enumerate}
	
\item[$3^o$] 
Аддитивность \\
Пусть 
	\begin{enumerate}
	\item[1)] 
	$D_1, D_2$ - плоские квадрируемые области
	
	\item[2)]
	f интегрируема в $D_1$ и в $D_2$
	
	\item[3)]
	$int D_i \cap int D_j = \emptyset$
	\end{enumerate}
	
Тогда f интегрируема в $D = D_1 \cup D_2$, \\
$\iint\limits_{D} fdxdy = \iint\limits_{D_1} fdxdy + \iint\limits_{D_2} fdxdy$ 

\item[$4^o$] 
О сохранении интегралом знака функции \\
Пусть 
	\begin{enumerate}
	\item[1)]
	$f(x,y) \geqslant 0$ в D
	
	\item[2)]
	f интегрируема в D
	\end{enumerate}
 
Тогда $\iint\limits_{D} f(x,y) dxdy \geqslant 0$


\item[$5^o$]
Пусть
	\begin{enumerate}
	\item[1)] 
	$f(x,y) \geqslant f(x,y)$ в D
	
	\item[2)]
	$f, g$ интегрируемы в D
	\end{enumerate}
	
Тогда $\iint\limits_{D} fdxdy \geqslant \iint\limits_{D} gdxdy$


\item[$6^o$] 
th об оценке модуля двойного интеграла. \\
Пусть f интегрируемы в D. \\
Тогда $|f|$ также интегрируема в D, причем \\
$\left| \iint\limits_{D} fdxdy \right| \leqslant \iint\limits_{D} \left|f\right| dxdy$


\item[$7^o$]
th об оценке двойного интеграла (обобщенная th). \\
Пусть
	\begin{enumerate}
	\item[1)] 
	$f,g$ - интегрируема в D
	
	\item[2)] 
	$m \leqslant f(x,y) \leqslant M$ в D
	
	\item[3)]
	$g(x,y) \geqslant 0$ в D
	\end{enumerate}
	
Тогда $m \iint\limits_{D} g(x,y) dxdy \leqslant \iint\limits_{D} f(x,y) g(x,y) dxdy \leqslant M \iint\limits_{D} g(x,y) dxdy$

\underline{Следствие:} Если $g(x,y) = 1$ в D, то получаем "просто" th об оценке двойного интеграла: \\
$m \cdot S \leqslant \iint f(x,y) dxdy \leqslant M \cdot S$, \\
где $S = S(D)$.


\item[$8^o$] 
th о среднем значении. \\
\underline{Определение:} Средним значением функции f области D называется число \\
$<f> = \frac{1}{S(D)} \iint\limits_{D} f(x,y) dxdy$

\underline{Свойство:} Пусть \\
	\begin{enumerate}
	\item[1)] 
	D - линейносвязная замкнутая область (т.е. граница D является связным множеством)
	
	\item[2)] 
	f непрерывна в D
	\end{enumerate}

Тогда $\exists M_0 \in D$ такая, что $f(M_0) = <f>$


\item[$9^o$]
Обобщенная th о среднем значении. \\
Пусть
	\begin{enumerate}
	\item[1)] 
	f - непрерывна
	
	\item[2)]
	g - интегрируема
	
	\item[3)]
	g знакопостоянна
	
	\item[4)]
	D линейно связное множество
	\end{enumerate}

Тогда
\end{enumerate}
























