% 7. Приложения двойного интеграла.

I. Вычисление площади плоской фигуры. \\
$S(D) = \iint\limits_{D} 1dxdy$. \\
(Свойство $1^o$ двойного интеграла) \\



II. Вычисление массы пластины. \\
Пусть \\
\begin{enumerate}
	\item[1)] 
	Пластина занимает область D на плоскости Oxy
	
	\item[2)]
	$f(x,y)$ - значение плотности
\end{enumerate}

Тогда масса этой пластины \\
$M = \iint\limits_{D} f(x,y) dxdy$



III. Вычисление объема тела. \\

Пусть тело Т: \\
$T = \left\{ \left(x,y,z\right) : (x,y) \in D_{xy}, z_1(x,y) \leqslant z \leqslant z_2(x,y) \right\}$ \\
$V(T) = \iint\limits_{D_{xy}} \left[ z_2(x,y) - z_1(x,y) \right] dxdy$