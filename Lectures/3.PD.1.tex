% Случайные эксперименты.

\underline{Определение}: Случайным называется эксперимент результат которого невозможно предсказать.

\underline{Пример}:
\begin{enumerate}
	\item[1)]
	Подбрасывают монету. Возможные исходы:
	$\Omega = \{ \text{Г,Р} \}$
	где Г - выпадение герба, Р - выпадение решки.
	$| \Omega | = 2$
	
	\item[2)]
	Бросают игральную кость. Наблюдают результат: число выпавших ...
	$\Omega = \{ 1,2,3,4,5,6 \}$ \\
	$| \Omega | = 6$
	
	\item[3)]
	Бросают монету до первого появления герба. Наблюдают результат - количество бросков ...
	$\Omega = {1,2,3, ...} = \mathbb{N}$ \\
	$|\Omega| = $... \\
	($\Omega$ является ...
	
	\item[4)] 
	В больнице измеряют температуру случайно выбранного пациента.
	$\Omega = [33, 42]$ \\
	$|\Omega| = c$ \\
	($\Omega$ имеет мощность континуума).
	
	\item[5)]
	Производят стрельбу по плоской мишени размеры которой 1м x 1м. Набл. результат - координаты (x,y). \\
	... \\
	$\Omega = \left\{ (x,y): |x| \leqslant \frac{1}{2}; |y| \leqslant \frac{1}{2} \right\}$ \\
	$|\Omega| = c$ \\
	

\end{enumerate}

\underline{Определение}: Множество $\Omega$ всех исходов данного случайных экспериментов называется пространством элементарных исходов.
	
\underline{Замечание}: При рассмотрении пространства элементарных исходов предполагает: что
\begin{enumerate}
	\item[1)] 
	Каждый элементарный исход неделим, так как не может быть разложен на более мелкие исходы.
	
	\item[2)]
	В результате случайного эксперимента всегда происходит ровно один элементарный исход из $\Omega$.
\end{enumerate}
	
\underline{Определение (нестрогое)}: Событием называется (любое) множество множества $\Omega$.

\underline{Определение}: Говорят, что в результате случайного эксперимента произошло событие А, если в результате этого эксперимента произошел один из входящих в А элементарных исходов.

\underline{Пример}: Бросают игральную кость. \\
$\Omega = \{ 1,2, ... ,6\}$ \\
$A = \{ \text{выпало четное число} \} = \{2,4,6\}$ \\
Если выпало 2 очка, то наступило А.

\underline{Определение}: Событие А называют следствием события В, если наступление события В влечет наступление события А, то есть $B \subseteq A$.

% Рисунок

\underline{Замечание}: Любое множество $\Omega$ содержит в ... подмножество $\emptyset$ и $\Omega$. Соответствующее события называются невозможными ($\emptyset$) и достоверными ($\Omega$). Оба этих события называются ...

\underline{Пример}: В урне находится 2 красных и 3 синих шара. \\
A = \{ извлеченный шар - зеленый \} = $\emptyset$. \\
B = \{ извлеченный шар - синий \} = $\Omega$

























