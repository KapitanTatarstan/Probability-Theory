% Вычисление двойного интеграла

\underline{Определение:} Область D на плоскости Oxy называется y-правильной, если любая прямая, $\parallel$-ая Oy, пересекает границу D не более чем в 2-х точках, либо содержит участок границы целиком.

\underline{Замечание:} \\
\begin{enumerate}
	\item[1)] 
	y-правильная область можно задать в виде: \\
	$D = \left\{ \left(x,y\right): a \leqslant x \leqslant b, \varphi_1(x) \leqslant y \leqslant \varphi_2(x) \right\}$ $(*)$ \\
	
	\item[2)]
	x- правильная область определяется аналогично.
\end{enumerate}


\underline{Th:} Пусть \\
\begin{enumerate}
	\item[1)]
	$\exists \iint\limits_{D} f(x,y) dxdy = I$
	
	\item[2)]
	D является y-правильной и задается соотношением  $(*)$.
	
	\item[3)]
	$\forall x \in [a,b]$ \\
	$\exists \int\limits_{\varphi_1(x)}^{\varphi_2(x)} f(x,y) dy = F(x)$
\end{enumerate}

Тогда 
\begin{enumerate}
	\item[1)] 
	$\exists$ повторный интеграл \\
	$\int\limits_{a}^{b} dx \int\limits_{\varphi_1(x)}^{\varphi_2(x)} f(x,y) = I_\text{повт.}$
	
	\item[2)]
	$I = I_\text{повт.}$
\end{enumerate}


\underline{Замечание:} Если область D не является правильной в направлении какой-нибудь из координатных осей, то её можно разбить на правильные части и воспользоваться свойством аддитивности двойного интеграла.














