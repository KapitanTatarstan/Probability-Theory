% Задачи о вычислении массы тела.
Пусть 
\begin{enumerate}
	\item[1)] 
	Тело T занимает области $G \subseteq \mathbb{R}^3$
	
	\item[2)]
	$f(x,y,z) \geqslant 0$ - значения плотности материала этого тела в точке $(x,y,z)$
\end{enumerate}

Требуется найти массу $m(T)$ тела T. \\

\begin{enumerate}
	\item[1)]
	Разобъем область G на части: \\
	$G = \bigcup\limits_{i = 1}^{n} G_i$ \\
	$int G_i \cap int G_j = \emptyset$ при $i \neq j$
	
	\item[2)] 
	В пределах каждой из побобластей выберем ...
	
	
	
	$G_i$ ... масса: \\
	$\Delta m_i = m(G_i) \approx f(M_i) \Delta V_i$ \\
	где $\Delta V_i = V(G_i)$, \\
	$\Delta m_i$ - масса части тела, занимающего подобласть  $G_i$
	
	
	\item[4)]
	 Масса тела T тогда: \\
	 $m(T) = \sum\limits_{i = 1}^{n} \Delta m_i \simeq \sum\limits_{i = 1}^{n} f(M_i) \Delta V_i$
	 
	 \item[5)]
	 Эта формула тем точнее, чем меньше размеры $G_i$, поэтому естественно перейти к пределу: \\
	 $m(T) = \lim\limits_{max diam (G_i) \rightarrow 0} \sum\limits_{i = 1}^{n} f(M_i) \Delta V_i$
\end{enumerate}

