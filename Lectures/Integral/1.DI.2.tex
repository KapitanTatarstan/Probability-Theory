% Задачи, приводящие к понятию двойного интеграла.

$I.$ Задача об объеме цилиндрического тела. \\

Пусть D - область на плоскости Oxy. \\
$f: D \rightarrow \mathbb{R}$ - функция определенная на множестве D. \\
$f(x,y) \geqslant 0$, $(x,y) \in D$ \\
 
Рассмотрим тело T, которое ограничено ... 

Разобъем область D на непересекющиеся части \\
$D = \bigcup\limits_{i=1}^{n} D_i$ \\
$int D_i \cap int D_j = \emptyset$ при $i \neq j$ $(*)$ \\
$int D_i$ - множество внутренних точек области $D_i$. \\

Условие $(*)$ означает, что различные элементы имеют общи.. внутренние точек.


\begin{enumerate}
\item[2)] 
Выберем точку $M_i \in D_i$, $i = \overline{1,m}$

\item[3)] 
Считая, что размеры подобласти $D_i$ малы, причем \\
$\Delta V_i \approx f(M_i) \Delta S_i$, \\
где $\Delta S_i = S(D_i)$. \\
$\Delta V_i$ - объем той части тела, которая ... под $D_i$.
\end{enumerate}


Тогда объем тела T \\
$V = \sum\limits_{i=1}^{n} \Delta V_i \approx \sum\limits_{i=1}^{n} f(M_i) \Delta S_i$ \\
Эта формула тем точнее, чем меньше размеры $D_i$, поэтому естественно перейти к пределу: \\
$ V = \lim\limits_{\underset{i = \overline{1,m}}{max diam (D_i) \rightarrow 0}} \sum\limits_{i = 1}^{n} f(M_i) \Delta S_i$. \\
Обозначим $(D) = \sup\limits_{M,N \in D} \left| \underline{MN} \right|$ - диаметр множества D.




$II.$ Задача о вычислении массы пластины. \\ \\

Пусть
\begin{enumerate}
\item[1)] 
пластина занимает область D на плоскости;

\item[2)] 
$f(x,y) \geqslant 0$ - плотность (поверхностного) материала пластины в точке $M(x,y)$.
\end{enumerate}


\begin{enumerate}
\item[1)]
Разобъем область D на непересекающиеся части $D_i$: \\
$D = \bigcup\limits_{i = 1}^{n} D_i$ \\
$int D_i \cap int D_j = \emptyset$ при $i \neq j$;

\item[2)]
В пределах $D_i$ выберем точку $M_i, i = \overline{1,n}$

\item[3)] 
Считая, что размеры $D_i$ малы, можно принять, что в пределах каждой из областей $D_i$ плотность пластины меняется незначительно, поэтому во всех точках области $D_i$ плотность $\approx f(M_i)$.

Тогда масса части $D_i$: \\
$\Delta m_i \approx f(M_i) \Delta S_i$, где $\Delta S_i = S(D_i), i = \overline{1,m}$

\item[4)]
Тогда масса всей пластины: \\
$m = \sum\limits_{i = 1}^{n} \Delta m_i \approx \sum\limits_{i = 1}^{n} f(M_i) \Delta S_i$ \\
Получается формула тем точнее, чем меньше размеры $D_i$, \\
$m = \lim\limits_{\underset{i = \overline{1,n}}{max diam D_i \rightarrow 0}} \sum\limits_{i = 1}^{n} f(M_i) \Delta S_i$
\end{enumerate}















